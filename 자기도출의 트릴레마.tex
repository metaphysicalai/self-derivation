\documentclass[12pt,a4paper]{article}

% --- 기본 패키지 ---
\usepackage[utf8]{inputenc}
\usepackage[T1]{fontenc}
\usepackage{kotex}
\usepackage{amsmath,amssymb,amsthm}
\usepackage{geometry}
\usepackage{setspace}
\usepackage{titlesec}
\usepackage[dvipsnames]{xcolor}
\usepackage[hidelinks]{hyperref}
\usepackage{graphicx}
\usepackage{booktabs}
\usepackage{array}
\usepackage{tikz}
\usetikzlibrary{arrows.meta, shapes, positioning}

% --- 문서 설정 ---
\geometry{a4paper, margin=2.5cm}
\onehalfspacing

% 정리 환경
\newtheorem{definition}{정의}[section]
\newtheorem{proposition}{명제}[section]
\newtheorem{theorem}{정리}[section]
\newtheorem{principle}{원리}[section]

\title{\textbf{자기도출의 트릴레마}\\[0.5ex]
\large 정당화 너머의 확신과 지식의 구조적 한계\\[1.5ex]
{\normalsize The Self-Derivation Trilemma:\\
Conviction Beyond Justification and the Structural Limits of Knowledge}}

\author{임지백 \\
한양대학교 일반대학원 철학과}

\date{2025년 2월}

\begin{document}

\maketitle

\begin{abstract}
\noindent
완전한 지식 체계는 가능한가? 뮌하우젠 트릴레마는 궁극적 정당화가 불가능함을 보여준다. 무한퇴행, 순환논증, 독단적 중단---정당화를 시도하는 한, 이 세 가지 막다른 길을 피할 수 없다. 그렇다면 정당화가 불가능한 세계에서, 강력한 확신은 어떻게 산출되는가?

본 논문은 정당화 게임의 \emph{외부}에서 작동하는 확신 생성 메커니즘으로 \textbf{자기도출}(Self-Derivation)을 제안한다. 이는 이론의 내용($M_{\text{subject}}$)으로부터 이론의 존재($M_{\text{object}}$)를 도출하는 구조다. 아브라함 계열 유신론에서 이 구조가 작동한다: ``인격신이 있다''는 내용을 전제하면 경전의 존재가 예측되고, 경전은 실제로 존재한다. 부정 불가능한 존재(텍스트)로의 착지가 내용에 대한 확신으로 전이되는 이 메커니즘을 본 논문은 \textbf{준-정당화}(Quasi-Justification)라 명명한다. 준-정당화는 믿음을 \emph{정당화}하지 않지만, 믿음에 대한 \emph{확신}을 산출한다---이것은 인식론적 성취가 아니라 인식적 현상이다.

그러나 과학의 언어로 이 준-정당화 조건을 충족하려는 시도($M_{\text{unified}}$)는 새로운 구조적 긴장에 봉착한다. 본 논문은 이를 자기도출 트릴레마로 정식화한다. 토대적 지식 체계에서 단순성, 외부 설명력, 자기도출이라는 세 조건 사이에는 구조적 긴장이 존재하며, 현존하는 체계들은 이 중 하나 이상을 희생한다. 과학은 자기도출을 유예함으로써 설명력을 얻었고, 종교는 설명력을 희생함으로써 자기도출을 얻었다.

뮌하우젠 트릴레마가 정당화의 `논리적 불가능성'을 보였다면, 자기도출 트릴레마는 준-정당화조차 `구조적 상충 관계(Trade-off)' 속에 있음을 보인다. 완벽한 토대는 없다. 서로 다른 비용을 치르는 전략적 선택들이 공존할 뿐이다.

\vspace{0.3cm}
\noindent
\textbf{주요어:} 자기도출, 준-정당화, 뮌하우젠 트릴레마, 자기도출 트릴레마, 아브라함 계열 유신론
\end{abstract}

\newpage
\tableofcontents
\newpage


%============================================================
\section{서론}
%============================================================

\subsection{정당화의 한계와 남겨진 문제}

한스 알베르트(Hans Albert)의 뮌하우젠 트릴레마는 인식론의 근본적 난제를 정식화한다.\footnote{Albert, H. (1968). \emph{Traktat über kritische Vernunft}. Tübingen: Mohr. 이 트릴레마는 아그리파의 오양상(five modes)으로 거슬러 올라가며, 섹스투스 엠피리쿠스의 \emph{Outlines of Pyrrhonism}에서 고전적 형태를 발견할 수 있다.} 어떤 명제의 정당화를 시도할 때, 우리는 세 가지 막다른 길 중 하나에 도달한다: 무한퇴행, 순환논증, 또는 독단적 중단. 칸트의 선험철학은 이 난제를 회피하려는 가장 정교한 시도였다. 경험의 가능조건 자체로부터 지식의 토대를 끌어내려는 전략이었으나, 비유클리드 기하학의 정합성 증명(힐베르트, 1899)과 일반상대성이론(아인슈타인, 1916)은 유클리드 기하학의 선험적 필연성이라는 칸트적 전제를 붕괴시켰다.\footnote{칸트 수학철학에 대한 비유클리드 기하학의 함의에 관해서는 논쟁이 있다. 일부 학자들은 칸트의 초월적 관념론이 유클리드-비유클리드 모두와 양립 가능하다고 주장한다. 그러나 최소한, 유클리드 기하학이 경험적 공간의 유일한 기하학이라는 칸트의 주장은 유지되기 어렵다.}

뮌하우젠 트릴레마 이후, 인식론은 대체로 ``궁극적 정당화''라는 프로젝트 자체를 포기했다. 토대주의(foundationalism)는 특정 믿음들이 비추론적으로 정당화된다고 주장하지만, 이는 트릴레마의 독단적 중단에 해당한다. 정합주의(coherentism)는 믿음들 간의 상호 지지를 정당화로 삼지만, 이는 순환논증의 세련된 변형이다. 무한주의(infinitism)는 무한퇴행을 수용하려 하지만, 유한한 인지 주체에게 무한한 정당화 사슬은 완결될 수 없다.

그런데 주목할 만한 사실이 있다. 종교적 믿음 체계는 철학적 정당화의 부재에도 불구하고, 단순히 작동하는 것을 넘어 \emph{강력한} 확신을 산출해왔다. 이 확신은 ``좋은 삶의 지침''이라는 실용적 가치로 환원되지 않는다. 순교를 가능하게 하는 절대적 확신, 세속적 논증으로 흔들리지 않는 불가침의 믿음---이것은 어디서 오는가? 단순히 비합리성이나 인지적 편향으로 설명하기에는, 그 구조가 너무 체계적이고 그 역사가 너무 지속적이다.


\subsection{자기도출: 정당화 이후의 확신 생성 메커니즘}

본 논문은 종교적 확신의 구조를 분석함으로써, 정당화와 구별되는 인식적 메커니즘을 식별하고자 한다. 이 메커니즘을 본 논문은 \textbf{자기도출}(Self-Derivation)이라 명명한다. 본 논문의 분석은 아브라함 계열 유신론---기독교, 이슬람, 유대교---에 초점을 맞춘다. 이 선택은 두 가지 근거에 기반한다. 첫째, 경험적 대표성이다. 퓨리서치센터(Pew Research Center)의 2025년 보고서에 따르면, 2020년 기준 기독교(28.8\%)와 이슬람(25.6\%)은 세계 인구의 절반 이상을 차지하며, 유대교(0.2\%)를 포함한 아브라함 계열 종교는 전체 종교 인구의 약 72\%에 해당한다.\footnote{Pew Research Center (2025). ``How the Global Religious Landscape Changed from 2010 to 2020.'' 이 비율은 무종교 인구(24.2\%)를 제외한 수치다.} 둘째, 구조적 동형성이다. 이들 전통은 ``인격신 $\to$ 계시 $\to$ 경전''이라는 공통 구조를 공유하며, 자기도출의 작동 방식을 가장 명확하게 보여준다.\footnote{불교, 힌두교 등 비아브라함 종교에서는 다른 메커니즘이 작동할 수 있다. 예컨대 불교의 사성제와 팔정도는 인격신의 계시가 아닌 붓다의 깨달음에 기반하며, 텍스트의 권위 구조가 상이하다. 이에 대한 분석은 별도의 연구를 요한다.}

자기도출의 구조는 다음과 같다. 이론 $M$은 두 측면을 갖는다: 이론이 \emph{말하는} 내용($M_{\text{subject}}$)과 이론이 \emph{존재한다는} 사실($M_{\text{object}}$). 자기도출은 전자로부터 후자가 도출되는 구조다. 종교적 텍스트에서 이 구조가 작동한다. ``창조물을 사랑하는 인격신이 있다''는 내용($M_{\text{subject}}$)을 전제하면, 그 신은 자기를 알리고자 할 것이고, 따라서 계시가 있을 것이며, 그 계시를 담은 텍스트가 존재할 것이라는 예측이 따라 나온다. 그리고 실제로 그런 텍스트---성경, 쿠란---가 존재한다($M_{\text{object}}$).

이것은 순환논증과 다르다. 순환논증은 내용들 사이에서 돈다: ``신이 존재한다, 왜냐하면 성경에 그렇게 쓰여 있으니까; 성경은 진리다, 왜냐하면 신의 계시이므로.'' 모든 항이 검증 불가능한 명제적 주장이며, 외부의 착지점이 없다. 반면 자기도출에서 $M_{\text{object}}$---텍스트의 존재---는 부정 불가능한 경험적 사실이다. 성경이 존재한다는 것은 논쟁의 대상이 아니다. 자기도출은 명제적 내용으로부터 경험적 존재로의 범주 전환을 포함하며, 이 전환이 착지점을 제공한다.

물론 자기도출은 정당화가 아니다. 이론의 내용이 참임을 증명하지 않는다. 자기도출은 정당화 게임 \emph{안에서} 경쟁하는 제4의 옵션이 아니라, 정당화 게임 \emph{이후}에 작동하는 별개의 메커니즘이다. 본 논문은 이를 \textbf{준-정당화}(quasi-justification)라 부른다. 준-정당화는 믿음을 정당화하지 않지만, 믿음에 대한 확신을 산출한다. 이것은 규범적 인식론의 범주가 아니라 인식적 현상의 기술이다.

본 논문의 분석은 순수하게 기술적(descriptive)이다. 자기도출이 ``좋다'' 또는 ``나쁘다''고 평가하지 않으며, 종교적 믿음이 ``정당하다'' 또는 ``부당하다''고 주장하지 않는다. 본 논문이 수행하는 것은 확신이 산출되는 \emph{구조}의 분석이다. 이 분석이 의미 있는 이유는 두 가지다. 첫째, 종교적 확신이 왜 외부의 비판에 강건한지를 설명함으로써, 종교-세속 대화의 구조적 조건을 이해하는 데 기여한다. 둘째, 과학이 자기도출을 달성하려 할 때 어떤 한계에 직면하는지를 보여줌으로써, 만물의 이론(Theory of Everything)에 대한 메타이론적 성찰을 제공한다.

이러한 기술적 분석은 규범적 인식론이 아니지만, 철학적으로 중요하다. 첫째, 기존 인식론이 간과한 확신 생성 메커니즘을 \emph{형식적으로} 포착한다. 정당화의 성공 여부만 묻던 인식론에, ``정당화 없이 확신이 어떻게 가능한가''라는 질문을 추가한다. 둘째, 과학과 종교의 관계를 \emph{구조적 trade-off}로 설명함으로써, 환원주의적 해석(``종교는 비합리적이다'')과 호교론적 해석(``종교도 합리적이다'') 모두를 넘어서는 제3의 관점을 제공한다.


\subsection{문제 제기: 과학적 자기도출은 가능한가}

과학은 이 구조를 갖추고 있지 않다. $F=ma$라는 법칙은 입자의 운동을 함축하지만, 뉴턴의 존재를 함축하지 않으며, \emph{Principia}라는 텍스트의 존재를 함축하지 않는다. 물리 법칙의 \emph{내용}으로부터 그 법칙을 기술한 \emph{텍스트}의 존재로 가는 경로가 없다. 만물의 이론이 완성된다 해도, 그 이론은 만물을 설명하면서 자기 자신의 표상---이론을 담은 논문, 이론을 이해하는 물리학자---이 왜 존재하는지는 설명하지 못한다.

이것은 결함인가, 전략적 선택인가? 그리고 과학의 언어로 자기도출을 달성하는 것이 원리적으로 가능한가?

본 논문은 이 질문을 탐구한다. 구체적으로, 다음 세 가지를 수행한다.

첫째, 자기도출 개념을 정립한다. 자기도출은 순환논증, 자기참조, 자기예언, 자기복제 등 유사 개념들과 구분되는 고유한 구조를 갖는다(3절). 이 구조가 뮌하우젠 트릴레마의 세 옵션과 어떻게 다른지를 명확히 한다.

둘째, 종교의 자기도출 구조가 종교적 확신을 어떻게 산출하는지 분석한다(4절). 텍스트 절대성의 스펙트럼에 따라 자기도출의 강도가 달라지며, 이것이 확신의 강도와 상관한다는 가설을 제시한다.

셋째, 과학의 언어로 자기도출을 달성하려는 시도---$M_{\text{unified}}$---를 구성하고, 그 한계를 분석한다(5--6절). 이 분석은 단순성, 외부 설명력, 자기도출이라는 세 조건을 동시에 달성하는 것이 불가능하다는 \textbf{자기도출 트릴레마}로 귀결된다.

뮌하우젠 트릴레마가 정당화의 논리적 불가능성을 보여준다면, 자기도출 트릴레마는 준-정당화조차 구조적 상충 관계(trade-off) 속에 있음을 보인다. 완벽한 토대는 없다. 있는 것은 서로 다른 비용을 치르는 전략적 선택들뿐이다.


%============================================================
\section{정당화의 한계}
%============================================================

\subsection{뮌하우젠 트릴레마}

어떤 명제 $A$를 정당화하려 한다고 하자. ``왜 $A$인가?''라는 질문에 답해야 한다.

한스 알베르트(Hans Albert, 1968)의 뮌하우젠 트릴레마는 이 시도가 세 가지 막다른 길 중 하나에 빠진다고 지적한다.\footnote{Albert, H. (1968). \emph{Traktat über kritische Vernunft}. Tübingen: Mohr. 이 논증은 섹스투스 엠피리쿠스가 보고한 아그리파의 오양상(five modes)으로 거슬러 올라간다. 다만 고대 피론주의가 판단중지(epoché)를 목표로 했다면, 알베르트는 가오류주의(fallibilism)를 옹호한다는 점에서 결론이 다르다.}

\textbf{무한퇴행}: $A$를 정당화하기 위해 $B$를 제시한다. ``왜 $B$인가?''라는 질문이 따라온다. $B$를 정당화하기 위해 $C$를 제시한다. 이 과정은 끝없이 이어진다.
\begin{equation}
A \leftarrow B \leftarrow C \leftarrow D \leftarrow \cdots
\end{equation}

\textbf{순환논증}: 어느 시점에서 이미 사용한 명제로 돌아온다. $A \leftarrow B \leftarrow C \leftarrow A$. 정당화가 순환을 이룬다. 아무것도 정당화하지 못한다.

\textbf{독단적 중단}: 어느 시점에서 ``$X$는 그냥 참이다''라고 선언하고 멈춘다. 근거 없는 선언이다.

이 셋 중 어느 것도 만족스럽지 않다. 무한퇴행은 완결되지 않고, 순환논증은 공허하며, 독단적 중단은 자의적이다.

그러나 알베르트의 결론은 회의주의적 판단중지가 아니다. 확실한 정당화가 불가능하다는 것은, 탐구를 포기하라는 것이 아니라 \textbf{확실성의 이상을 포기하라}는 것이다. 알베르트는 포퍼의 비판적 합리주의를 따라, 모든 지식은 원리적으로 오류 가능하며(fallible), 비판적 시험과 반증을 통해 점진적으로 개선될 수 있다고 주장한다. 교조주의---비판에 면역된 명제를 설정하는 것---가 문제이지, 잠정적 수용 자체가 문제는 아니다.


\subsection{칸트: 자연과학 토대의 확립 시도}

뮌하우젠 트릴레마는 1968년에 정식화되었지만, 그것이 포착하는 문제---지식의 궁극적 근거는 무엇인가---는 오래되었다. 근대 철학에서 이 문제를 가장 체계적으로 해결하려 한 것이 칸트의 선험철학이다.

칸트의 전략은 독창적이었다. 그는 ``경험의 가능조건 자체''에서 정당화를 끌어내려 했다. 시공간과 인과성은 우리가 세계를 경험하기 위해 \emph{반드시} 전제해야 하는 형식이다. 이것 없이는 경험 자체가 불가능하다. 따라서 유클리드 기하학과 뉴턴역학은 단순히 경험적으로 참인 것이 아니라, 경험의 조건 자체에 의해 \emph{필연적으로} 참이다.

뮌하우젠의 프레임으로 소급하면, 칸트는 세 옵션을 모두 회피하려 했다:
\begin{itemize}
    \item 무한퇴행이 아니다---경험의 조건에서 멈추므로
    \item 순환논증이 아니다---조건이 경험을 가능하게 하는 일방향 관계이므로
    \item 독단이 아니다---``그냥 참''이 아니라 경험의 필연적 조건이므로
\end{itemize}

그러나 칸트의 ``부정 불가능''은 경험 주체에게 부과된 \emph{초월론적} 불가능성이었다. 유클리드 기하학이 아닌 공간은 경험하는 존재인 한에서 \emph{생각할 수 없다}고 주장했다. 이것이 문제였다. 비유클리드 기하학이 등장하자, 생각할 수 없다던 것이 생각 가능해졌다. 상대성이론이 등장하자, 필연적이라던 것이 경험적으로 거짓이 되었다.

칸트의 실패가 가르치는 것: 초월론적 필연성에 기반한 정당화도 경험에 의해 뒤집힐 수 있다. 칸트 이후, 정당화는 더 이상 필연적 토대의 확립이라는 형태로는 유지되지 않았고, 작동 방식의 분석으로 전환되었다. 현대 과학철학---논리실증주의, 포퍼, 쿤, 라카토스---은 ``과학은 왜 참인가''를 묻지 않는다. 대신 ``과학은 어떻게 작동하는가''를 묻는다.


\subsection{각 지식 체계의 대응}

현존하는 지식 체계들은 뮌하우젠 트릴레마에 어떻게 대응하는가?

\textbf{과학}은 무한퇴행을 \textbf{유예}한다. ``왜 $F=ma$인가?''라는 질문에 과학은 답하지 않는다. 더 근본적인 법칙으로 환원하는 시도는 있다---뉴턴역학을 상대론으로, 상대론을 끈이론으로. 그러나 가장 근본적인 법칙에 대해서는 ``왜''가 유예된다. 과학은 ``어떻게''에 답하지, ``왜''에 답하지 않는다.

\textbf{수학}은 독단적 중단을 \textbf{명시적으로 채택}한다. 공리는 증명 없이 받아들여진다. 그러나 수학은 이것을 문제로 보지 않는다. 공리의 ``참''을 주장하는 것이 아니라, ``공리가 참이면 $X$가 따라 나온다''를 탐구하기 때문이다. 수학은 정당화 게임 자체를 거부한다.

\textbf{종교}는 흔히 순환논증으로 비판받는다. ``신은 존재한다. 왜냐하면 성경에 그렇게 쓰여 있으니까. 그리고 성경은 진리다. 왜냐하면 신의 계시이므로.'' 신 $\to$ 성경 $\to$ 신. 이것은 내용들 사이의 순환이다.

그러나 이 비판은 종교가 작동하는 방식을 완전히 설명하지 못한다. 종교는 순환논증임에도 불구하고 강력하게 작동해왔다. 단순히 ``오류''라고 치부하기엔, 그 영향력이 너무 크고 지속적이다. 무언가 다른 메커니즘이 있어야 한다.


\subsection{남은 질문}

뮌하우젠 트릴레마는 정당화의 불가능성을 보여준다. 칸트의 실패는 이 불가능성이 회피될 수 없음을 역사적으로 확인해준다.

그런데 흥미로운 사례가 있다. 뮌하우젠 남작은 자기 머리카락을 잡아당겨 늪에서 빠져나왔다는 허풍으로 유명하다. 영어권에서는 이 이야기의 변형으로 ``pull oneself up by one's bootstraps''---자기 부츠끈을 잡아당겨 스스로를 들어올린다---라는 표현이 생겼다. 둘 다 같은 메타포다: 자기 자신으로 자기를 들어올리는 것은 불가능하다. 뮌하우젠 트릴레마는 이 메타포에서 이름을 따왔다. 그런데 컴퓨터과학의 ``부트스트랩''은 같은 메타포를 정반대 의미로 사용한다---자기 자신으로 자기를 들어올리는 것을 \emph{실제로 한다}.

셀프-호스팅 컴파일러는 겉보기에 순환처럼 보인다. C 컴파일러는 C로 작성되어 있다. 그런데 C로 작성된 코드를 실행하려면 C 컴파일러가 필요하다. C 컴파일러를 만들려면 이미 C 컴파일러가 있어야 한다---전형적인 닭과 달걀 문제다. 그러나 실제로는 순환이 아니라 단계적 체인이고, \emph{작동한다}. 물론 이것은 뮌하우젠의 인식론적 문제를 직접 해결하지 않는다. 부트스트랩은 인과적 과정이고, 뮌하우젠이 다루는 것은 논리적 정당화다. 범주가 다르다. 그러나 이것이 시사하는 바가 있다: 겉보기 순환이 반드시 막다른 길은 아닐 수 있다. 순환처럼 보이지만 착지점이 있는 구조가 가능하다.

그리고 종교가 있다. 종교는 정당화 없이도 \emph{강력하게} 작동한다---단순히 ``좋은 삶의 지침''을 제공하는 수준이 아니라, 강도 높은 확신까지 가능하게 한다. 어떻게 가능한가?

칸트가 이성의 선험적 구조를 출발점으로 정당화를 시도했다면, 본 논문은 이론의 \emph{존재} 자체를 출발점으로 정당화의 \emph{대안}을 탐색한다. 본 논문은 \textbf{준-정당화}(quasi-justification)의 한 형식으로 \textbf{자기도출}을 제안한다.


%============================================================
\section{자기도출: 준-정당화의 구조}
%============================================================

\subsection{객관적 주관성: $M_{\text{subject}}$와 $M_{\text{object}}$}

모든 이론 $M$은 두 측면을 갖는다.

\begin{definition}[객관적 주관성]
이론 $M$의 두 측면:
\begin{itemize}
    \item $M_{\text{subject}}$: 이론의 내용. 이론이 \emph{말하는} 것. (주관적 주장)
    \item $M_{\text{object}}$: 이론의 존재. 이론이 \emph{있다는} 것. (객관적 사실)
\end{itemize}
\end{definition}

이것은 이론이 ``자기를 언급한다''는 것이 아니다. 이론이면 필연적으로 갖는 두 측면의 개념화다. 여기서 `존재'란 단순한 물리적 실재가 아니라, 의미 있는 기호 체계로서 사회적으로 기능하는 존재를 가리킨다.

\textbf{예시 1: 물리학}
\begin{itemize}
    \item $M_{\text{subject}}$: ``힘은 질량 곱하기 가속도다''라는 명제적 내용
    \item $M_{\text{object}}$: 프린키피아라는 물리적 텍스트가 존재한다는 사실
\end{itemize}

\textbf{예시 2: 종교}
\begin{itemize}
    \item $M_{\text{subject}}$: ``창조물을 사랑하는 인격신이 있다''라는 신학적 주장
    \item $M_{\text{object}}$: 성경이라는 물리적 텍스트가 존재한다는 사실
\end{itemize}

핵심적 차이: $M_{\text{subject}}$의 진리값은 논쟁 가능하다. 신이 존재하는지, $F=ma$가 궁극적으로 참인지는 논쟁의 대상이다. 그러나 $M_{\text{object}}$---이론의 존재---는 \textbf{부정 불가능}하다. 성경이 존재한다는 것, 프린키피아가 존재한다는 것은 누구도 부정할 수 없는 경험적 사실이다.

``객관적 주관성''이란 이 구조를 가리킨다: 이론의 내용은 주관적 주장이지만, 그런 주장을 담은 텍스트가 존재한다는 것은 객관적 사실이다.\footnote{본 논문의 ``객관적 주관성''은 나겔(Nagel)의 용법과 다르다. 나겔에게 ``objective subjectivity''는 주관적 경험을 객관적 관점에서 이해하는 것을 의미한다. 본 논문에서는 주관적 내용을 담은 텍스트의 객관적 실재를 지칭한다.}

칸트의 ``부정 불가능''이 논리적 필연성에 기반했다면, 자기도출의 ``부정 불가능''은 경험적 사실에 기반한다. 전자는 뒤집힐 수 있었다. 후자는 뒤집힐 수 없다.


\subsection{자기도출의 정의}

\begin{definition}[자기도출]
이론 $M$이 자기도출 구조를 갖는다는 것은, $M$의 내용($M_{\text{subject}}$)을 전제했을 때 $M$의 존재($M_{\text{object}}$)가 논리적으로 따라 나온다는 것이다:
\begin{equation}
M_{\text{subject}} \rightarrow M_{\text{object}}
\end{equation}
\end{definition}

자기도출의 구조:
\begin{enumerate}
    \item \textbf{전제}: 이론의 내용($M_{\text{subject}}$)을 가정한다.
    \item \textbf{도출}: 그 내용으로부터 이론의 존재($M_{\text{object}}$)가 따라 나온다.
    \item \textbf{착지}: 실제로 이론이 존재함을 확인한다.
\end{enumerate}


\subsection{뮌하우젠의 세 옵션과의 비교}

자기도출이 뮌하우젠의 세 옵션과 어떻게 다른가? 여기서 중요한 구분이 필요하다. 순환논증의 문제는 두 가지 차원에서 분석될 수 있다:

\begin{itemize}
    \item \textbf{논리적 공허함}: 전제가 결론을 순환적으로 지지하여, 실질적인 정보를 제공하지 않는다.
    \item \textbf{경험적 공허함}: 모든 항이 검증 불가능한 명제적 주장이어서, 외부 세계와의 접점이 없다.
\end{itemize}

\textbf{순환논증과의 차이}: 순환논증은 내용들 사이에서 돈다. $A \to B \to A$. 모든 항이 내용(명제)이고, 서로가 서로를 지지할 뿐 외부의 착지점이 없다. 이 구조는 논리적으로도, 경험적으로도 공허하다.

자기도출은 내용에서 존재로 간다. $M_{\text{subject}} \to M_{\text{object}}$. 존재는 내용과 다른 범주다. 자기도출은 \textbf{논리적 공허함은 해결하지 못한다}---내용의 참을 증명하지 않기 때문이다. 그러나 \textbf{경험적 공허함은 해결한다}---부정 불가능한 경험적 사실에서 착지하기 때문이다. 이것이 순환논증과의 결정적 차이다.

\textbf{무한퇴행과의 차이}: 무한퇴행은 끝없이 이어진다. $A \leftarrow B \leftarrow C \leftarrow \cdots$. 완결이 없다. 자기도출은 존재에서 멈춘다. 존재라는 경험적 사실에서 착지한다.

\textbf{독단적 중단과의 차이}: 독단적 중단은 ``$X$는 그냥 참이다''라고 선언한다. 근거가 없다. 자기도출은 근거가 있다---이론의 내용으로부터 이론의 존재가 도출된다. 물론 이것이 이론의 \emph{참}을 증명하는 것은 아니다. 그러나 근거 없는 선언과는 다르다.


\subsection{준-정당화로서의 자기도출}

자기도출은 정당화인가? 아니다. 자기도출은 이론이 \emph{참}임을 증명하지 않는다.

``인격신이 있다''는 내용에서 ``성경이 존재한다''가 도출되고, 실제로 성경이 존재한다. 그러나 이것이 ``인격신이 있다''가 참임을 증명하는가? 아니다. 다른 이유로 성경이 존재할 수 있다. 인간의 심리적 필요, 사회적 기능 등. 형식적으로 보면, 이것은 후건 긍정(affirming the consequent)이다: $H \to P$이고 $P$이면 $H$다---논리적으로 타당하지 않다.

그러나 이 구조는 과학철학에서 말하는 가설연역법(hypothetico-deductive method)과 \emph{형식적으로} 동형이다. 가설 $H$에서 예측 $P$가 도출되고, $P$가 관찰되었다. 이것이 $H$가 참임을 \emph{증명}하는가? 아니다. 다른 가설 $H'$도 $P$를 예측할 수 있기 때문이다. 형식적으로 가설연역법도 후건 긍정이다. 그러나 우리는 가설연역법을 ``오류''라고 부르지 않는다. ``지지''(confirmation)라고 부른다. 논리적으로 타당하지 않지만, 과학적 실천에서 핵심적 역할을 수행하기 때문이다.\footnote{가설연역법의 논리적 지위에 대한 논쟁은 광범위하다. 헴펠(Hempel)의 확증 이론, 베이즈주의, 가설추론(abduction) 등 다양한 접근이 있다. 본 논문은 이 논쟁에 개입하지 않는다. 다만 가설연역법이 ``증명''이 아닌 ``지지''의 논리로 기능한다는 점을 지적할 뿐이다.}

그러나 자기도출은 가설연역법과 \emph{인식론적으로} 동등하지 않다. 결정적 비대칭이 있다. 가설연역법에서 예측 $P$는 \textbf{구체적}이고 \textbf{반증 가능}하다. 일반상대성이론은 ``수성 근일점이 세기당 43초각 이동한다''를 예측했다. 이 예측이 틀렸다면---가령 50초각이 관측되었다면---이론은 반증되었을 것이다. 예측의 구체성이 이론에 제약을 부과한다.

반면 자기도출에서 $P$(``이 경전이 존재한다'')는 거의 모든 형태로 충족 가능하다. ``인격신이 있고 자신을 알리고자 한다''는 내용은 어떤 종류의 계시든---돌판, 두루마리, 인쇄본---예측과 양립한다. 예측이 포괄적이어서 실질적 제약력이 없다. 이것이 자기도출에서 반증 경로가 구조적으로 부재하는 이유다.

이 비대칭이 바로 과학과 종교의 인식론적 차이를 구조적으로 설명한다. 과학은 $P$의 구체성을 통해 강한 반증 가능성을 확보하지만, 자기도출의 존재론적 착지는 포기한다. 종교는 $P$의 포괄성을 통해 자기도출을 달성하지만, 반증 가능성을 희생한다. 이것이 trade-off다.

자기도출은 이 비대칭에도 불구하고 기능한다. 이론의 내용에서 이론의 존재가 도출되고, 이론이 실제로 존재한다. 이것은 이론을 \emph{증명}하지 않고 \emph{지지}한다. 본 논문은 이것을 \textbf{준-정당화}(quasi-justification)라 부른다.\footnote{본 논문이 ``pseudo-justification''(유사-정당화, 사이비-정당화)이 아닌 ``quasi-justification''(준-정당화)이라는 용어를 선택한 것은 의도적이다. ``Pseudo-''는 그리스어 \emph{pseudes}(거짓)에서 유래하며, 대상이 진짜인 척하지만 제대로 기능하지 않음을 함의한다(예: pseudoscience). 반면 ``quasi-''는 라틴어로 ``as if''(마치 ~인 것처럼)를 의미하며, 존재론적으로는 다르지만 기능적으로는 동등함을 가리킨다. 물리학의 quasi-particle이 대표적 사례다. Quasi-particle은 다체계(many-body system)에서 집단적 들뜸(collective excitation)이 마치 독립된 입자처럼 행동하는 현상이다. ``진짜 입자''는 아니다---전자나 양성자처럼 진공에서 독립적으로 존재할 수 없다. 그러나 운동량, 에너지, 스핀 등 입자의 물리적 속성을 가지고 입자\emph{처럼} 기능한다. 준-정당화도 마찬가지다: ``논리적 정당화''는 아니다---명제의 참을 증명하지 않는다. 그러나 확신을 산출한다는 점에서 정당화\emph{처럼} 인식적으로 기능한다. 이 구분은 본 논문의 기술적(descriptive) 입장과 일관된다---준-정당화는 ``거짓 정당화''가 아니라 ``정당화와 다른 방식으로 같은 기능을 수행하는 메커니즘''이다.}

여기서 본 논문의 주장을 명확히 해야 한다. 준-정당화는 새로운 \textbf{인식론적 범주}가 아니다. 믿음이 정당화된다고 주장하는 것이 아니기 때문이다. 준-정당화는 \textbf{인식적 현상의 기술}이다. 정당화가 불가능한 세계에서, 확신이 어떻게 산출되는가를 설명하는 구조다. 이것은 규범적 인식론이 아니라 기술적 분석이다.

자기도출은 정당화 게임 \emph{안에서} 경쟁하는 제4의 옵션이 아니다. 정당화 게임의 \emph{외부}에서 작동하는 별개의 메커니즘이다. 뮌하우젠 트릴레마는 ``참을 어떻게 증명할 것인가''라는 게임의 불가능성을 보여준다. 자기도출은 이 게임을 포기하고, ``확신이 어떻게 산출되는가''라는 다른 질문에 답한다.\footnote{확신 생성 메커니즘에 대한 논의는 기존 문헌에서 다양하게 이루어져 왔다. 인지종교학은 HADD(Barrett, 2004), CREDs(Henrich, 2009) 등 \emph{심리적-진화적} 기제를 분석하며, 개혁주의 인식론(Plantinga, 2000)은 신 믿음이 \emph{인식론적으로 정당화된다}고 주장한다. 종교경험 인식론(Alston, 1991; Swinburne, 2004)은 \emph{경험}의 자기-인증을 다루고, 준-신앙주의(Wittgenstein 해석)는 모든 합리성이 전제하는 \emph{범용적} hinge commitments를 논한다. 본 논문의 자기도출은 이들과 구별된다: (1) 심리적 기제가 아닌 \emph{논리적 구조}를 다루며, (2) 정당화를 주장하지 \emph{않고}, (3) 경험이 아닌 \emph{텍스트}에서 출발하며, (4) 범용적 전제가 아닌 \emph{종교 특수적} 구조를 분석한다.}

가추(abduction)와의 비교도 유익하다. 가추는 현상 $E$가 주어졌을 때, $E$를 가장 잘 설명하는 가설 $H$를 추론하는 방법이다. ``땅이 젖어 있다''($E$)를 관찰하고, ``비가 왔다''($H$)를 추론하는 것이 예다. 자기도출도 형식적으로 유사해 보인다: 경전의 존재($M_{\text{object}}$)를 관찰하고, 신의 존재($M_{\text{subject}}$)를 추론한다.

그러나 결정적 차이가 있다. 가추에서 현상($E$)과 가설($H$)은 \textbf{존재론적으로 독립적}이다. 젖은 땅은 비와 별개로 존재하며, 스프링클러로도 땅이 젖을 수 있다. 이 독립성 때문에 다른 설명이 경쟁할 수 있고, ``땅이 젖어 있다''는 ``비가 왔다''를 \emph{지지}하되 결정하지 않는다.

자기도출에서도 경쟁 설명이 존재한다---경전은 인간의 심리적 필요나 사회적 기능으로도 설명될 수 있다. 그러나 결정적 차이가 있다. 가추에서 현상과 가설은 분리된 존재자다. 자기도출에서 $M_{\text{object}}$와 $M_{\text{subject}}$는 \textbf{존재론적으로 동일한 대상}이다. 성경이라는 물리적 객체($M_{\text{object}}$)는 ``신이 자신을 알리고자 한다''라는 내용($M_{\text{subject}}$)을 \emph{담지하는} 바로 그 객체다. 가추에서 가설은 현상의 \emph{해석}이다. 자기도출에서 가설은 현상의 \emph{구성요소}다.


\subsection{반증주의적 관점에서의 자기도출}

포퍼의 반증주의에 따르면, 가설 $H$가 예측 $O$를 산출할 때 $\neg O$가 관찰되면 $H$는 반증된다(후건 부정). 가설연역법에서 예측이 확인되면 가설은 \textbf{지지}(confirmation)된다---연역적으로 정당화되지는 않지만, 인식적 신뢰를 얻는다.

자기도출은 형식적으로 가설연역법과 동일한 구조를 공유한다: $M_{\text{subject}} \to M_{\text{object}}$이고 $M_{\text{object}}$이므로 $M_{\text{subject}}$가 지지된다. 그러나 차이는 형식이 아니라 \textbf{항의 존재론적 성격}에 있다.

가설연역법에서 예측 대상($O$)은 이론($H$)과 존재론적으로 독립적이다. 뉴턴 역학이 참이든 거짓이든, 행성의 궤도는 그 자체로 존재한다. 이론과 증거가 분리되어 있기 때문에, 증거가 이론을 반증하는 것이 가능하다.

그러나 자기도출에서 $M_{\text{object}}$---텍스트의 존재---는 $M_{\text{subject}}$---텍스트의 내용---의 물리적 실현이다. 성경은 ``신의 계시''라는 내용을 \emph{담고 있는} 바로 그 물리적 객체다. 이론과 증거가 존재론적으로 분리되지 않는다.

이 특성을 \textbf{존재의 객관적 주관성}이라 부를 수 있다---$M_{\text{object}}$가 객관적 사실이면서 동시에 $M_{\text{subject}}$를 담지하는 이중적 성격을 가리킨다. $M_{\text{object}}$는 객관적이다---물리적으로 존재하며, 부정 불가능하다. 그러나 동시에 $M_{\text{subject}}$를 담지한다는 점에서 주관적이다. 예측 대상이 이론 자체의 존재이기 때문에, 이론의 내용이 참이든 거짓이든 예측($M_{\text{object}}$)은 항상 충족된다. 이것이 반증 경로가 구조적으로 부재한 이유다.

이것이 자기도출의 ``결함''인지는 별개의 문제다. 어떤 설명이 대체되려면 더 나은 대안이 있어야 한다. 행성 궤도의 경우, ``신의 설계''는 뉴턴 역학으로 대체되었다---후자가 더 정밀한 예측을 제공했기 때문이다.

그러나 경전의 존재---왜 이런 내용을 담은 텍스트가 도출되었는가---에 대해서는 사정이 다르다. 과학은 종교 현상의 일반적 조건(사회적 필요, 심리적 기능)과 텍스트의 물리적 생산 및 전파는 설명할 수 있다. 그러나 왜 \emph{하필 이 특정 내용}이 도출되었는가, 그리고 그 내용이 왜 텍스트의 존재와 정합적인 구조를 갖는가는 충분히 설명하지 못한다.\footnote{``사회적 구성물''이라는 설명은 경전이 존재한다는 사실은 설명하지만, 왜 하필 ``인격신의 계시''라는 내용인지, 그리고 왜 이 내용이 텍스트의 존재를 함축하는 구조를 갖는지는 해명하지 못한다. 인지종교학의 연구들---예컨대 Boyer, Atran---이 종교적 관념의 인지적 기반을 탐구하고 있으나, 특정 텍스트의 특정 내용까지 설명하는 데는 이르지 못했다.}

자기도출은 이 공백을 채운다. ``인격신이 인간에게 자신을 알리고자 했다''는 내용($M_{\text{subject}}$)은 ``그 계시를 담은 텍스트가 존재할 것''이라는 예측을 산출하고, 그 예측은 경험적으로 충족된다($M_{\text{object}}$). 내용과 존재 사이에 내적 정합성이 성립한다.

자기도출이 반증 불가능함에도 확신을 산출하는 이유는, 그것이 논리적으로 우월해서가 아니다. 현재로서 \textbf{내용과 존재의 정합성}을 설명하는 대안적 체계가 부재하기 때문이다. 대안이 등장하지 않는 한, 자기도출은 그 설명적 공백을 점유하며 확신의 기반으로 기능한다.


\subsection{유사 개념과의 구분}

자기도출은 여러 유사 개념들과 혼동될 수 있다. 체계적 구분이 필요하다.

\subsubsection{자기복제: 존재에서 존재로}

DNA는 자기복제가 가능하다. 그러나 DNA는 자기도출을 하지 않는다.

자기도출은 이론의 \emph{내용}을 전제했을 때 이론의 \emph{존재}가 예측되는 구조다. DNA 복제는 \emph{이미 존재하는} DNA가 복사본을 만드는 인과적 과정이다. DNA의 내용(염기서열)만 전제하고 DNA의 존재를 전제하지 않은 상태에서, DNA의 존재가 논리적으로 따라 나오지는 않는다.

\begin{center}
\begin{tabular}{ll}
자기복제: & 존재(A) $\to$ 존재(A') [인과적] \\
자기도출: & 내용($M_{\text{sub}}$) $\to$ 존재($M_{\text{obj}}$) [논리적]
\end{tabular}
\end{center}


\subsubsection{자기참조: 내용에서 내용으로}

괴델 수는 수학적 진술을 자연수로 인코딩하여, 수학이 수학에 대해 말할 수 있게 한다. 불완전성 정리의 자기참조 문장(``이 진술은 증명 불가능하다'')은 자기참조의 전형이다.

자기참조의 구조는 내용(A)이 내용(A) 자신을 직접 가리키는 것이다. 괴델 문장 $G$는 ``$G$는 증명 불가능하다''라고 말한다---$G$의 내용이 $G$ 자신에 대한 진술이다.

그러나 이것은 자기도출이 아니다. 괴델 문장은 자기 \emph{내용}을 가리킬 뿐, 자기 \emph{존재}를 도출하지 않는다. ``이 진술은 증명 불가능하다''는 내용을 전제해도, 그 문장이 왜 존재하는지는 따라 나오지 않는다.

\begin{center}
\begin{tabular}{ll}
자기참조: & 내용(A) $\to$ 내용(A) [동일 범주] \\
자기도출: & 내용($M_{\text{sub}}$) $\to$ 존재($M_{\text{obj}}$) [범주 전환]
\end{tabular}
\end{center}


\subsubsection{콰인 프로그램: 내용에서 내용으로 (인과적)}

콰인(quine)은 자기 자신의 소스코드를 출력하는 프로그램이다. 입력 없이 실행하면, 출력으로 자기 자신의 코드가 나온다.

콰인은 자기참조와 유사하지만, 결정적 차이가 있다:
\begin{itemize}
    \item 자기참조(괴델 수): 내용이 내용을 \textbf{가리킨다}. 논리적 관계.
    \item 콰인: 내용이 내용을 \textbf{생성한다}. 실행이라는 인과적 과정을 거친다.
\end{itemize}

콰인은 자기참조의 인과적 버전이라 할 수 있다. 그러나 콰인도 자기도출은 아니다. 콰인이 생성하는 것은 자기 \emph{내용}(소스코드)이지, 자기 \emph{존재}(그 프로그램이 왜 있는가)가 아니다. 콰인의 내용을 전제해도, 콰인이라는 프로그램이 왜 존재하는지는 따라 나오지 않는다.

\begin{center}
\begin{tabular}{ll}
콰인: & 내용(A) $\to$ 내용(A') [인과적] \\
자기도출: & 내용($M_{\text{sub}}$) $\to$ 존재($M_{\text{obj}}$) [범주 전환]
\end{tabular}
\end{center}


\subsubsection{순환논증: 내용에서 내용으로, 경유}

순환논증의 구조는 내용(A)이 내용(B)를 경유하여 다시 내용(A)로 돌아오는 것이다.

\begin{quote}
``신은 존재한다. 왜냐하면 성경에 그렇게 기록되어 있으니까. 그리고 성경의 기록은 모두 진리이다. 그것은 신의 계시이므로.''
\end{quote}

여기서 ``성경에 그렇게 기록되어 있다''는 그 기록이 \textbf{참이라는 명제 주장}이다. A(신의 존재) $\to$ B(성경의 진리성) $\to$ A(신의 존재)로 순환하며, 모든 항이 검증 불가능한 \emph{내용}(명제)이고, 서로가 서로를 지지할 뿐 외부의 착지점이 없다.

자기도출에서 ``성경이 존재한다''는 그 기록을 담은 \textbf{텍스트가 실재한다는 사실 주장}이다. 성경의 내용이 참인지는 논쟁 가능하지만, 성경이라는 텍스트가 존재한다는 것은 부정 불가능한 경험적 사실이다. 이것이 착지점이다.

\begin{center}
\begin{tabular}{ll}
순환논증: & 내용(A) $\to$ 내용(B) $\to$ 내용(A) [착지점 없음] \\
자기도출: & 내용($M_{\text{sub}}$) $\to$ 존재($M_{\text{obj}}$) [착지]
\end{tabular}
\end{center}


\subsubsection{자기예언: 내용에서 존재로, 다시 내용으로}

자기도출과 가장 혼동되기 쉬운 것이 \textbf{자기예언}(self-fulfilling prophecy)이다.\footnote{이 개념은 로버트 K. 머튼(Robert K. Merton)이 1948년 정립했다. 머튼은 W.I. 토마스의 정리---``사람들이 상황을 실재라고 정의하면, 그 결과에서 실재가 된다''---를 발전시켜, 처음에는 거짓이었던 정의가 그것을 믿는 행동을 유발하여 결국 참이 되는 현상을 분석했다.}

\textbf{뱅크런}: ``이 은행은 망할 것이다''라고 예언한다. 사람들이 이를 믿고 예금을 인출한다. 실제로 은행이 망한다. ``예언이 맞았다.'' 이것은 머튼이 제시한 전형적 사례다.

자기예언의 구조는 \textbf{내용 $\to$ 존재 $\to$ 내용}이다. 내용(예언)이 존재(사태)를 인과적으로 야기하고, 그 존재가 다시 내용(``맞았다'')을 강화한다.

자기도출과 결정적으로 다른 점 두 가지:

\textbf{첫째, 경로가 다르다.} 자기예언은 존재를 경유하여 내용으로 \textbf{돌아온다}. 자기도출은 존재에서 \textbf{착지하고 끝난다}. 경전이 존재한다고 해서 ``인격신이 있다''가 참이 되는 것은 아니다. 가설이 지지될 뿐이다.

\textbf{둘째, 대상이 다르다.} 자기예언에서 내용 = ``은행이 망할 것''(A), 존재 = 은행 파산(B). $A \neq B$. 자기도출에서 내용 = $M$이 말하는 것, 존재 = $M$이 있다는 것. \textbf{둘 다 같은 $M$의 두 측면}이다.

\begin{center}
\begin{tabular}{ll}
자기예언: & 내용(A) $\to$ 존재(B) $\to$ 내용(A) [순환, 대상 불일치] \\
자기도출: & 내용($M_{\text{sub}}$) $\to$ 존재($M_{\text{obj}}$) [착지, 동일 대상]
\end{tabular}
\end{center}


\subsubsection{인류원리: 존재에서 내용으로}

인류원리(anthropic principle)는 자기도출과 비교할 가치가 있다. 둘 다 ``왜 이것인가?''라는 질문에 대해 순환 없이 답하려는 시도이기 때문이다. 인류원리는 ``왜 이 물리 상수인가?''에 대해, ``관측자가 존재할 수 있는 우주만 관측된다''고 답한다. 자기도출은 ``왜 이 경전이 존재하는가?''에 대해, ``경전의 내용이 경전의 존재를 도출한다''고 답한다.

그러나 두 전략은 구조적으로 다르다.

\textbf{첫째, 방향이 반대다.} 인류원리는 존재(관측자)에서 출발하여 내용(상수의 조건)을 역으로 제한한다. 이는 \emph{관찰 선택 효과}(observational selection effect)라 불리며, 설명력이 부족한 동어반복(tautology)이라는 비판을 받는다.\footnote{Bostrom, \emph{Anthropic Bias} (2002); Leslie, \emph{Universes} (1989) 참조.} 반면 자기도출은 내용($M_{\text{subject}}$)에서 존재($M_{\text{object}}$)로 간다.

\textbf{둘째, 대상 동일성이 없다.} 인류원리에서 관측자와 물리 상수는 서로 다른 대상이다. 반면 자기도출에서는 ``이론의 내용''과 ``이론의 존재''가 \emph{같은 대상} $M$의 두 측면이다. 이 자기포함성이 자기도출의 핵심이다.


\subsubsection{일곱 개념의 비교}

\begin{table}[ht]
\centering
\begin{tabular}{lccl}
\toprule
개념 & 경로 & 성격 & 예시 \\
\midrule
자기복제 & 존재(A) $\to$ 존재(A') & 인과적 & DNA \\
자기참조 & 내용(A) $\to$ 내용(A) & 논리적 & 괴델 문장 \\
콰인 & 내용(A) $\to$ 내용(A') & 인과적 & 콰인 프로그램 \\
순환논증 & 내용(A) $\to$ 내용(B) $\to$ 내용(A) & 논리적 & ``신 $\because$ 성경 $\because$ 신'' \\
자기예언 & 내용(A) $\to$ 존재(B) $\to$ 내용(A) & 인과적 & 뱅크런 \\
인류원리 & 존재(A) $\to$ 내용(B) & 선택적 & 미세조정 \\
\textbf{자기도출} & 내용($M_{\text{sub}}$) $\to$ 존재($M_{\text{obj}}$) & 준-정당화 & 종교 경전 \\
\bottomrule
\end{tabular}
\caption{재귀적 구조들의 분류}
\end{table}

자기도출의 고유한 특징:
\begin{enumerate}
    \item \emph{내용}에서 \emph{존재}로 가는 경로가 있다. (자기복제, 자기참조, 콰인, 순환논증과 구분)
    \item 그 경로는 인과적 순환이 아니라 착지다. (자기예언과 구분)
    \item 내용과 존재가 \emph{같은 대상} $M$의 두 측면이다. (자기예언과 구분)
\end{enumerate}


%============================================================
\section{종교: 자기도출의 사례}
%============================================================

\subsection{자기도출의 기본 구조}

인격신을 전제하면 계시가 따라 나온다. 왜 단순한 ``제1원인''이 아니라 ``창조물을 사랑하는 인격신''인가?

\begin{itemize}
    \item 제1원인: 세계를 도출할 수 있다. 그러나 계시를 도출할 이유가 없다.
    \item 인격신: 창조물을 사랑한다 $\to$ 창조물에게 자신을 알리고 싶다 $\to$ 계시.
\end{itemize}

``사랑''과 ``계시 욕구''라는 인격적 속성이 $M_{\text{subject}}$(이론의 내용)에서 $M_{\text{object}}$(이론의 존재)로 가는 경로를 제공한다. 아리스토텔레스의 부동의 원동자나 이신론의 신은 자기도출 경로를 갖지 않는다. 세계를 시작하게 했지만, 자기를 알릴 이유가 없다. 인격신만이 ``내용에서 존재로'' 가는 경로를 내장하고 있다.

그러나 계시가 반드시 텍스트일 필요는 없다. 계시는 구전, 성사(聖事), 직접 체험, 육화(肉化) 등 다양한 형태를 취할 수 있다. 자기도출의 강도는 \textbf{계시가 텍스트와 얼마나 동일시되는가}에 따라 달라진다.


\subsection{텍스트 절대성의 스펙트럼}

종교 전통들은 텍스트의 지위에서 스펙트럼을 이룬다.

\begin{table}[ht]
\centering
\begin{tabular}{llc}
\toprule
전통 & 텍스트의 지위 & 자기도출 강도 \\
\midrule
이슬람 & 쿠란 = 창조되지 않은 신의 말씀 자체 & 최강 \\
루터 개신교 & Sola Scriptura, 성경만이 유일한 권위 & 강함 \\
가톨릭 & 성경 + 전통 + 교도권 (삼중 권위) & 중간 \\
자유주의 신학 & 성경 = 인간의 신앙 경험 기록 & 약함 \\
\bottomrule
\end{tabular}
\caption{텍스트 절대성 스펙트럼}
\end{table}

\textbf{이슬람}: 쿠란은 ``신의 말씀 자체''(칼람 알라, كلام الله)다. 창조되지 않았고, 영원히 신과 함께 존재했다. 아랍어 원문 자체가 신성하며, 번역은 ``해석''일 뿐 쿠란이 아니다. $M_{\text{subject}}$(신의 말씀)와 $M_{\text{object}}$(쿠란 텍스트)의 동일성이 가장 직접적이다.

\textbf{루터 개신교}: 종교개혁의 핵심 원리 중 하나가 ``오직 성경''(Sola Scriptura)이다. 교황, 공의회, 전통의 권위를 거부하고, 성경만이 신앙과 실천의 유일한 규범이라고 선언했다. 텍스트의 권위를 극대화함으로써 자기도출 구조를 강화했다.

\textbf{가톨릭}: 성경, 전통(사도로부터 이어지는 구전 교리), 교도권(교황과 공의회의 가르침)이라는 삼중 권위를 인정한다. 텍스트가 유일한 착지점이 아니므로, 자기도출 구조가 분산된다.

\textbf{자유주의 신학}: 성경을 인간이 기록한 신앙 경험의 기록으로 본다. 역사적 비평의 대상이며, 문자적 영감을 부정한다. $M_{\text{subject}}$와 $M_{\text{object}}$의 연결이 가장 약하다.

요한복음 1:1-3(``태초에 말씀이 계시니라...'')은 이 스펙트럼 어디에서나 해석 가능하다. 신학적으로 ``말씀''(로고스)은 그리스도를 가리키며, 성경은 그 말씀의 ``증언''이지 말씀 자체가 아니다. 그러나 텍스트 절대성이 높은 전통에서는 이 구별이 실천적으로 희미해진다---성경이 곧 신의 말씀처럼 기능한다.


\subsection{가설: 텍스트 절대성과 확신의 강도}

이 분석에 기초하여, 본 논문은 다음 \emph{가설}을 제안한다:

\begin{quote}
\textbf{텍스트 절대성이 높을수록, 종교적 확신이 강해진다.}
\end{quote}

왜 그런가? 텍스트 절대성이 높을수록:
\begin{enumerate}
    \item $M_{\text{subject}}$와 $M_{\text{object}}$의 동일성이 강해진다.
    \item 착지점(텍스트의 존재)에서 내용(신학적 주장)으로의 전이가 더 직접적이다.
    \item 확실성의 전이가 더 강력하게 작동한다.
\end{enumerate}

이 가설은 현재 사변적이며, 엄밀한 경험적 검증을 요한다. 가능한 검증 방법으로는 텍스트 절대성 수준이 다른 종교 집단 간 확신 강도 비교 연구가 있다. 본 논문은 이 가설의 \emph{검증}이 아닌 \emph{정식화}를 목표로 한다.

그러나 예비적 관찰은 가설과 양립하는 것처럼 보인다. 이슬람 근본주의와 개신교 근본주의는 텍스트의 문자적 권위를 강조한다. 반면 자유주의 신학은 ``확실히 신이 있다''보다 ``신을 향한 열림''을 말한다. 텍스트 절대성이 낮아지면 확신의 강도도 낮아지는 경향이 있다.\footnote{물론 이것은 단순화다. 종교적 확신에는 공동체, 의례, 개인적 경험 등 다른 요인도 작용한다. 본 논문이 주장하는 것은 텍스트 절대성이 \emph{하나의} 구조적 요인이라는 것이지, 유일한 요인이라는 것이 아니다. 이 다변인 맥락에서 텍스트 절대성의 독립적 기여를 분리하는 것이 경험적 검증의 과제가 될 것이다.}


\subsection{극단적 사례: 카리스마적 종파의 변형된 자기도출}

앞 절에서 텍스트 절대성이 확신의 강도와 상관한다고 보았다. 그러나 자기도출의 착지점($M_{\text{object}}$)이 반드시 텍스트여야 하는 것은 아니다. 자기도출 스펙트럼의 극단에서는 착지점 자체가 텍스트에서 \textbf{인간}으로 전이되는 변형된 형태가 나타난다.

막스 베버가 분석한 \textbf{카리스마적 권위}에 기반한 신흥 종교 운동이 그 예다.\footnote{베버는 카리스마적 권위를 ``특정 개인의 비범한 자질에 의해 성립하는 권위''로 정의했다. 이러한 권위는 전통적 권위나 합법적-합리적 권위와 달리 지도자 개인의 인격에 의존하며, 본질적으로 불안정하다.} 이들은 자기도출의 착지점($M_{\text{object}}$)을 불변하는 텍스트에서 가변적인 \textbf{교주의 육체적 존재}로 전이시킨다.

\begin{center}
\begin{tabular}{lll}
\toprule
& 제도화된 종교 & 카리스마적 종파 \\
\midrule
$M_{\text{subject}}$ (내용) & ``신이 계시하신다'' & ``예언된 자가 도래한다'' \\
$M_{\text{object}}$ (존재) & 경전이 존재한다 (Text) & \textbf{교주가 존재한다 (Body)} \\
\bottomrule
\end{tabular}
\end{center}

교주는 자신의 우연적 속성들---출생지, 이름, 수감 이력 등---을 경전의 상징과 자의적으로 결합하여, ``내가 곧 그 예언의 성취''라는 과적합(overfitting)된 가설연역 체계를 구축한다. 이 과정에서 발생하는 되먹임(feedback)으로 인해, 교주 본인이 가장 먼저 확실성의 전이를 경험한다. ``내가 여기 실존하고($M_{\text{object}}$), 내 삶이 저기에 기록되어 있다($M_{\text{subject}}$).'' 이로써 교주는 확신의 최초 수용자이자 발신자가 된다.

경전은 침묵하지만, 교주는 실시간으로 확신을 발산한다. 신도들을 매혹하는 것은 교주의 논리가 아니라, 자기도출 회로가 뿜어내는 \emph{존재론적 확신}이다. 교주가 노쇠하거나 병드는 객관적 모순 앞에서도 믿음이 붕괴하지 않는 이유는, 믿음의 토대가 교주의 `능력'이 아니라 교주의 `존재 그 자체'($M_{\text{object}}$)에 착지해 있기 때문이다. 존재는 논리적으로 반박될 수 없다.

이는 자기도출의 \textbf{변형된 형태}다. 불변하는 텍스트의 권위를 가변적인 인간의 육체로 \textbf{전유(appropriation)}하여, 확실성 전이 회로를 인간에게 직접 이식한 구조다.

역사적 사례들이 이 구조를 보여준다. 짐 존스(Jim Jones)의 인민사원(Peoples Temple)에서, 존스는 자신을 성경적 예언의 성취로 제시하며 점차 성경보다 자신의 ``살아있는 말씀''에 더 큰 권위를 부여했다.\footnote{Reiterman, T. \& Jacobs, J. (1982). \emph{Raven: The Untold Story of the Rev. Jim Jones and His People}. E.P. Dutton.} 데이비드 코레시(David Koresh)의 다윗파(Branch Davidians)는 더 명시적이다---코레시는 자신이 요한계시록의 ``일곱 인(Seven Seals)''을 해제할 유일한 권위자라 주장했고, 성경 해석의 권위가 텍스트에서 해석자 자신으로 완전히 이전되었다.\footnote{Tabor, J. \& Gallagher, E. (1995). \emph{Why Waco? Cults and the Battle for Religious Freedom in America}. University of California Press.} 두 사례 모두에서 $M_{\text{object}}$는 더 이상 ``경전의 존재''가 아니라 ``교주의 존재''가 되었다.

베버가 분석한 ``카리스마의 일상화(routinization of charisma)''---지도자 사후 권위가 제도나 전통으로 이전되는 과정---는 이 불안정한 구조가 제도화된 형태로 이행하는 한 경로다.


\subsection{확실성의 전이: 종교적 확신의 메커니즘}

자기도출 구조가 설명하는 것은 종교가 왜 \emph{그토록} 강력한 확신을 제공하는가이다. 단순히 ``좋은 삶의 지침''이라서가 아니다. 다른 체계와 비교할 수 없을 만큼 강도 높은 믿음---이것은 어디서 오는가?

핵심은 \textbf{확실성의 전이}다.

\begin{enumerate}
    \item \textbf{착지점은 부정 불가능하다}: ``경전이 존재한다''는 경험적 사실이다. 쿠란이 존재한다는 것, 성경이 존재한다는 것은 누구도 부정할 수 없다.
    \item \textbf{자기도출 구조 안에서, 이 착지점은 내용과 연결된다}: ``인격신이 있다''를 전제하면 ``경전이 존재한다''가 따라 나온다.
    \item \textbf{착지점의 확실성이 내용의 확신으로 전이된다}: ``경전이 존재하니까, 인격신이 있는 것이다.''
\end{enumerate}

이 전이는 논리적으로 타당하지 않다. 후건 긍정의 오류다. 그러나 \emph{심리적으로} 강력하게 작동한다. 존재의 부정 불가능성이 내용의 부정 불가능성으로 \emph{느껴진다}.

그리고 텍스트 절대성이 이 전이를 증폭한다. 텍스트가 ``신의 말씀 자체''라면, 텍스트의 존재는 곧 신의 말씀의 존재다. 착지점과 내용 사이에 간극이 없다. 반면 텍스트가 ``신앙 경험의 기록''이라면, 텍스트의 존재는 신의 존재를 직접 함축하지 않는다. 전이가 약해진다.

자기도출은 아브라함 계열 유신론에서 확신을 산출하는 \textbf{하나의} 구조적 메커니즘이다. 그러나 이것이 종교적 확신의 \emph{유일한} 원천은 아니다. 종교학과 종교심리학은 다른 메커니즘들을 제시한다:

\begin{itemize}
    \item \textbf{종교적 경험}: 신비 체험, 개심 경험, 기도 중의 현존 경험 등 직접적 경험이 확신의 원천이 될 수 있다.
    \item \textbf{의례적 실천}: 반복적 의례를 통해 확신이 신체에 각인될 수 있다.
    \item \textbf{공동체적 정체성}: 종교 공동체에의 소속감과 집단 정체성이 확신을 강화할 수 있다.
\end{itemize}

본 논문이 주장하는 것은 자기도출이 이러한 메커니즘들 중 \emph{하나}라는 것이다. 특히 텍스트 중심적 아브라함 계열 유신론에서, 자기도출은 확신의 \textbf{구조적 토대}를 제공한다. 다른 메커니즘들이 이 토대 위에서 작동하거나, 또는 독립적으로 작동할 수 있다. 이들 사이의 관계에 대한 분석은 별도의 연구를 요한다.

자기도출이 설명하는 것은 \textbf{확신의 강도}다. 왜 신자는 ``아마도 신이 있을 것이다''가 아니라 ``확실히 신이 있다''고 믿는가? 착지점의 확실성이 전이되기 때문이다. 그리고 텍스트 절대성이 높을수록 이 전이가 강력해진다.

이것은 종교를 옹호하는 것도, 비판하는 것도 아니다. 메커니즘을 기술하는 것이다.


%============================================================
\section{과학: 자기도출의 부재}
%============================================================

\subsection{과학은 왜 자기도출이 없는가}

물리학의 자기도출을 시도해보자.

\begin{itemize}
    \item $M_{\text{subject}}$: ``$F=ma$''
    \item $M_{\text{object}}$: ``프린키피아가 존재한다''
\end{itemize}

$F=ma$로부터 프린키피아의 존재를 도출하는 경로가 있는가?

$F=ma$가 말하는 것은 ``힘은 질량 곱하기 가속도다.'' 이것은 입자의 운동을 함축한다. 그러나 이것은 물리학자의 존재를 함축하지 않고, 물리 논문의 존재를 함축하지 않으며, 뉴턴의 뇌가 $F=ma$를 발견하는 것을 함축하지 않는다.

\textbf{수행적 법칙과 코드적 법칙의 구분}이 필요하다.

\begin{center}
\begin{tabular}{lll}
\toprule
& 수행적 $F=ma$ & 코드적 $F=ma$ \\
\midrule
의미 & 입자들이 $F=ma$에 따라 움직임 & ``$F=ma$''라는 기호열, 텍스트 \\
위치 & 물리적 세계 안에서 작동 & 프린키피아, 교과서, 논문 \\
\bottomrule
\end{tabular}
\end{center}

자기도출에서 $M$은 \textbf{코드적} $M$이다. 종교에서 ``말씀''(Logos)은 처음부터 코드다. 그러나 물리학에서 코드적 $F=ma$(프린키피아)는 뉴턴이 써야 존재한다. 수행적 법칙이 태초에 있었다고 해서 코드적 법칙의 존재가 도출되지는 않는다.

\textbf{라플라스의 악마}: 이론적으로, 빅뱅 초기조건에서 출발하여 프린키피아에 도달하는 경로가 있을 수 있다:
\begin{equation}
x_{\text{init}} \to \text{입자들} \to \text{별} \to \text{지구} \to \text{생명} \to \text{인간} \to \text{뉴턴} \to \text{프린키피아}
\end{equation}

그러나 두 가지 문제가 있다. 첫째, 비용이 천문학적이다---빅뱅에서 프린키피아까지는 138억 년의 우주 역사 전체다. 둘째, 초기조건을 특정할 수 없다---어떤 $x_{\text{init}}$에서 출발해야 프린키피아가 나오는지 알 수 없다.

\textbf{결론}: 과학은 내용($F=ma$)에서 존재(프린키피아)로 가는 경로가 없거나, 있더라도 천문학적으로 길다. 자기도출이 부재하거나 유예된다.


\subsection{결함인가 전략인가}

과학의 자기도출 부재는 결함인가?

아니다. 이것은 \textbf{전략적 선택}이다.

과학은 ``어떻게''에 답하지, ``왜''에 답하지 않는다. ``왜 $F=ma$인가''에 답하지 않는 것은 과학의 한계가 아니라 과학의 방법론이다. 이 방법론 덕분에 과학은 \textbf{외부 설명력}을 극대화할 수 있었다.

여기서 외부 설명력 개념을 명확히 해야 한다.

\begin{definition}[외부 설명력]
이론 $M$의 외부 설명력이란, $M$이 자기 존재($M_{\text{object}}$) 이외의 현상에 대해 새로운 예측을 산출하고, 그 예측이 경험적으로 확인되는 정도를 말한다.
\end{definition}

뉴턴 역학은 외부 설명력이 높다. $F=ma$라는 단일 법칙에서 행성 궤도, 사과 낙하, 조석 현상, 인공위성 궤도 등 광범위한 현상이 도출되고, 이 예측들은 경험적으로 확인된다. 반면 ``신이 세계를 창조했다''는 주장은 외부 설명력이 낮다. 이 주장에서 행성 궤도가 타원인지 원인지, 중력이 역제곱 법칙을 따르는지 역세제곱 법칙을 따르는지가 도출되지 않는다. 현상의 구체적 구조를 설명하려면 사후적 해석이 필요하다.\footnote{본 논문은 외부 설명력을 정량화하지 않는다. 이것은 별도의 방법론적 연구를 요한다. 여기서는 개념적 구분만을 제시한다.}

자기도출을 갖추려면 비용이 든다. 종교는 그 비용을 치렀다---인격신, 계시, 경전이라는 복잡한 구조를 도입해야 했고, 그 대가로 외부 설명력이 낮아졌다. 과학은 그 비용을 치르지 않기로 선택했다.


\subsection{딜레마의 발견}

종교와 과학의 구조를 비교하면 딜레마가 드러난다.

\textbf{종교}:
\begin{itemize}
    \item 자기도출: 있음 (인격신 $\to$ 계시 $\to$ 경전)
    \item 외부 설명력: 낮음 (현대 물리학의 상세를 설명하려면 방대한 해석적 비용)
\end{itemize}

\textbf{과학}:
\begin{itemize}
    \item 외부 설명력: 높음 (간결한 법칙으로 광범위한 현상을 정밀하게 예측)
    \item 자기도출: 없음 또는 유예 (``왜 이 법칙인가''에 답하지 않음)
\end{itemize}

\begin{center}
\begin{tabular}{lcc}
\toprule
& 외부 설명력 & 자기도출 \\
\midrule
과학 & 높음 & 없음/유예 \\
종교 & 낮음 & 있음 \\
\bottomrule
\end{tabular}
\end{center}

\begin{theorem}[자기도출의 딜레마]
토대적 지식 시스템에서, 외부 설명력과 자기도출을 동시에 달성하는 것은 일반적으로 어렵다.
\end{theorem}

\textbf{과학}은 외부 설명력을 최적화한다. 간결한 법칙으로 광범위한 현상을 정밀하게 예측한다. 그러나 자기도출은 유예한다. ``왜 이 법칙인가''에 답하지 않는다.

\textbf{종교}는 자기도출을 최적화한다. 초기조건에 코드를 내장하여 자기도출 경로를 확보한다. 그러나 외부 설명력이 낮다. 현대 과학이 밝힌 물리 세계의 상세를 설명하려면 방대한 해석적 비용이 든다.

이것이 과학과 종교가 공존하는 이유다. 각각이 다른 것을 최적화하므로, 하나가 다른 하나를 완전히 대체할 수 없다.


%============================================================
\section{딜레마 해결 시도: $M_{\text{unified}}$}
%============================================================

\subsection{종교적 논증의 형식 추출}

딜레마를 해결할 수 있는가? 과학의 언어로 자기도출을 달성하는 것이 가능한가?

먼저 종교적 논증의 \textbf{형식}을 추출하자.

\begin{definition}[자기도출 논증의 형식]
모델 $M$이 초기조건 $x$에 적용되었을 때, 출력이 $M$ 자신과 외부 데이터 $D$를 포함한다:
\begin{equation}
M(x) = \{M, D\}
\end{equation}
\end{definition}

종교에서 이 형식이 작동하는 이유는:
\begin{itemize}
    \item 초기조건에 코드가 포함됨: ``태초에 말씀이 계시니라''
    \item 인격신의 속성: 코드가 자기를 계시할 이유가 있음
\end{itemize}

이 형식을 추출하면, \textbf{내용은 바꿀 수 있다}: 인격신 대신 다른 초기조건. \textbf{구조는 보존된다}: $M$이 초기조건에 적용되고, 출력에 $M$과 $D$가 포함되는 구조.


\subsection{$M_{\text{unified}}$의 전략}

$M_{\text{unified}}$는 종교적 논증의 형식을 과학의 언어로 재구성한다. 그러나 종교와 정반대 방향에서 출발한다.

\textbf{종교의 전략 (top-down)}: 초기조건에 완성된 코드의 선재를 선언한다. ``태초에 말씀이 계시니라.'' 코드가 자기를 계시하는 하향식 구조다.

\textbf{$M_{\text{unified}}$의 전략 (bottom-up)}: 초기조건을 최소화한다.
\begin{itemize}
    \item 초기조건: 공집합 $\emptyset$과 멱집합 함수 $\mathcal{P}$
    \item 첫 명령어: ``멱집합을 공집합에 무한 적용하라''
\end{itemize}

이 명령어가 실행되면서 계층이 전개되고, 그 끝에서 $M$이 창발한다. 시작에서 끝으로 전개하는 상향식 구조다.


\subsection{$M_{\text{unified}}$의 조건: 두 가지 고정점}

물리학 단독으로는 자기도출이 불가능하다. 그러나 과학 전체를 고려하면 어떨까?

역사적으로 과학은 계층적으로 분화해왔다. 물리학이 기술하는 현상 위에서 화학이 작동하고, 화학이 기술하는 현상 위에서 생물학이 작동한다:
\begin{equation}
\text{물리학} \to \text{화학} \to \text{생물학} \to \text{신경과학} \to \text{인지과학} \to \text{사회과학} \to \text{과학학}
\end{equation}

각 단계에서 하위 레벨의 현상이 상위 레벨의 대상이 된다. 이 계층의 끝에서 과학 자체가 과학학의 대상이 된다.

자기도출이 가능하려면 이 계층이 \textbf{고정점}에 도달해야 한다. 여기서 두 종류의 고정점을 구분해야 한다.

\textbf{형식적 고정점}: 제도적 지식이 제도적 지식을 대상으로 삼고, 그 결과도 제도적 지식이다. 인식론, 과학학, 과학철학이 이미 이 구조를 갖는다---지식에 대한 지식.

그러나 형식적 고정점만으로는 자기도출이 달성되지 않는다. 인식론이 ``지식에 대한 지식''이라고 해도, 물리학에서 인식론까지 오는 \textbf{경로}가 명시되지 않으면, 내용에서 존재로 가는 도출이 완결되지 않는다.

\textbf{내용적 고정점}: 모든 제도적 지식이 통합되어 수렴하는 지점이다. 물리학, 화학, 생물학, 인지과학, 사회과학, 그리고 그것들 사이의 창발적 연결까지---모든 분과 지식과 그 통합이 하나의 체계 안에 포함된다.\footnote{인류원리(anthropic principle)는 ``왜 이 물리 상수/법칙인가''에 대해 ``관측자가 존재할 수 있는 우주만 관측된다''고 답한다. 그러나 인류원리는 우주 자체의 구조에 관한 것이지, 그 구조가 우리의 지식으로 표상되는 과정에 관한 것이 아니다. 뉴턴이 프린키피아를 쓰지 않았더라도 물리 상수는 (실재한다면) 존재한다. 자기도출은 이 추가 조건---법칙이 왜 특정한 물리적 형태(논문, 경전, 신경 패턴)로 표상되는가---까지 요구한다. 인류원리와 자기도출의 구조적 비교는 3절을 참조.}

$M_{\text{unified}}$는 이 내용적 고정점을 목표로 한다:
\begin{equation}
M_{\text{unified}}(x) = \{M_{\text{unified}}, D\}
\end{equation}

형식적 고정점에서 내용적 고정점으로의 이행은 \textbf{재귀적 통합}을 통해 이루어진다. 제도적 지식이 자기 자신을 대상으로 삼고, 그 결과를 다시 통합하고, 그 통합을 다시 대상으로 삼는 과정이 반복된다. 이 재귀적 적용이 수렴하면 내용적 고정점에 도달한다.

여기서 두 가지 가능성이 열린다:
\begin{itemize}
    \item 이 과정이 내용적 고정점에 도달할 수 있다면, 과학의 언어로 자기도출 구조를 갖는 통합이론이 존재하게 된다.
    \item 이 과정이 내용적 고정점에 도달할 수 없다면, 그러한 통합이론은 원리적으로 존재하지 않는다.
\end{itemize}

어느 쪽이 참인지는 열린 문제로 남는다.\footnote{대규모 언어 모델은 인류의 제도적 지식 전체를 학습 데이터로 삼는다는 점에서, 이러한 재귀적 통합의 시험적 실현 가능성을 제공한다. 이 시스템이 제도적 지식의 통합을 시도할 때, 형식적 고정점에서 내용적 고정점으로의 이행이 가능한지를 경험적으로 탐색해볼 수 있다.}


\subsection{계층 모델: $M_{\text{unified}}$의 구조}

$M_{\text{unified}}$가 자기도출을 달성하려면, 과학을 특정한 방식으로 재배열해야 한다. 

자기도출의 핵심은 ``내용이 존재를 도출한다''는 것이다. 이것을 계층적으로 구현하려면, 각 레벨에서 두 가지를 구분해야 한다:
\begin{itemize}
    \item \textbf{코드}: 그 레벨의 규칙, 법칙, 정보를 담은 것
    \item \textbf{데이터}: 그 규칙이 작동하는 대상, 현상
\end{itemize}

그리고 결정적으로, 한 레벨의 코드가 다음 레벨의 데이터가 되는 구조가 필요하다. 이렇게 \textbf{코드와 데이터의 반복}을 통해 쌓아올려야, 최종적으로 ``코드가 자기 자신을 데이터로 삼는'' 고정점에 도달할 수 있다.

$M_{\text{unified}}$의 완전한 구조는 세 가지 요소로 구성된다:
\begin{enumerate}
    \item \textbf{포인터 구조}: 계층표 자체. 레벨 간 관계와 고정점.
    \item \textbf{각 학문의 내용}: 물리학, 생물학, 인지과학 등 각 레벨의 구체적 이론.
    \item \textbf{창발의 원리}: 레벨 간 전환 메커니즘. ``양자장에서 어떻게 물질이 창발하는가'', ``신경 활동에서 어떻게 의식이 창발하는가''.
\end{enumerate}

초기조건 $x = (\emptyset, \mathcal{P})$에서 시작하여, ``멱집합을 무한 적용하라''는 명령어가 실행되면서 계층이 전개된다. 이 계층은 집합론에서 출발하여, 수론, 계산이론, 물리학, 생물학, 인지과학, 사회과학을 거쳐 제도적 지식(과학 자체)에 도달한다. 최종 레벨에서 $M$은 자기 자신을 대상으로 삼는 고정점을 형성한다.\footnote{이 계층의 예시적 도식은 부록 A에서 제시한다. 각 레벨의 구체적 내용(존재론적 대응, 학술적 대응, 코드/데이터의 구체적 형태)은 본 논문의 주장이 아니며 논증의 대상도 아니다. 도식의 목적은 ``자기도출이 가능하려면 어떤 형식의 구조가 필요한가''를 직관적으로 보여주는 것이다.}

본 논문이 주장하는 것은 계층의 구체적 내용이 아니라, 형식적 구조에 관한 것이다:
\begin{enumerate}
    \item 자기도출이 가능하려면 계층 구조가 필요하다.
    \item 이 구조를 통해 고정점에 도달해야 한다.
    \item $M_{\text{unified}}$는 기존 과학에 이러한 메타구조를 부여한 것이다.
\end{enumerate}


\subsection{이것은 철학적 정당화가 아니다}

한 가지 오해를 방지해야 한다. 이 계층 모델과 자기도출의 구조는 과학의 \textbf{철학적 정당화}가 아니다.

뮌하우젠 트릴레마가 보여주듯, ``과학은 왜 참인가''에 대한 궁극적 정당화는 불가능하다. 본 논문이 시도하는 것은 정당화가 아니다.

본 논문이 시도하는 것은 \textbf{과학의 언어로 수행한 종교적 논증의 형식}이다.

종교가 하는 것:
\begin{quote}
``인격신이 있다. 그 속성상 계시가 있을 것이다. 이 경전이 그 계시다.''
\end{quote}

과학이 할 수 있는 것:
\begin{quote}
``공집합과 멱집합 함수가 있다. 계층적 창발을 통해 제도적 지식이 나온다. 이 이론이 그 제도적 지식이다.''
\end{quote}

둘 다 같은 형식의 자기도출 논증이다. 내용만 다를 뿐.


%============================================================
\section{트릴레마로의 확장}
%============================================================

\subsection{두 전략}

$M_{\text{unified}}$가 실제로 작동하려면 무엇이 필요한가? 부록 A의 계층표를 다시 보자. 이 표는 본질적으로 \textbf{포인터(pointer)의 체계}다. 표는 ``Level 5는 생명이고, 분자생물학에 대응한다''고 지시하지만, 분자생물학의 구체적 내용---단백질 접힘, 유전자 발현 조절, 진화적 동역학---은 포함하지 않는다.

자기도출은 포인터 구조(요소 1)만으로 달성된다. 포인터 구조가 자기 자신에게 이르는 경로를 제공하기 때문이다. 문제는 외부 설명력이다. 여기서 $M_{\text{unified}}$는 두 가지 전략 중 하나를 택해야 한다.


\subsection{전략 1: 포인터만 유지}

계층표를 포인터 체계로만 유지한다. 요소 1만 포함하고, 요소 2와 3은 포함하지 않는다.

이것은 성경의 전략과 동형이다. ``태초에 하나님이 천지를 창조하시니라''는 빅뱅의 물리학, 항성 진화, 생명의 기원을 모두 ``창조하시니라''로 압축한다. 단순하다.

그러나 외부 설명력이 낮아진다. 포인터 구조는 현상을 분류하고 조직화하지만, 현상의 구체적 구조를 설명하지 못한다.

행성 운동의 실제 구조---태양 중심, 타원 궤도, 역제곱 법칙---는 ``신의 섭리''나 ``Level 4에서 창발''이라는 포인터로 설명되지 않는다. 현상을 설명하려면 방대한 사후적 해석이 필요하다.

\begin{center}
\begin{tabular}{lccc}
\toprule
\textbf{전략 1} & 단순성 & 외부 설명력 & 자기도출 \\
\midrule
포인터만 & 높음 & 낮음 & 있음 \\
\bottomrule
\end{tabular}
\end{center}

포인터만 유지한 $M_{\text{unified}}$는 종교와 구조적으로 동형이 된다.


\subsection{전략 2: 내용 채우기}

각 포인터의 내용을 실제로 채운다. 요소 1, 2, 3을 모두 포함한다.

각 레벨의 실제 내용을 채우려면 방대한 정보가 필요하다:
\begin{itemize}
    \item Level 3--4: 양자역학, 상대성 이론, 표준 모형 전체
    \item Level 5--6: 분자생물학, 진화론, 신경과학 전체
    \item Level 7--8: 인지과학, 언어학, 사회과학 전체
    \item Level 9: 과학사, 종교학, 제도 이론 전체
\end{itemize}

더욱이, 개별 분과 지식의 총합만으로는 부족하다. 레벨 간 전환---창발의 원리---을 명시해야 한다. ``양자장에서 어떻게 물질이 창발하는가'', ``물질에서 어떻게 생명이 창발하는가'', ``신경 활동에서 어떻게 의식이 창발하는가''---이 전환들은 현재 과학에서 가장 어려운 미해결 문제들이다.

$M_{\text{unified}}$는 백과사전 이상이다. 백과사전에 없는 것---분과들 사이의 창발적 연결---까지 포함해야 한다.

\begin{center}
\begin{tabular}{lccc}
\toprule
\textbf{전략 2} & 단순성 & 외부 설명력 & 자기도출 \\
\midrule
내용 채우기 & 낮음 (폭발) & 높음 & 있음 \\
\bottomrule
\end{tabular}
\end{center}


\subsection{트릴레마의 정식화}

두 전략을 종합하면, $M_{\text{unified}}$는 딜레마에 직면한다:

\begin{center}
\begin{tabular}{lcccc}
\toprule
\textbf{전략} & 단순성 & 외부 설명력 & 자기도출 & \textbf{귀결} \\
\midrule
포인터만 & 높음 & 낮음 & 있음 & 종교와 동형 \\
내용 채우기 & 낮음 & 높음 & 있음 & 복잡성 폭발 \\
\bottomrule
\end{tabular}
\end{center}

어느 전략을 택하든, 세 조건을 동시에 만족할 수 없다.

\begin{theorem}[자기도출 트릴레마]
광범위한 영역에서 작동하는 토대적 지식 시스템에서, 다음 세 조건 사이에는 구조적 긴장이 존재한다:
\begin{enumerate}
    \item 토대는 \textbf{단순}해야 한다
    \item 토대는 \textbf{외부를 잘 설명}해야 한다
    \item 토대는 \textbf{자기를 도출}해야 한다
\end{enumerate}

현존하는 지식 체계들은 이 세 조건 중 하나 이상을 희생한다. $M_{\text{unified}}$가 조건 3을 만족하려 할 때, 포인터 구조만 유지하면 조건 2를 위반하고, 내용을 채우면 조건 1을 위반한다.
\end{theorem}

자기도출 트릴레마는 뮌하우젠 트릴레마와 다른 성격을 갖는다. 뮌하우젠은 \emph{논리적} 불가능성을 보여준다---정당화를 시도하면 세 막다른 길 중 하나에 반드시 도달한다. 자기도출 트릴레마는 \emph{구조적 긴장}을 보여준다---세 조건을 동시에 만족시키는 체계가 현존하지 않으며, 그 이유는 다음과 같다.

\textbf{단순성과 자기도출의 상충}: 단순성은 공리의 최소화를 요구한다---가능한 한 적은 가정으로 출발해야 한다. 그러나 자기도출은 이론이 자기 존재를 함축하는 내용을 요구한다---공리가 ``왜 이 공리가 기술된 텍스트가 존재하는가''를 설명해야 한다. 자기 존재를 함축하려면 존재자(관찰자, 언어, 기록 매체 등)에 대한 내용이 공리에 포함되어야 하며, 이는 공리의 수를 필연적으로 증가시킨다. 단순성과 자기도출은 반대 방향을 가리킨다.

\textbf{외부 설명력과 자기도출의 상충}: 외부 설명력은 이론이 자기 존재 이외의 현상을 예측하고 설명할 것을 요구한다. 그러나 경험적으로 검증 가능한 예측을 내놓으면 반증 가능성이 생기고, 이는 이론의 존재를 ``확실한 착지점''으로 삼는 자기도출 구조를 약화시킨다. 6절에서 보았듯, 자기도출에 최적화된 구조---경전처럼 자기 존재를 함축하는 데 특화된 내용---는 외부 현상에 대해 새로운 예측을 산출하지 않는 경향이 있다.

\textbf{단순성과 외부 설명력의 상충}: 외부 현상을 설명하려면 내용이 풍부해야 하고, 내용이 풍부해지면 단순성이 희생된다. 이것은 오컴의 면도날과 설명적 포괄성 사이의 고전적 긴장이다.

이 상충들은 \emph{논리적} 모순이 아니라 \emph{목표 간 긴장}이다. 원리적으로 불가능하다는 증명은 없다. 그러나 현존하는 모든 지식 체계가 이 긴장 속에서 하나 이상을 희생한다는 관찰은, 동시 충족이 극도로 어렵다는 \emph{경험적 귀납}을 뒷받침한다. 세 조건 각각을 충족하려는 시도가 다른 조건의 충족을 어렵게 만드는 이유가 명확하다면, 이 패턴이 현존 체계들에서 반복적으로 나타나는 것은 우연이 아니라 설명 가능한 현상이다.

5절의 딜레마는 외부 설명력과 자기도출 사이의 트레이드오프였다. 트릴레마는 여기에 단순성을 추가한다. $M_{\text{unified}}$의 두 전략이 보여주는 것은, 자기도출을 달성하려는 시도가 다른 비용을 증가시키는 경향이 있다는 것이다.


\subsection{트릴레마 공간에서의 위치}

\begin{figure}[h]
\centering
\begin{tikzpicture}[scale=1.2, every node/.style={scale=0.9}]
    \coordinate (Top) at (0, 4);
    \coordinate (Left) at (-3.5, -2);
    \coordinate (Right) at (3.5, -2);

    \fill[gray!5] (Top) -- (Left) -- (Right) -- cycle;

    \draw[thick, gray] (Top) -- (Left);
    \draw[thick, gray] (Top) -- (Right);
    \draw[thick, gray] (Left) -- (Right);

    \node[circle, fill=blue!10, draw=blue, align=center, inner sep=3pt] at (Top) 
        {\textbf{단순성}};
    \node[circle, fill=red!10, draw=red, align=center, inner sep=3pt] at (Left) 
        {\textbf{외부 설명력}};
    \node[circle, fill=green!10, draw=green!60!black, align=center, inner sep=3pt] at (Right) 
        {\textbf{자기도출}};

    \node[rectangle, fill=white, draw=black, thick, rounded corners, align=center] 
        at (-1.75, 1.5) {\textbf{과학} \\ \footnotesize 자기도출 희생};
    \node[rectangle, fill=white, draw=black, thick, rounded corners, align=center] 
        at (1.75, 1.5) {\textbf{종교} \\ \footnotesize 외부 설명력 희생};
    \node[rectangle, fill=white, draw=black, thick, rounded corners, align=center] 
        at (0, -1.5) {\textbf{$M_{\text{unified}}$} \\ \footnotesize (내용 채우기) \\ \footnotesize 단순성 희생};
        
    \node[rectangle, fill=gray!20, draw=black, dashed, rounded corners, align=center] 
        at (2.8, 0) {\footnotesize $M_{\text{unified}}$ \\ \footnotesize (포인터만)};
    \draw[->, dashed, gray] (2.5, 0.6) -- (2.0, 1.1);
\end{tikzpicture}
\caption{자기도출 트릴레마 공간. 중심---세 조건 모두 만족하는 완벽한 토대---은 비어 있다.}
\label{fig:trilemma}
\end{figure}

각 시스템의 위치:
\begin{itemize}
    \item \textbf{과학}: 단순성-외부 설명력 변. 단순하고 외부 설명력이 높지만, 자기도출을 유예한다.
    \item \textbf{종교}: 단순성-자기도출 변. 단순하고 자기도출이 있지만, 외부 설명력이 낮다.
    \item \textbf{$M_{\text{unified}}$ (내용 채우기)}: 외부 설명력-자기도출 변. 외부 설명력과 자기도출이 있지만, 단순성을 상실한다.
    \item \textbf{$M_{\text{unified}}$ (포인터만)}: 종교와 동형으로 수렴.
\end{itemize}

핵심 통찰: $M_{\text{unified}}$가 종교의 한계를 회피하려면 내용을 채워야 하고, 내용을 채우면 단순성을 상실한다. 포인터 전략과 내용 전략 사이에 중간 지대는 없다.


\subsection{완벽한 토대는 없다}

트릴레마의 함의는 명확하다: \textbf{완벽한 토대는 없다.}

\begin{itemize}
    \item 과학은 ``왜 이 법칙인가''에 답하지 않는다---자기도출의 유예.
    \item 종교는 현상의 구체적 구조를 설명하지 못한다---외부 설명력의 희생.
    \item $M_{\text{unified}}$는 둘 다 하려 하지만, 인류 지식의 총합과 그 통합적 연결을 요구한다---단순성의 희생.
\end{itemize}

뮌하우젠 트릴레마는 정당화가 불가능함을 보여주었다. 자기도출 트릴레마는 준-정당화에도 한계가 있음을 보여준다. 정당화도 안 되고, 준-정당화도 완벽하지 않다.

그러나 이것은 비관적 결론이 아니다. 각 시스템은 트릴레마에서 서로 다른 비용을 희생하는 \textbf{전략적 선택}을 한다. 이 선택들은 서로 대체 불가능하다. 완벽한 토대는 없지만, 각 영역에서 잘 작동하는 토대들이 공존한다. 이것이 과학과 종교가 공존하는 구조적 이유다.


%============================================================
\section{논의}
%============================================================

\subsection{정보 동역학: 만물의 이론을 향한 방향}

본 논문의 분석은 만물의 이론 연구에 방향을 제시한다. 자기도출이 가능하려면, 이론의 내용(코드)과 이론의 존재(물리적 텍스트)가 같은 언어로 기술되어야 한다. 물질과 정보가 별개의 범주로 남아 있는 한, 내용에서 존재로 가는 경로는 원리적으로 열리지 않는다.

이 관점에서, 휠러(John Archibald Wheeler)의 ``it from bit''---물리적 존재가 정보로부터 나온다는 관점---은 단순한 철학적 사변이 아니라, 만물의 이론을 향한 \textbf{구조적으로 요청되는} 방향이다. 존재 자체가 정보라면, 코드가 자기 존재를 도출하는 경로가 원리적으로 가능해진다.

양자 정보 이론, 홀로그래픽 원리 등은 각자의 동기와 목표를 가지고 진행되는 연구 프로그램이다. 그러나 본 논문의 관점에서 보면, 이들은 공통적으로 정보를 물리적 실재의 근본에 놓는다는 점에서, 자기도출의 가능 조건을 충족하는 방향 위에 있다.

다만 이 방향에 대한 본격적인 탐구는 본 논문의 범위를 넘어선다. 본 논문이 제시하는 것은 ``자기도출이 가능하려면 어떤 조건이 필요한가''라는 구조적 분석이지, 그 조건이 물리학적으로 어떻게 충족될 수 있는가에 대한 이론이 아니다. 휠러의 프로그램과 자기도출의 관계에 대한 상세한 분석은 후속 연구의 과제로 남긴다.


\subsection{부트스트랩: 이미 작동하는 자기도출 구조}

2절에서 잠시 언급한 셀프-호스팅 컴파일러를 다시 보자. 이것은 $M_{\text{unified}}$와 같은 구조가 실제로 작동함을 보여준다.

C 컴파일러(GCC)는 C로 작성되어 자기 자신을 컴파일한다. 겉보기에 순환처럼 보인다---C 컴파일러를 만들려면 이미 C 컴파일러가 필요하지 않은가? 그러나 실제로는 순환이 아니라 \textbf{단계적 체인}이다:

\begin{center}
\begin{tabular}{l}
기계어 $\to$ 간단한 컴파일러 (C subset) \\
\hspace{2em} $\downarrow$ \\
중간 컴파일러 (더 많은 C 지원) \\
\hspace{2em} $\downarrow$ \\
완전한 컴파일러 (full C) \\
\hspace{2em} $\downarrow$ \\
self-hosting 달성
\end{tabular}
\end{center}

처음에는 기계어로 아주 제한된 C subset만 컴파일할 수 있는 간단한 컴파일러를 작성한다. 이 간단한 컴파일러로 더 나은 컴파일러를 컴파일하고, 그 컴파일러로 다시 더 나은 컴파일러를 컴파일한다. 사다리를 올라가면서 사다리 자체를 재건축하는 것이다. 최종적으로 완전한 컴파일러가 자기 자신의 소스코드를 컴파일할 수 있게 된다.

$M_{\text{unified}}$의 계층 모델과 비교하면:

\begin{center}
\begin{tabular}{ll}
\toprule
부트스트랩 & $M_{\text{unified}}$ \\
\midrule
기계어 (최소 초기조건) & 공집합 + 멱집합 (최소 초기조건) \\
C subset $\to$ 중간 C $\to$ full C & 집합 $\to$ 수 $\to$ ... $\to$ 제도적 지식 \\
self-hosting (자기 컴파일) & 고정점 (자기 도출) \\
\bottomrule
\end{tabular}
\end{center}

둘 다 bottom-up 전략이다. 최소한의 초기조건에서 시작하여, 단계적으로 복잡성을 쌓아올리고, 최종적으로 자기참조적 고정점에 도달한다. 셀프-호스팅 컴파일러는 이 구조의 제한된 도메인(C 언어)에서의 실제 인스턴스다. $M_{\text{unified}}$는 이 구조를 우주 전체로 확장한 것이다.

이것은 $M_{\text{unified}}$를 정당화하지 않는다. 그러나 중요한 것을 보여준다: \textbf{이런 구조 자체가 원리적으로 작동할 수 있다}. 남은 문제는 구조의 가능성이 아니라 스케일과 명시성이다---컴파일에서 각 단계는 명시적으로 정의된 변환이지만, $M_{\text{unified}}$에서 각 단계(창발)는 여전히 미해결 문제다.


\subsection{한계}

\textbf{정량화의 부재}: 본 논문은 ``단순성'', ``외부 설명력'', ``자기도출''을 질적으로만 다루었다. 이것들을 정량화하는 방법론은 후속 연구의 과제다.

\textbf{사례 선정의 제한}: 물리학과 아브라함 종교 전통(기독교, 이슬람)에 집중했다. 다른 과학(생물학, 경제학), 비아브라함 종교(불교, 힌두교), 다른 철학적 전통은 다른 구조를 보일 수 있다. 특히 인격신 없는 종교에서 자기도출 구조가 어떻게 작동하는지는 별도의 분석이 필요하다.

\textbf{$M_{\text{unified}}$의 미완성}: 계층 모델은 스케치일 뿐이다. 각 레벨 사이의 창발 관계를 엄밀하게 정당화하는 것은 별도의 연구 프로그램이다.


\subsection{후속 연구 방향}

\textbf{정량화와 형식화}: 본 논문은 ``비용''과 ``단순성''을 개념적으로 다루었으나, 이를 정보이론적으로 형식화할 가능성이 열려 있다. 표준적인 \textbf{최소 기술 길이(MDL, Minimum Description Length)} 원리는 모델 $M$이 데이터 $D$를 설명하는 비용 $L(D|M)$과 모델 자체의 복잡도 $L(M)$만을 고려한다. 이는 모델의 존재($M_{\text{object}}$)를 ``주어진 것''으로 간주하는 과학의 방법론적 태도를 반영한다.

이에 대해 본 논문은 모델이 자기 자신의 존재까지 설명 범위에 포함하는 \textbf{자기포함적 MDL(SI-MDL)} 프레임워크를 제안한다. 이는 비용 함수에 자기도출 항 $L(M_{\text{object}}|M_{\text{subject}})$를 추가함으로써, 왜 표준적인 과학적 설명이 ``이론의 존재 이유''를 설명하지 못하는지를 구조적으로 보여준다. 물론 존재론적 실재($M_{\text{object}}$)를 정보이론적 비트로 환원하는 것은---정보의 물리적 기반에 대한 입장(란다우어 원리 등)을 요구하는---난해한 문제이며, 이 항의 계산 가능성은 열린 문제로 남는다. 그러나 이 형식화 시도는 지식 체계의 구조적 맹점을 드러내는 발견법적(heuristic) 도구로서 가치를 가질 수 있다.

\textbf{다른 사례로의 확장}: 불교의 자기참조적 특성, 힌두교의 베다 권위 구조, 생물학의 환원주의 등에 대한 분석.

\textbf{동역학적 확장}: 시스템들이 트릴레마 공간 내에서 어떻게 이동하는지 추적. 과학혁명, 종교개혁의 비용 재배치로서의 분석.


%============================================================
\section{결론}
%============================================================

\subsection{요약}

본 논문은 종교적 확신의 ``정당화''를 시도하지 않았다. 그러나 정당화 없이 확신이 어떻게 산출되는지를 \emph{개념적으로} 분석했다. 이것이 철학적 기여다---현상을 정확히 기술하는 개념 틀은, 그 자체로 이해를 진전시킨다.

본 논문은 ``정당화가 불가능한 세계에서, 강력한 확신은 어떻게 산출되는가''라는 질문에서 출발했다.

뮌하우젠 트릴레마는 궁극적 정당화가 불가능함을 보여준다: 무한퇴행, 순환논증, 독단적 중단. 정당화를 시도하는 한, 이 세 가지 막다른 길을 피할 수 없다.

그러나 아브라함 계열 유신론은 정당화 없이도 강력하게 작동해왔다. 본 논문은 그 메커니즘으로 \textbf{자기도출}을 제안했다. 자기도출은 이론의 내용($M_{\text{subject}}$)으로부터 이론의 존재($M_{\text{object}}$)를 도출하는 구조다. 존재는 부정 불가능한 경험적 사실이므로, 착지점이 된다. 이 착지점의 확실성이 내용에 대한 확신으로 전이된다.

자기도출은 정당화가 아니라 \textbf{준-정당화}다. 준-정당화는 새로운 인식론적 범주가 아니라, 확신이 산출되는 인식적 현상에 대한 기술이다. 자기도출은 정당화 게임 \emph{안에서} 경쟁하는 제4의 옵션이 아니라, 정당화 게임 \emph{이후}에 작동하는 별개의 메커니즘이다. 이 구조는 가설연역법과 형식적으로 동형이지만, 인식론적으로 동등하지는 않다---예측의 구체성과 반증 가능성에서 결정적 비대칭이 존재하며, 이 비대칭이 과학과 종교의 구조적 차이를 설명한다.

과학은 이 구조를 갖추고 있지 않다. 과학의 언어로 자기도출을 달성하려는 시도---$M_{\text{unified}}$---는 \textbf{트릴레마}에 직면한다. 단순성, 외부 설명력, 자기도출 사이에는 구조적 긴장이 존재하며, 현존하는 지식 체계들은 이 중 하나 이상을 희생한다.

뮌하우젠 트릴레마는 정당화의 불가능성을 보여준다. 자기도출 트릴레마는 준-정당화에도 한계가 있음을 보여준다. 완벽한 토대는 없다. 그러나 각 영역에서 잘 작동하는 토대들이 공존한다.


\subsection{기여}

\begin{enumerate}
    \item \textbf{개념적 기여}: 자기도출, 객관적 주관성, 준-정당화 개념을 도입했다. 특히 준-정당화를 새로운 인식론적 범주가 아닌, 확신 생성 메커니즘에 대한 기술적 개념으로 정립했다.

    \item \textbf{분석적 기여}: 자기도출을 유사 개념들(자기복제, 자기참조, 순환논증, 자기예언)과 체계적으로 구분했다. 순환논증과의 차이를 ``논리적 공허함''과 ``경험적 공허함''의 구분을 통해 명확히 했다.

    \item \textbf{진단적 기여}: 아브라함 계열 유신론에서 종교적 확신이 산출되는 메커니즘을 ``확실성의 전이''로 설명했다. 과학과 종교의 공존을 딜레마로, 통합 시도의 한계를 트릴레마로 정식화했다. 트릴레마를 논리적 불가능성이 아닌 구조적 긴장으로 제시했다.

    \item \textbf{방향 제시}: 과학적 자기도출의 조건($M_{\text{unified}}$)과 가능한 형식(계층 모델)을 스케치했다.

    \item \textbf{질문적 기여}: ``이론의 내용으로 이론의 존재를 도출할 수 있는가''라는 질문을 형식화했다. 이 질문은 토대적 지식을 주장하는 모든 시스템에 적용 가능하다.
\end{enumerate}

본 논문은 종교나 과학을 평가하지 않는다. 자기도출이 ``좋다'' 또는 ``나쁘다''고 주장하지 않으며, 어떤 지식 체계가 ``우월하다''고 판단하지 않는다. 본 논문이 수행한 것은 순수하게 기술적 분석---확신이 산출되는 구조의 해부---이다. 이 분석은 종교와 과학의 대화가 왜 어려운지, 그리고 그 어려움이 개인의 비합리성이 아니라 체계의 구조적 특성에서 비롯됨을 보여준다.


\subsection{전망}

완벽한 토대는 없다. 그러나 각 영역에서 작동하는 불완전한 토대들이 있다.

\begin{itemize}
    \item 과학은 ``어떻게''에 답한다. 자기도출을 유예한다.
    \item 종교는 ``왜''에 답한다. 외부 설명력을 희생한다.
    \item 둘 다 잘하려는 시도는 복잡성의 폭발을 감수해야 한다.
\end{itemize}

과학과 종교의 공존은 우연이 아니라 구조적 필연이다. 트릴레마는 한계를 보이지만, 동시에 \textbf{지식의 다양성이 불가피함}을 보인다.

\begin{center}
\textbf{완벽한 토대는 없다. 그러나 이것으로 충분하다.}
\end{center}


%============================================================
% 참고문헌
%============================================================

\bibliographystyle{plain}
\begin{thebibliography}{99}

\bibitem{albert1968}
Albert, H. (1968). \emph{Traktat über kritische Vernunft}. Tübingen: Mohr.

\bibitem{kant1781}
Kant, I. (1781). \emph{Kritik der reinen Vernunft}. Riga: Hartknoch.

\bibitem{kuhn1962}
Kuhn, T. S. (1962). \emph{The Structure of Scientific Revolutions}. University of Chicago Press.

\bibitem{popper1959}
Popper, K. R. (1959). \emph{The Logic of Scientific Discovery}. Hutchinson.

\bibitem{lakatos1970}
Lakatos, I. (1970). Falsification and the Methodology of Scientific Research Programmes. In Lakatos, I. \& Musgrave, A. (Eds.), \emph{Criticism and the Growth of Knowledge} (pp. 91--196). Cambridge University Press.

\bibitem{wheeler1990}
Wheeler, J. A. (1990). Information, physics, quantum: The search for links. In W. Zurek (Ed.), \emph{Complexity, Entropy, and the Physics of Information}. Addison-Wesley.

\bibitem{merton1948}
Merton, R. K. (1948). The Self-Fulfilling Prophecy. \emph{Antioch Review}, 8(2), 193--210.

\bibitem{sextus}
Sextus Empiricus. \emph{Outlines of Pyrrhonism}. Trans. R. G. Bury. Loeb Classical Library, 1933.

\bibitem{weber1922}
Weber, M. (1922). \emph{Wirtschaft und Gesellschaft}. Tübingen: Mohr. [영역: \emph{Economy and Society}, University of California Press, 1978.]

\bibitem{barrett2004}
Barrett, J. L. (2004). \emph{Why Would Anyone Believe in God?} AltaMira Press.

\bibitem{henrich2009}
Henrich, J. (2009). The evolution of costly displays, cooperation and religion: Credibility enhancing displays and their implications for cultural evolution. \emph{Evolution and Human Behavior}, 30(4), 244--260.

\bibitem{plantinga2000}
Plantinga, A. (2000). \emph{Warranted Christian Belief}. Oxford University Press.

\bibitem{alston1991}
Alston, W. P. (1991). \emph{Perceiving God: The Epistemology of Religious Experience}. Cornell University Press.

\bibitem{swinburne2004}
Swinburne, R. (2004). \emph{The Existence of God} (2nd ed.). Oxford University Press.

\end{thebibliography}


\appendix

\section{$M_{\text{unified}}$ 계층 모델 도식}

다음 표는 6절에서 논의한 $M_{\text{unified}}$의 계층 구조에 대한 \textbf{예시적 스케치}다. 각 레벨의 구체적 내용(존재론적 대응, 학술적 대응, 코드/데이터의 구체적 형태)은 본 논문의 주장이 아니며, 논증의 대상도 아니다. 표의 목적은 ``자기도출이 가능하려면 어떤 형식의 구조가 필요한가''를 직관적으로 보여주는 것이다.

\begin{table}[ht]
\centering
\small
\renewcommand{\arraystretch}{1.4}
\begin{tabular}{cllll}
\toprule
\textbf{Level} & \textbf{존재론적 대응} & \textbf{학술적 대응} & \textbf{데이터 ($D_n$)} & \textbf{코드 ($M_n$)} \\
\midrule
0 & 집합 (Set) & 집합론 & 구별 가능성 & 공집합 \\
1 & 수 (Number) & 수론 & \textbf{집합}의 구조들 & 비트 (Bit) \\
2 & 함수 (Function) & 계산이론 & \textbf{수}의 나열들 & Gödel 수 \\
3 & 양자장 (Quantum Field) & 양자장론 & 가능한 경로(\textbf{함수})들 & 라그랑지안 \\
4 & 물질 (Matter) & 물리학 & \textbf{양자 장}의 요동들 & 에너지-운동량 텐서 \\
5 & 생명 (Life) & 화학/분자생물학 & 분자(\textbf{물질}) 반응들 & 유전자 \\
6 & 유기체 (Organism) & 신경과학 & 세포(\textbf{생명}) 신호들 & 신경망 \\
7 & 주체 (Conscious Agent) & 인지과학 & 감각기관(\textbf{유기체}) 자극들 & 기억 \\
8 & 사회 (Society) & 사회과학/언어학 & \textbf{주체}의 의도들 & 자연어 \\
9 & 문명 (Civilization) & 법학/종교/과학 & \textbf{사회}적 복잡성 & 제도적 지식 \\
\midrule
\addlinespace
$9^*$ & (고정점) & 인식론/과학학 & 제도적 지식들 & 제도적 지식 \\
\bottomrule
\end{tabular}
\caption{계층 모델 스케치. Level 0의 공집합에서 시작하여, 멱집합 적용을 통해 계층이 전개된다. Level $9^*$에서 $M$이 자기 자신을 대상으로 삼는다.}
\label{tab:hierarchy}
\end{table}

이 도식에서 핵심적인 것은 다음 세 가지다:

\begin{enumerate}
    \item \textbf{시작점 ($L_0$)}: 공집합 $\emptyset$에서 출발한다. 이것이 ``부트스트랩 조건''이다.
    \item \textbf{중간 계층 ($L_1$ -- $L_9$)}: 각 레벨에서 이전 레벨의 코드가 현재 레벨의 데이터가 된다. 코드와 데이터의 반복을 통해 복잡성이 증가한다.
    \item \textbf{고정점 ($L_{9^*}$)}: 최종 레벨에서 코드가 자기 자신을 데이터로 삼는다. $M$이 $M$을 대상으로 하는 자기참조 구조가 완성된다.
\end{enumerate}

레벨 간 구체적인 창발 메커니즘---양자장에서 물질로, 신경망에서 의식으로---은 각 학문 분야의 연구 대상이며, 본 논문의 범위를 넘어선다.


\end{document}