\documentclass[12pt,a4paper]{article}

% --- 기본 패키지 ---
\usepackage[utf8]{inputenc}
\usepackage[T1]{fontenc}
\usepackage{kotex}
\usepackage{amsmath,amssymb,amsthm}
\usepackage{geometry}
\usepackage{setspace}
\usepackage{titlesec}
\usepackage[dvipsnames]{xcolor}
\usepackage[hidelinks]{hyperref}
\usepackage{graphicx}
\usepackage{booktabs}
\usepackage{array}
\usepackage{tikz}
\usetikzlibrary{arrows.meta, shapes, positioning}

% --- 문서 설정 ---
\geometry{a4paper, margin=2.5cm}
\onehalfspacing

% 정리 환경
\newtheorem{definition}{정의}[section]
\newtheorem{proposition}{명제}[section]
\newtheorem{theorem}{정리}[section]
\newtheorem{principle}{원리}[section]

\title{\textbf{자기도출과 확신의 구조}\\[0.5ex]
\large 정당화 너머에서 확신은 어떻게 창발하는가\\[1.5ex]
{\normalsize Self-Derivation and the Structure of Conviction:\\
How Conviction Emerges Beyond Justification}}

\author{임지백 \\
한양대학교 일반대학원 철학과}

\date{2025년 2월}

\begin{document}

\maketitle

\begin{abstract}
\noindent
정당화가 불가능한 세계에서, 강력한 확신은 어떻게 창발하는가? 뮌하우젠 트릴레마는 궁극적 정당화가 불가능함을 보여주지만, 인류의 절반 이상이 특정 믿음 체계에 깊은 확신을 유지한다. 본 논문은 이 간극을 설명하는 구조적 조건을 해명한다.

핵심 개념은 \textbf{자기도출}(Self-Derivation)이다: 이론의 내용이 참이라면 그 이론의 존재가 정합적으로 기대되는 구조. 자기도출은 정당화가 아니지만, 정당화와 유사한 인식적 효과---\textbf{준-정당화}(Quasi-Justification)---를 창발시키는 구조적 조건을 제공한다.

본 논문은 세 가지 사례를 통해 이 구조를 분석한다. (1) \textbf{종교}: 아브라함 계열 유신론에서 ``신이 존재한다면 계시가 있을 것이다''라는 추론은 경전의 존재를 경전의 내용으로부터 도출한다. (2) \textbf{철학}: 헤겔의 절대적 관념론은 ``절대정신이 자기를 인식하는 최고 형태가 철학''이라는 주장을 통해, 헤겔 철학의 존재를 그 내용으로부터 도출하는 구조를 갖는다. 이 구조는 마르크스의 역사유물론에 계승되었으며, 20세기 공산주의가 보인 확신의 양상은 이 구조의 효과를 시사한다. (3) \textbf{과학}: 현대 과학은 자기도출 구조를 갖추지 못하며, 이것이 과학이 ``확신''보다 ``잠정적 수용''을 권장하는 구조적 이유다.

자기도출은 종교 특수적 현상이 아니라, \textbf{토대적 지식 체계 일반}에 적용 가능한 구조적 분석이다. 이 구조를 아는 것이, 확신에 대한 비판적 거리를 확보하는 가능조건이다.

\vspace{0.3cm}
\noindent
\textbf{주요어:} 자기도출, 내용-존재 정합성, 준-정당화, 아브라함 계열 유신론, 헤겔 절대적 관념론
\end{abstract}

\newpage
\tableofcontents
\newpage


%============================================================
\section{서론}
%============================================================

\subsection{정당화의 한계와 남겨진 문제}

한스 알베르트(Hans Albert)의 뮌하우젠 트릴레마는 인식론의 근본적 난제를 정식화한다.\footnote{Albert, H. (1968). \emph{Traktat über kritische Vernunft}. Tübingen: Mohr. 이 트릴레마는 아그리파의 오양상(five modes)으로 거슬러 올라가며, 섹스투스 엠피리쿠스의 \emph{Outlines of Pyrrhonism}에서 고전적 형태를 발견할 수 있다.} 어떤 명제의 정당화를 시도할 때, 우리는 세 가지 막다른 길 중 하나에 도달한다: 무한퇴행, 순환논증, 또는 독단적 중단. 칸트의 선험철학은 이 난제를 회피하려는 가장 정교한 시도였다. 경험의 가능조건 자체로부터 지식의 토대를 끌어내려는 전략이었으나, 비유클리드 기하학의 정합성 증명(힐베르트, 1899)과 일반상대성이론(아인슈타인, 1916)은 유클리드 기하학의 선험적 필연성이라는 칸트적 전제를 붕괴시켰다.\footnote{칸트 수학철학에 대한 비유클리드 기하학의 함의에 관해서는 논쟁이 있다. 일부 학자들은 칸트의 초월적 관념론이 유클리드-비유클리드 모두와 양립 가능하다고 주장한다. 그러나 최소한, 유클리드 기하학이 경험적 공간의 유일한 기하학이라는 칸트의 주장은 유지되기 어렵다.}

뮌하우젠 트릴레마 이후, 인식론은 대체로 ``궁극적 정당화''라는 프로젝트 자체를 포기했다. 토대주의(foundationalism)는 특정 믿음들이 비추론적으로 정당화된다고 주장하지만, 이는 트릴레마의 독단적 중단에 해당한다. 정합주의(coherentism)는 믿음들 간의 상호 지지를 정당화로 삼지만, 이는 순환논증의 세련된 변형이다. 무한주의(infinitism)는 무한퇴행을 수용하려 하지만, 유한한 인지 주체에게 무한한 정당화 사슬은 완결될 수 없다.

그런데 주목할 만한 사실이 있다. 종교적 믿음 체계는 철학적 정당화의 부재에도 불구하고, 단순히 작동하는 것을 넘어 \emph{강력한} 확신을 산출해왔다. 이 확신은 ``좋은 삶의 지침''이라는 실용적 가치로 환원되지 않는다. 순교를 가능하게 하는 절대적 확신, 세속적 논증으로 흔들리지 않는 불가침의 믿음---이것은 어디서 오는가? 단순히 비합리성이나 인지적 편향으로 설명하기에는, 그 구조가 너무 체계적이고 그 역사가 너무 지속적이다.


\subsection{자기도출: 정당화 이후의 확신 가능 조건}

본 논문은 종교적 확신의 구조를 분석함으로써, 정당화와 구별되는 인식적 구조를 식별하고자 한다. 이 구조를 본 논문은 \textbf{자기도출}(Self-Derivation)이라 명명한다. 본 논문의 분석은 아브라함 계열 유신론---기독교, 이슬람, 유대교---에 초점을 맞춘다. 이 선택은 두 가지 근거에 기반한다. 첫째, 경험적 대표성이다. 퓨리서치센터(Pew Research Center)의 2025년 보고서에 따르면, 2020년 기준 기독교(28.8\%)와 이슬람(25.6\%)은 세계 인구의 절반 이상을 차지하며, 유대교(0.2\%)를 포함한 아브라함 계열 종교는 전체 종교 인구의 약 72\%에 해당한다.\footnote{Pew Research Center (2025). ``How the Global Religious Landscape Changed from 2010 to 2020.'' 이 비율은 무종교 인구(24.2\%)를 제외한 수치다.} 둘째, 구조적 동형성이다. 이들 전통은 ``인격신 $\to$ 계시 $\to$ 경전''이라는 공통 구조를 공유하며, 자기도출의 작동 방식을 가장 명확하게 보여준다.\footnote{불교, 힌두교 등 비아브라함 종교에서는 다른 구조가 작동할 수 있다. 예컨대 불교의 사성제와 팔정도는 인격신의 계시가 아닌 붓다의 깨달음에 기반하며, 텍스트의 권위 구조가 상이하다. 이에 대한 분석은 별도의 연구를 요한다.}

자기도출의 구조는 다음과 같다. 이론 $M$은 두 측면을 갖는다: 이론이 \emph{말하는} 내용($M_{\text{subject}}$)과 이론이 \emph{존재한다는} 사실($M_{\text{object}}$). 자기도출은 전자로부터 후자가 정합적으로 기대되는 구조다. 종교적 텍스트에서 이 구조가 작동한다. ``창조물을 사랑하는 인격신이 있다''는 내용($M_{\text{subject}}$)을 전제하면, 그 신은 자기를 알리고자 할 것이고, 따라서 계시가 있을 것이며, 그 계시를 담은 텍스트가 존재할 것이라는 기대가 자연스럽다. 그리고 실제로 그런 텍스트---성경, 쿠란---가 존재한다($M_{\text{object}}$). 이것은 연역적 필연이 아니라, 신학적 전제 안에서의 내적 정합성이다.

이것은 순환논증과 다르다. 순환논증은 내용들 사이에서 돈다: ``신이 존재한다, 왜냐하면 성경에 그렇게 쓰여 있으니까; 성경은 진리다, 왜냐하면 신의 계시이므로.'' 모든 항이 검증 불가능한 명제적 주장이며, 외부의 착지점이 없다. 반면 자기도출에서 $M_{\text{object}}$---텍스트의 존재---는 부정 불가능한 경험적 사실이다. 성경이 존재한다는 것은 논쟁의 대상이 아니다. 자기도출은 명제적 내용으로부터 경험적 존재로의 범주 전환을 포함하며, 이 전환이 착지점을 제공한다.

물론 자기도출은 정당화가 아니다. 이론의 내용이 참임을 증명하지 않는다. 자기도출은 정당화 게임 \emph{안에서} 경쟁하는 제4의 옵션이 아니라, 정당화 게임 \emph{이후}에 작동하는 별개의 구조다. 본 논문은 이를 \textbf{준-정당화}(quasi-justification)라 부른다. 준-정당화는 믿음을 정당화하지 않지만, 믿음에 대한 확신이 가능해지는 조건을 제공한다. 이것은 규범적 인식론의 범주가 아니라 인식적 현상의 구조적 분석이다.

본 논문의 분석은 순수하게 기술적(descriptive)이다. 자기도출이 ``좋다'' 또는 ``나쁘다''고 평가하지 않으며, 종교적 믿음이 ``정당하다'' 또는 ``부당하다''고 주장하지 않는다. 본 논문이 수행하는 것은 확신이 가능해지는 \emph{구조적 조건}의 분석이다.

\textbf{본 논문의 방법론적 입장}: 본 논문은 심리학적 인과 메커니즘이 아니라 \textbf{구조적 가능조건}을 분석한다. ``확신이 어떻게 발생하는가''(인과적 물음)가 아니라 ``확신이 어떤 구조 위에서 가능한가''(조건적 물음)에 답한다. 전자는 인지과학의 영역이며, 본 논문의 범위를 넘어선다. 본 논문이 제시하는 것은 ``이 구조가 있으면 확신이 가능해진다''는 가능조건 분석이지, ``이 구조가 확신을 인과적으로 산출한다''는 심리학적 주장이 아니다.

이 분석이 의미 있는 이유는 두 가지다. 첫째, 종교적 확신이 왜 외부의 비판에 강건한지를 설명함으로써, 종교-세속 대화의 구조적 조건을 이해하는 데 기여한다. 둘째, 과학이 자기도출을 달성하려 할 때 어떤 한계에 직면하는지를 보여줌으로써, 만물의 이론(Theory of Everything, 이하 ToE)에 대한 메타이론적 성찰을 제공한다.\footnote{물리학에서 ToE는 통상 네 가지 기본 힘의 통합---대통일이론(GUT)의 확장---을 의미한다. 그러나 본 논문에서 ToE는 문자 그대로 ``만물''의 이론을 뜻한다: `만물'에는 그 이론 자체와 그 이론을 주장하는 과학자도 포함되어야 한다. 자기 자신을 설명하지 못하는 이론은 ``만물''의 이론이라 부를 자격이 없다는 것이 본 논문의 관점이다.}

이러한 기술적 분석은 규범적 인식론이 아니지만, 철학적으로 중요하다. 첫째, 기존 인식론이 간과한 확신의 구조적 조건을 \emph{형식적으로} 포착한다. 정당화의 성공 여부만 묻던 인식론에, ``정당화 없이 확신이 어떻게 가능한가''라는 질문을 추가한다. 둘째, 과학과 종교의 관계를 \emph{구조적 trade-off}로 설명함으로써, 환원주의적 해석(``종교는 비합리적이다'')과 호교론적 해석(``종교도 합리적이다'') 모두를 넘어서는 제3의 관점을 제공한다.


\subsection{본 논문의 구성}

본 논문은 자기도출 개념을 정립하고, 세 가지 사례를 통해 이 구조를 분석한다.

첫째, 자기도출 개념을 정립한다(3절). 자기도출은 순환논증, 자기참조, 자기예언, 자기복제 등 유사 개념들과 구분되는 고유한 구조를 갖는다. 이 구조가 뮌하우젠 트릴레마의 세 옵션과 어떻게 다른지를 명확히 한다.

둘째, 세 가지 사례를 분석한다. (4절) 종교---아브라함 계열 유신론---의 자기도출 구조가 종교적 확신을 어떻게 가능하게 하는지 분석한다. (5절) 철학---헤겔의 절대적 관념론---이 유사한 구조를 가지며, 19세기 헤겔주의의 ``종교적'' 양상이 이 구조의 효과임을 보인다. (6절) 과학---현대 자연과학---은 자기도출 구조를 갖추지 못하며, 이것이 과학이 ``확신''보다 ``잠정적 수용''을 권장하는 구조적 이유임을 분석한다.

이 분석은 자기도출이 종교 특수적 현상이 아니라, \textbf{토대적 지식 체계 일반}에 적용 가능한 구조적 분석임을 보여준다. 확신은 정당화의 산물이 아니라, 특정한 구조적 조건 위에서 창발한다. 그리고 그 구조를 아는 것이, 확신에 대한 비판적 거리를 확보하는 가능조건이다.


%============================================================
\section{정당화의 한계}
%============================================================

\subsection{뮌하우젠 트릴레마}

어떤 명제 $A$를 정당화하려 한다고 하자. ``왜 $A$인가?''라는 질문에 답해야 한다.

한스 알베르트(Hans Albert, 1968)의 뮌하우젠 트릴레마는 이 시도가 세 가지 막다른 길 중 하나에 빠진다고 지적한다.\footnote{Albert, H. (1968). \emph{Traktat über kritische Vernunft}. Tübingen: Mohr. 이 논증은 섹스투스 엠피리쿠스가 보고한 아그리파의 오양상(five modes)으로 거슬러 올라간다. 다만 고대 피론주의가 판단중지(epoché)를 목표로 했다면, 알베르트는 가오류주의(fallibilism)를 옹호한다는 점에서 결론이 다르다.}

\textbf{무한퇴행}: $A$를 정당화하기 위해 $B$를 제시한다. ``왜 $B$인가?''라는 질문이 따라온다. $B$를 정당화하기 위해 $C$를 제시한다. 이 과정은 끝없이 이어진다.
\begin{equation}
A \leftarrow B \leftarrow C \leftarrow D \leftarrow \cdots
\end{equation}

\textbf{순환논증}: 어느 시점에서 이미 사용한 명제로 돌아온다. $A \leftarrow B \leftarrow C \leftarrow A$. 정당화가 순환을 이룬다. 아무것도 정당화하지 못한다.

\textbf{독단적 중단}: 어느 시점에서 ``$X$는 그냥 참이다''라고 선언하고 멈춘다. 근거 없는 선언이다.

이 셋 중 어느 것도 만족스럽지 않다. 무한퇴행은 완결되지 않고, 순환논증은 공허하며, 독단적 중단은 자의적이다.

그러나 알베르트의 결론은 회의주의적 판단중지가 아니다. 확실한 정당화가 불가능하다는 것은, 탐구를 포기하라는 것이 아니라 \textbf{확실성의 이상을 포기하라}는 것이다. 알베르트는 포퍼의 비판적 합리주의를 따라, 모든 지식은 원리적으로 오류 가능하며(fallible), 비판적 시험과 반증을 통해 점진적으로 개선될 수 있다고 주장한다. 교조주의---비판에 면역된 명제를 설정하는 것---가 문제이지, 잠정적 수용 자체가 문제는 아니다.


\subsection{칸트: 자연과학 토대의 확립 시도}

뮌하우젠 트릴레마는 1968년에 정식화되었지만, 그것이 포착하는 문제---지식의 궁극적 근거는 무엇인가---는 오래되었다. 근대 철학에서 이 문제를 가장 체계적으로 해결하려 한 것이 칸트의 선험철학이다.

칸트의 전략은 독창적이었다. 그는 ``경험의 가능조건 자체''에서 정당화를 끌어내려 했다. 시공간과 인과성은 우리가 세계를 경험하기 위해 \emph{반드시} 전제해야 하는 형식이다. 이것 없이는 경험 자체가 불가능하다. 따라서 유클리드 기하학과 뉴턴역학은 단순히 경험적으로 참인 것이 아니라, 경험의 조건 자체에 의해 \emph{필연적으로} 참이다.

뮌하우젠의 프레임으로 소급하면, 칸트는 세 옵션을 모두 회피하려 했다:
\begin{itemize}
    \item 무한퇴행이 아니다---경험의 조건에서 멈추므로
    \item 순환논증이 아니다---조건이 경험을 가능하게 하는 일방향 관계이므로
    \item 독단이 아니다---``그냥 참''이 아니라 경험의 필연적 조건이므로
\end{itemize}

그러나 칸트의 ``부정 불가능''은 경험 주체에게 부과된 \emph{초월론적} 불가능성이었다. 유클리드 기하학이 아닌 공간은 경험하는 존재인 한에서 \emph{생각할 수 없다}고 주장했다. 이것이 문제였다. 비유클리드 기하학이 등장하자, 생각할 수 없다던 것이 생각 가능해졌다. 상대성이론이 등장하자, 필연적이라던 것이 경험적으로 거짓이 되었다.

칸트의 실패가 가르치는 것: 초월론적 필연성에 기반한 정당화도 경험에 의해 뒤집힐 수 있다. 칸트 이후, 정당화는 더 이상 필연적 토대의 확립이라는 형태로는 유지되지 않았다.

20세기 과학철학---포퍼, 쿤, 라카토스---은 정당화주의(justificationism)의 실패를 인정하고, 오류주의(fallibilism)로 전환했다. ``이 이론이 궁극적으로 참인가''라는 질문은 ``잠정적으로 수용하되 비판과 반증에 열려 있는가''로 대체되었다. 과학은 정당화 게임을 떠났다.


\subsection{각 지식 체계의 대응}

현존하는 지식 체계들은 뮌하우젠 트릴레마에 어떻게 대응하는가?

\textbf{과학}은 무한퇴행을 \textbf{유예}한다. ``왜 $F=ma$인가?''라는 질문에 과학은 답하지 않는다. 더 근본적인 법칙으로 환원하는 시도는 있다---뉴턴역학을 상대론으로, 상대론을 끈이론으로. 그러나 가장 근본적인 법칙에 대해서는 ``왜''가 유예된다. 과학은 ``어떻게''에 답하지, ``왜''에 답하지 않는다.

\textbf{수학}은 독단적 중단을 \textbf{명시적으로 채택}한다. 공리는 증명 없이 받아들여진다. 그러나 수학은 이것을 문제로 보지 않는다. 공리의 ``참''을 주장하는 것이 아니라, ``공리가 참이면 $X$가 따라 나온다''를 탐구하기 때문이다. 수학은 정당화 게임 자체를 거부한다.

\textbf{종교}는 흔히 순환논증으로 비판받는다. ``신은 존재한다. 왜냐하면 성경에 그렇게 쓰여 있으니까. 그리고 성경은 진리다. 왜냐하면 신의 계시이므로.'' 신 $\to$ 성경 $\to$ 신. 이것은 내용들 사이의 순환이다.

그러나 이 비판은 종교가 작동하는 방식을 완전히 설명하지 못한다. 종교는 순환논증임에도 불구하고 강력하게 작동해왔다. 단순히 ``오류''라고 치부하기엔, 그 영향력이 너무 크고 지속적이다. 무언가 다른 구조가 있어야 한다.


\subsection{남은 질문}

뮌하우젠 트릴레마는 정당화의 불가능성을 보여준다. 칸트의 실패는 이 불가능성이 회피될 수 없음을 역사적으로 확인해준다.

그런데 흥미로운 사례가 있다. 뮌하우젠 남작은 자기 머리카락을 잡아당겨 늪에서 빠져나왔다는 허풍으로 유명하다. 영어권에서는 이 이야기의 변형으로 ``pull oneself up by one's bootstraps''---자기 부츠끈을 잡아당겨 스스로를 들어올린다---라는 표현이 생겼다. 둘 다 같은 메타포다: 자기 자신으로 자기를 들어올리는 것은 불가능하다. 뮌하우젠 트릴레마는 이 메타포에서 이름을 따왔다. 그런데 컴퓨터과학의 ``부트스트랩''은 같은 메타포를 정반대 의미로 사용한다---자기 자신으로 자기를 들어올리는 것을 \emph{실제로 한다}.

셀프-호스팅 컴파일러는 겉보기에 순환처럼 보인다. C 컴파일러는 C로 작성되어 있다. 그런데 C로 작성된 코드를 실행하려면 C 컴파일러가 필요하다. C 컴파일러를 만들려면 이미 C 컴파일러가 있어야 한다---전형적인 닭과 달걀 문제다. 그러나 실제로는 순환이 아니라 단계적 체인이고, \emph{작동한다}. 물론 이것은 뮌하우젠의 인식론적 문제를 직접 해결하지 않는다. 부트스트랩은 인과적 과정이고, 뮌하우젠이 다루는 것은 논리적 정당화다. 범주가 다르다. 그러나 이것이 시사하는 바가 있다: 겉보기 순환이 반드시 막다른 길은 아닐 수 있다. 순환처럼 보이지만 착지점이 있는 구조가 가능하다.

그리고 종교가 있다. 종교는 정당화 없이도 \emph{강력하게} 작동한다---단순히 ``좋은 삶의 지침''을 제공하는 수준이 아니라, 강도 높은 확신까지 가능하게 한다. 어떻게 가능한가?

본 논문은 이론의 \emph{존재} 자체를 출발점으로 이 질문에 답한다. 핵심 개념은 \textbf{자기도출}(self-derivation)이다: 이론의 내용이 참이라면 그 이론의 존재가 정합적으로 기대되는 \emph{구조}. 자기도출은 정당화가 아니다. 그러나 이 구조가 충족되면, 정당화와 유사한 인식적 효과---\textbf{준-정당화}(quasi-justification)---가 \emph{창발}한다.

과학과 종교의 차이는 방법론적 \emph{선택}이 아니라 \emph{구조적 특성}이다. 종교 경전은 자기도출 구조를 갖는다---경전의 내용이 경전의 존재를 설명한다. 이 구조는 의도적으로 설계된 것이 아니라, 경전이라는 텍스트가 갖게 된 특성이다. 반면 현재 과학에는 자기도출 구조가 없다---물리학은 물리학자의 존재를 설명하지 못한다. 이것은 과학의 선택이 아니라, 현재 과학의 설명 범위가 자기 자신에 도달하지 못하기 때문이다.\footnote{만약 미래의 과학이 물질에서 생명으로, 생명에서 의식으로, 의식에서 과학자로 이어지는 연속적 설명 경로를 확보한다면---진정한 만물 이론(Theory of Everything)이 된다면---그때 과학도 자기도출 구조를 갖게 되고, 준-정당화가 창발할 것이다. 이에 대해서는 6.3절에서 논의한다.}


%============================================================
\section{자기도출: 준-정당화의 구조}
%============================================================

\subsection{객관적 주관성: $M_{\text{subject}}$와 $M_{\text{object}}$}

모든 이론 $M$은 두 측면을 갖는다.

\begin{definition}[객관적 주관성]
이론 $M$의 두 측면:
\begin{itemize}
    \item $M_{\text{subject}}$: 이론의 내용. 이론이 \emph{말하는} 것. (주관적 주장)
    \item $M_{\text{object}}$: 이론의 존재. 이론이 \emph{있다는} 것. (객관적 사실)
\end{itemize}
\end{definition}

이것은 이론이 ``자기를 언급한다''는 것이 아니다. 이론이면 필연적으로 갖는 두 측면의 개념화다. 여기서 `존재'란 단순한 물리적 실재가 아니라, 의미 있는 기호 체계로서 사회적으로 기능하는 존재를 가리킨다.

\textbf{예시 1: 물리학}
\begin{itemize}
    \item $M_{\text{subject}}$: ``힘은 질량 곱하기 가속도다''라는 명제적 내용
    \item $M_{\text{object}}$: 프린키피아라는 물리적 텍스트가 존재한다는 사실
\end{itemize}

\textbf{예시 2: 종교}
\begin{itemize}
    \item $M_{\text{subject}}$: ``창조물을 사랑하는 인격신이 있다''라는 신학적 주장
    \item $M_{\text{object}}$: 성경이라는 물리적 텍스트가 존재한다는 사실
\end{itemize}

핵심적 차이: $M_{\text{subject}}$의 진리값은 논쟁 가능하다. 신이 존재하는지, $F=ma$가 궁극적으로 참인지는 논쟁의 대상이다. 그러나 $M_{\text{object}}$---이론의 존재---는 \textbf{부정 불가능}하다. 성경이 존재한다는 것, 프린키피아가 존재한다는 것은 누구도 부정할 수 없는 경험적 사실이다.

``객관적 주관성''이란 이 구조를 가리킨다: 이론의 내용은 주관적 주장이지만, 그런 주장을 담은 텍스트가 존재한다는 것은 객관적 사실이다.\footnote{본 논문의 ``객관적 주관성''은 나겔(Nagel)의 용법과 다르다. 나겔에게 ``objective subjectivity''는 주관적 경험을 객관적 관점에서 이해하는 것을 의미한다. 본 논문에서는 주관적 내용을 담은 텍스트의 객관적 실재를 지칭한다.}

칸트의 ``부정 불가능''이 논리적 필연성에 기반했다면, 자기도출의 ``부정 불가능''은 경험적 사실에 기반한다. 전자는 뒤집힐 수 있었다. 후자는 뒤집힐 수 없다.


\subsection{자기도출의 정의}

\begin{definition}[내용-존재 정합성]
이론 $M$이 \textbf{내용-존재 정합성}(content-existence coherence)을 갖는다는 것은, $M_{\text{subject}}$가 참이라면 $M_{\text{object}}$의 존재가 설명적으로 자연스럽다는 것이다. 이것은 연역적 함축이 아니라 개연적 정합성이다.
\end{definition}

그러나 내용-존재 정합성만으로는 자기도출이 성립하지 않는다. ``인간은 글을 쓴다''라는 문장을 생각해보자. 이 문장이 참이라면, 이 문장이 적힌 텍스트가 존재하는 것은 정합적이다---내용-존재 정합성이 있다. 그러나 이것을 자기도출이라 부르기는 어렵다. 왜인가?

``인간은 글을 쓴다''는 텍스트의 존재를 설명하지만, \textbf{그 외에는 아무것도 설명하지 않는다}. 텍스트의 존재 자체가 그 명제의 증거이자 전부다. 순환이 너무 좁다---자기 존재만을 가리키는 \textbf{자기참조}(self-reference)에 불과하다.

반면 ``인격신이 창조물을 사랑하고 자신을 알리고자 한다''는 내용은 광범위한 현상을 설명한다: 우주의 존재, 도덕 법칙, 삶의 의미, 죽음 이후. 그리고 이러한 설명들이 펼쳐지는 \textbf{경로의 연장선에서}, 경전의 존재에 자연스럽게 \textbf{도달}한다. 자기 존재는 설명의 부산물이 아니라, 설명 체계가 흘러가서 닿는 종착지다.

이것이 단순한 ``자기참조''(self-reference)나 ``자기설명''(self-explanation)이 아닌 ``자기도출''(self-derivation)이라는 용어가 적합한 이유다. ``이 문장은 거짓이다''(자기참조)는 자기 내용을 언급하지만, 외부 현상을 설명하지 않는다. ``이 텍스트는 인간이 썼다''(자기설명)는 자기 존재를 설명하지만, 오직 그것만을 설명한다. 자기도출은 단순히 자기 내용을 언급하거나 자기 존재를 설명하는 것이 아니라, 광범위한 현상을 설명하는 경로를 따라 자기 존재에 \emph{도달}하는 것이다.

\begin{definition}[자기도출]
이론 $M$이 \textbf{자기도출}(self-derivation) 구조를 갖는다는 것은 다음 두 조건을 충족하는 것이다:\footnote{여기서 ``정합적으로 기대된다''는 연역적 필연이 아니라 개연적 정합성을 의미한다. $M_{\text{subject}}$가 참이라면 $M_{\text{object}}$의 존재가 설명적으로 자연스럽다는 것이지, $M_{\text{subject}}$로부터 $M_{\text{object}}$가 논리적으로 강제된다는 것이 아니다. 이 구분은 본 논문 전체에서 유지된다.}
\begin{enumerate}
    \item \textbf{설명적 포괄성}: $M$의 내용($M_{\text{subject}}$)이 $M$ 자신의 존재 외에도 광범위한 현상을 설명한다.
    \item \textbf{연속적 도달}: 그러한 설명들이 펼쳐지는 경로의 연장선에서, $M$의 존재($M_{\text{object}}$)가 정합적으로 기대된다.
\end{enumerate}
\begin{equation}
M_{\text{subject}} \xrightarrow{\text{외부 설명}} \cdots \xrightarrow{\text{연속적 도달}} M_{\text{object}}
\end{equation}
\end{definition}

여기서 두 조건의 관계를 명확히 해야 한다. 자기도출은 외부 설명과 \textbf{대립}하는 것이 아니라 외부 설명의 \textbf{연장}이다. 이론이 광범위한 현상을 설명하는 경로를 따라가다가, 그 경로가 이론 자신의 존재에까지 도달할 때 자기도출이 성립한다. 외부 설명이 있어야 경로가 있고, 경로가 있어야 도달이 가능하다. 이것이 ``인간은 글을 쓴다''가 자기도출이 아닌 이유를 더 명확히 해준다---외부 설명 없이 바로 자기를 가리키므로, 도달이 아니라 참조에 불과하다.

이 정의에 따르면 자기도출, 자기참조, 일반 과학이 명확히 구분된다:

\begin{table}[ht]
\centering
\begin{tabular}{lccc}
\toprule
& 외부 설명력 & 자기 포함 & 구조 \\
\midrule
``인간은 글을 쓴다'' & \texttimes & \checkmark & 자기참조 \\
$F=ma$ & \checkmark & \texttimes & 일반 과학 \\
아브라함 유신론 & \checkmark & \checkmark & \textbf{자기도출} \\
헤겔 체계 & \checkmark & \checkmark & \textbf{자기도출} \\
\bottomrule
\end{tabular}
\caption{자기참조, 일반 과학, 자기도출의 구분}
\end{table}

과학이 자기도출 구조를 갖지 못하는 이유도 이제 명확하다. $F=ma$는 외부 현상을 잘 설명하지만, 그 설명 경로가 프린키피아의 존재에 도달하지 않는다. 물리법칙은 입자의 운동을 설명하지, 물리학자의 존재나 물리 논문의 존재를 설명하지 않는다.

자기도출의 또 다른 특성은 $M$이 자기 자신의 메타이론으로 기능한다는 점이다. 일반적으로 이론 $M$과 그 이론에 대한 메타이론 $M'$는 구분된다. 예컨대 물리학은 자연현상을 설명하지만, ``물리학 이론은 어떻게 구성되고 발전하는가?''라는 물음에는 과학철학(패러다임론, 반증주의 등)이 답한다---물리학 자체가 아니라. 그러나 자기도출 구조에서 $M$의 내용($M_{\text{subject}}$)은 $M$의 존재($M_{\text{object}}$)를 설명한다: ``신이 존재하므로 계시가 있다.'' 이것이 자기도출이 순환논증과 구별되는 지점이다---순환논증은 같은 논리적 수준에서 순환하지만, 자기도출은 대상 수준과 메타 수준을 횡단한다.

자기도출의 구조:
\begin{enumerate}
    \item \textbf{전제}: 이론의 내용($M_{\text{subject}}$)을 가정한다.
    \item \textbf{도출}: 그 내용으로부터 이론의 존재($M_{\text{object}}$)가 정합적으로 기대된다.
    \item \textbf{착지}: 실제로 이론이 존재함을 확인한다.
\end{enumerate}


\subsection{뮌하우젠의 세 옵션과의 비교}

자기도출이 뮌하우젠의 세 옵션과 어떻게 다른가? 여기서 중요한 구분이 필요하다. 순환논증의 문제는 두 가지 차원에서 분석될 수 있다:

\begin{itemize}
    \item \textbf{논리적 공허함}: 전제가 결론을 순환적으로 지지하여, 실질적인 정보를 제공하지 않는다.
    \item \textbf{경험적 공허함}: 모든 항이 검증 불가능한 명제적 주장이어서, 외부 세계와의 접점이 없다.
\end{itemize}

\textbf{순환논증과의 차이}: 순환논증은 내용들 사이에서 돈다. $A \to B \to A$. 모든 항이 내용(명제)이고, 서로가 서로를 지지할 뿐 외부의 착지점이 없다. 이 구조는 논리적으로도, 경험적으로도 공허하다.

자기도출은 내용에서 존재로 간다. $M_{\text{subject}} \to M_{\text{object}}$. 존재는 내용과 다른 범주다. 자기도출은 \textbf{논리적 공허함은 해결하지 못한다}---내용의 참을 증명하지 않기 때문이다. 그러나 \textbf{경험적 공허함은 해결한다}---부정 불가능한 경험적 사실에서 착지하기 때문이다. 이것이 순환논증과의 결정적 차이다.

\textbf{무한퇴행과의 차이}: 무한퇴행은 끝없이 이어진다. $A \leftarrow B \leftarrow C \leftarrow \cdots$. 완결이 없다. 자기도출은 존재에서 멈춘다. 존재라는 경험적 사실에서 착지한다.

\textbf{독단적 중단과의 차이}: 독단적 중단은 ``$X$는 그냥 참이다''라고 선언한다. 근거가 없다. 자기도출은 근거가 있다---이론의 내용으로부터 이론의 존재가 도출된다. 물론 이것이 이론의 \emph{참}을 증명하는 것은 아니다. 그러나 근거 없는 선언과는 다르다.


\subsection{준-정당화로서의 자기도출}

자기도출은 정당화인가? 아니다. 자기도출은 이론이 \emph{참}임을 증명하지 않는다.

``인격신이 있다''는 내용에서 ``성경이 존재한다''가 도출되고, 실제로 성경이 존재한다. 그러나 이것이 ``인격신이 있다''가 참임을 증명하는가? 아니다. 다른 이유로 성경이 존재할 수 있다. 인간의 심리적 필요, 사회적 기능 등. 형식적으로 보면, 이것은 후건 긍정(affirming the consequent)이다: $H \to P$이고 $P$이면 $H$다---논리적으로 타당하지 않다.

이 구조는 과학철학에서 말하는 가설연역법(hypothetico-deductive method)과 \emph{확증의 논리에서} 유사하다---예측이 충족되면 가설이 지지된다. 가설 $H$에서 예측 $P$가 도출되고, $P$가 관찰되었다. 이것이 $H$가 참임을 \emph{증명}하는가? 아니다. 다른 가설 $H'$도 $P$를 예측할 수 있기 때문이다. 형식적으로 가설연역법도 후건 긍정이다. 그러나 우리는 가설연역법을 ``오류''라고 부르지 않는다. ``지지''(confirmation)라고 부른다. 논리적으로 타당하지 않지만, 과학적 실천에서 핵심적 역할을 수행하기 때문이다.\footnote{가설연역법의 논리적 지위에 대한 논쟁은 광범위하다. 헴펠(Hempel)의 확증 이론, 베이즈주의, 가설추론(abduction) 등 다양한 접근이 있다. 본 논문은 이 논쟁에 개입하지 않는다. 다만 가설연역법이 ``증명''이 아닌 ``지지''의 논리로 기능한다는 점을 지적할 뿐이다.}

그러나 자기도출은 가설연역법과 \emph{인식론적으로} 동등하지 않다. 결정적 비대칭이 있다. 가설연역법에서 예측 $P$는 \textbf{구체적}이고 \textbf{반증 가능}하다. 일반상대성이론은 ``수성 근일점이 세기당 43초각 이동한다''를 예측했다. 이 예측이 틀렸다면---가령 50초각이 관측되었다면---이론은 반증되었을 것이다. 예측의 구체성이 이론에 제약을 부과한다.

반면 자기도출에서 $P$(``이 경전이 존재한다'')는 거의 모든 형태로 충족 가능하다. ``인격신이 있고 자신을 알리고자 한다''는 내용은 어떤 종류의 계시든---돌판, 두루마리, 인쇄본---예측과 양립한다. 예측이 포괄적이어서 실질적 제약력이 없다. 이것이 자기도출에서 반증 경로가 구조적으로 부재하는 이유다.

이 비대칭이 바로 과학과 종교의 인식론적 차이를 구조적으로 설명한다. 과학은 $P$의 구체성을 통해 강한 반증 가능성을 확보하지만, 자기도출의 존재론적 착지를 갖지 못한다. 종교는 $P$의 포괄성을 통해 자기도출 구조를 갖지만, 반증 가능성을 갖지 못한다. 이것이 두 체계의 구조적 차이다.

자기도출은 이 비대칭에도 불구하고 기능한다. 이론의 내용에서 이론의 존재가 도출되고, 이론이 실제로 존재한다. 이것은 이론을 \emph{증명}하지 않고 \emph{지지}한다. 본 논문은 이것을 \textbf{준-정당화}(quasi-justification)라 부른다.\footnote{본 논문이 ``pseudo-justification''(유사-정당화, 사이비-정당화)이 아닌 ``quasi-justification''(준-정당화)이라는 용어를 선택한 것은 의도적이다. ``Pseudo-''는 그리스어 \emph{pseudes}(거짓)에서 유래하며, 대상이 진짜인 척하지만 제대로 기능하지 않음을 함의한다(예: pseudoscience). 반면 ``quasi-''는 라틴어로 ``as if''(마치 ~인 것처럼)를 의미하며, 존재론적으로는 다르지만 기능적으로는 동등함을 가리킨다. 물리학의 quasi-particle이 대표적 사례다. Quasi-particle은 다체계(many-body system)에서 집단적 들뜸(collective excitation)이 마치 독립된 입자처럼 행동하는 현상이다. ``진짜 입자''는 아니다---전자나 양성자처럼 진공에서 독립적으로 존재할 수 없다. 그러나 운동량, 에너지, 스핀 등 입자의 물리적 속성을 가지고 입자\emph{처럼} 기능한다. 준-정당화도 마찬가지다: ``논리적 정당화''는 아니다---명제의 참을 증명하지 않는다. 그러나 확신이 가능해지는 조건을 제공한다는 점에서 정당화\emph{처럼} 인식적으로 기능한다. 이 구분은 본 논문의 기술적(descriptive) 입장과 일관된다---준-정당화는 ``거짓 정당화''가 아니라 ``정당화와 다른 방식으로 같은 기능을 수행하는 구조''다. 본 논문의 영문 부제 ``How Conviction \emph{Emerges} Beyond Justification''에서 `emerges'(창발하다)를 사용한 것은 이 유비를 반영한다: quasi-particle이 다체계에서 창발하듯, 확신은 자기도출 구조에서 창발한다---정당화로 환원되지 않으면서도.}

여기서 본 논문의 주장을 명확히 해야 한다. 준-정당화는 새로운 \textbf{인식론적 범주}가 아니다. 믿음이 정당화된다고 주장하는 것이 아니기 때문이다. 준-정당화는 \textbf{인식적 현상의 구조적 분석}이다. 정당화가 불가능한 세계에서, 확신이 어떤 구조 위에서 가능해지는가를 분석하는 것이다. 이것은 규범적 인식론이 아니라 기술적 분석이다.

자기도출은 정당화 게임 \emph{안에서} 경쟁하는 제4의 옵션이 아니다. 정당화 게임의 \emph{외부}에서 작동하는 별개의 구조다. 뮌하우젠 트릴레마는 ``참을 어떻게 증명할 것인가''라는 게임의 불가능성을 보여준다. 자기도출은 이 게임과 무관하게, ``확신이 어떤 구조적 조건 위에서 가능한가''라는 다른 질문에 답한다.

\textbf{기존 연구와의 관계}: 확신 생성에 대한 기존 논의는 각기 다른 \emph{분석 수준}(level of analysis)에서 이루어져 왔다:
\begin{itemize}
    \item \textbf{개혁주의 인식론}(Plantinga, 2000)은 \emph{경험}에 초점을 맞춘다---성령의 내적 증거(sensus divinitatis)가 신 믿음을 직접적으로 정당화한다고 주장한다.
    \item \textbf{자기-인증 교리}(self-authentication)는 \emph{속성}에 초점을 맞춘다---경전의 웅장함, 일관성, 도덕적 고양 등 내재적 속성이 권위를 확립한다.
    \item \textbf{인지종교학}(Barrett, 2004; Henrich, 2009)은 \emph{심리적 기제}에 초점을 맞춘다---HADD(과잉 행위자 탐지 장치), CREDs(신뢰성 강화 표시) 등 진화적으로 형성된 인지적 편향을 분석한다.
\end{itemize}

본 논문의 자기도출은 이들과 다른 층위---\textbf{메타적 구조}---를 분석한다. ``텍스트가 존재한다는 부정 불가능한 사실이 착지점이 되어, 내용에 대한 확신의 구조적 조건을 제공한다''는 분석은 경험, 속성, 심리적 기제 어느 것에도 환원되지 않는다. 기존 문헌에서 이 층위의 분석이 부재한 것은 결함이 아니라, 본 논문이 새로운 분석 수준을 도입하기 때문이다.\footnote{구체적으로: (1) 개혁주의 인식론은 신 믿음의 \emph{정당성}을 주장하지만, 본 논문은 정당화를 주장하지 않는다. (2) 자기-인증 교리는 텍스트의 \emph{내재적 속성}에 호소하지만, 본 논문은 텍스트의 \emph{존재 자체}를 착지점으로 삼는다. (3) 인지종교학은 심리적 \emph{인과 기제}를 분석하지만, 본 논문은 \emph{구조적 가능조건}을 분석한다. 종교경험 인식론(Alston, 1991; Swinburne, 2004)은 \emph{경험}의 자기-인증을 다루고, 준-신앙주의(Wittgenstein 해석)는 모든 합리성이 전제하는 \emph{범용적} hinge commitments를 논하는데, 본 논문은 경험이 아닌 \emph{텍스트}에서 출발하며 범용적 전제가 아닌 \emph{종교 특수적} 구조를 분석한다는 점에서 구별된다.}

가추(abduction)와의 비교도 유익하다. 가추는 현상 $E$가 주어졌을 때, $E$를 가장 잘 설명하는 가설 $H$를 추론하는 방법이다. ``땅이 젖어 있다''($E$)를 관찰하고, ``비가 왔다''($H$)를 추론하는 것이 예다. 자기도출도 형식적으로 유사해 보인다: 경전의 존재($M_{\text{object}}$)를 관찰하고, 신의 존재($M_{\text{subject}}$)를 추론한다.

그러나 결정적 차이가 있다. 가추에서 현상($E$)과 가설($H$)은 \textbf{존재론적으로 독립적}이다. 젖은 땅은 비와 별개로 존재하며, 스프링클러로도 땅이 젖을 수 있다. 이 독립성 때문에 다른 설명이 경쟁할 수 있고, ``땅이 젖어 있다''는 ``비가 왔다''를 \emph{지지}하되 결정하지 않는다.

자기도출에서도 경쟁 설명이 존재한다---경전은 인간의 심리적 필요나 사회적 기능으로도 설명될 수 있다. 그러나 결정적 차이가 있다. 가추에서 현상과 가설은 분리된 존재자다. 자기도출에서 $M_{\text{object}}$와 $M_{\text{subject}}$는 \textbf{존재론적으로 동일한 대상}이다. 성경이라는 물리적 객체($M_{\text{object}}$)는 ``신이 자신을 알리고자 한다''라는 내용($M_{\text{subject}}$)을 \emph{담지하는} 바로 그 객체다. 가추에서 가설은 현상의 \emph{해석}이다. 자기도출에서 가설은 현상의 \emph{구성요소}다.


\subsection{반증주의적 관점에서의 자기도출}

포퍼의 반증주의에 따르면, 가설 $H$가 예측 $O$를 산출할 때 $\neg O$가 관찰되면 $H$는 반증된다(후건 부정). 가설연역법에서 예측이 확인되면 가설은 \textbf{지지}(confirmation)된다---연역적으로 정당화되지는 않지만, 인식적 신뢰를 얻는다.

자기도출은 형식적으로 가설연역법과 동일한 구조를 공유한다: $M_{\text{subject}} \to M_{\text{object}}$이고 $M_{\text{object}}$이므로 $M_{\text{subject}}$가 지지된다. 그러나 차이는 형식이 아니라 \textbf{항의 존재론적 성격}에 있다.

가설연역법에서 예측 대상($O$)은 이론($H$)과 존재론적으로 독립적이다. 뉴턴 역학이 참이든 거짓이든, 행성의 궤도는 그 자체로 존재한다. 이론과 증거가 분리되어 있기 때문에, 증거가 이론을 반증하는 것이 가능하다.

그러나 자기도출에서 $M_{\text{object}}$---텍스트의 존재---는 $M_{\text{subject}}$---텍스트의 내용---의 물리적 실현이다. 성경은 ``신의 계시''라는 내용을 \emph{담고 있는} 바로 그 물리적 객체다. 이론과 증거가 존재론적으로 분리되지 않는다.

이 특성을 \textbf{존재의 객관적 주관성}이라 부를 수 있다---$M_{\text{object}}$가 객관적 사실이면서 동시에 $M_{\text{subject}}$를 담지하는 이중적 성격을 가리킨다. $M_{\text{object}}$는 객관적이다---물리적으로 존재하며, 부정 불가능하다. 그러나 동시에 $M_{\text{subject}}$를 담지한다는 점에서 주관적이다. 예측 대상이 이론 자체의 존재이기 때문에, 이론의 내용이 참이든 거짓이든 예측($M_{\text{object}}$)은 항상 충족된다. 이것이 반증 경로가 구조적으로 부재한 이유다.

이것이 자기도출의 ``결함''인지는 별개의 문제다. 어떤 설명이 대체되려면 더 나은 대안이 있어야 한다. 행성 궤도의 경우, ``신의 설계''는 뉴턴 역학으로 대체되었다---후자가 더 정밀한 예측을 제공했기 때문이다.

그러나 경전의 존재---왜 이런 내용을 담은 텍스트가 도출되었는가---에 대해서는 사정이 다르다. 과학은 종교 현상의 일반적 조건(사회적 필요, 심리적 기능)과 텍스트의 물리적 생산 및 전파는 설명할 수 있다. 그러나 왜 \emph{하필 이 특정 내용}이 도출되었는가, 그리고 그 내용이 왜 텍스트의 존재와 정합적인 구조를 갖는가는 충분히 설명하지 못한다.\footnote{``사회적 구성물''이라는 설명은 경전이 존재한다는 사실은 설명하지만, 왜 하필 ``인격신의 계시''라는 내용인지, 그리고 왜 이 내용이 텍스트의 존재를 함축하는 구조를 갖는지는 해명하지 못한다. 인지종교학의 연구들---예컨대 Boyer, Atran---이 종교적 관념의 인지적 기반을 탐구하고 있으나, 특정 텍스트의 특정 내용까지 설명하는 데는 이르지 못했다.}

자기도출은 이 공백을 채운다. ``인격신이 인간에게 자신을 알리고자 했다''는 내용($M_{\text{subject}}$)은 ``그 계시를 담은 텍스트가 존재할 것''이라는 예측을 산출하고, 그 예측은 경험적으로 충족된다($M_{\text{object}}$). 내용과 존재 사이에 내적 정합성이 성립한다.

자기도출이 반증 불가능함에도 확신의 구조적 조건으로 기능하는 이유는, 그것이 논리적으로 우월해서가 아니다. 현재로서 \textbf{내용과 존재의 정합성}을 설명하는 대안적 체계가 부재하기 때문이다. 대안이 등장하지 않는 한, 자기도출은 그 설명적 공백을 점유하며 확신의 기반으로 기능한다.


\subsection{유사 개념과의 구분}

자기도출은 여러 유사 개념들과 혼동될 수 있다. 체계적 구분이 필요하다.

\subsubsection{자기복제: 존재에서 존재로}

DNA는 자기복제가 가능하다. 그러나 DNA는 자기도출을 하지 않는다.

자기도출은 이론의 \emph{내용}을 전제했을 때 이론의 \emph{존재}가 예측되는 구조다. DNA 복제는 \emph{이미 존재하는} DNA가 복사본을 만드는 인과적 과정이다. DNA의 내용(염기서열)만 전제하고 DNA의 존재를 전제하지 않은 상태에서, DNA의 존재가 논리적으로 따라 나오지는 않는다.

\begin{center}
\begin{tabular}{ll}
자기복제: & 존재(A) $\to$ 존재(A') [인과적] \\
자기도출: & 내용($M_{\text{sub}}$) $\to$ 존재($M_{\text{obj}}$) [논리적]
\end{tabular}
\end{center}


\subsubsection{자기참조: 내용에서 내용으로}

괴델 수는 수학적 진술을 자연수로 인코딩하여, 수학이 수학에 대해 말할 수 있게 한다. 불완전성 정리의 자기참조 문장(``이 진술은 증명 불가능하다'')은 자기참조의 전형이다.

자기참조의 구조는 내용(A)이 내용(A) 자신을 직접 가리키는 것이다. 괴델 문장 $G$는 ``$G$는 증명 불가능하다''라고 말한다---$G$의 내용이 $G$ 자신에 대한 진술이다.

그러나 이것은 자기도출이 아니다. 두 가지 이유가 있다. 첫째, 괴델 문장은 자기 \emph{내용}을 가리킬 뿐, 자기 \emph{존재}를 도출하지 않는다. ``이 진술은 증명 불가능하다''는 내용을 전제해도, 그 문장이 왜 존재하는지는 따라 나오지 않는다. 둘째, 그리고 더 중요하게, 괴델 문장에는 \textbf{설명적 포괄성}이 없다. 괴델 문장은 자기 자신 외에는 아무것도 설명하지 않는다---3.2절에서 분석한 ``인간은 글을 쓴다''와 마찬가지로, 순환이 너무 좁다. 자기도출이 되려면 외부 현상을 설명하는 경로의 \emph{연장선에서} 자기 존재에 도달해야 한다.

\begin{center}
\begin{tabular}{ll}
자기참조: & 내용(A) $\to$ 내용(A) [동일 범주, 외부 설명 없음] \\
자기도출: & 내용($M_{\text{sub}}$) $\xrightarrow{\text{외부 설명}}$ $\cdots$ $\to$ 존재($M_{\text{obj}}$) [범주 전환, 연속적 도달]
\end{tabular}
\end{center}


\subsubsection{콰인 프로그램: 내용에서 내용으로 (인과적)}

콰인(quine)은 자기 자신의 소스코드를 출력하는 프로그램이다. 입력 없이 실행하면, 출력으로 자기 자신의 코드가 나온다.

콰인은 자기참조와 유사하지만, 결정적 차이가 있다:
\begin{itemize}
    \item 자기참조(괴델 수): 내용이 내용을 \textbf{가리킨다}. 논리적 관계.
    \item 콰인: 내용이 내용을 \textbf{생성한다}. 실행이라는 인과적 과정을 거친다.
\end{itemize}

콰인은 자기참조의 인과적 버전이라 할 수 있다. 그러나 콰인도 자기도출은 아니다. 콰인이 생성하는 것은 자기 \emph{내용}(소스코드)이지, 자기 \emph{존재}(그 프로그램이 왜 있는가)가 아니다. 콰인의 내용을 전제해도, 콰인이라는 프로그램이 왜 존재하는지는 따라 나오지 않는다.

\begin{center}
\begin{tabular}{ll}
콰인: & 내용(A) $\to$ 내용(A') [인과적] \\
자기도출: & 내용($M_{\text{sub}}$) $\to$ 존재($M_{\text{obj}}$) [범주 전환]
\end{tabular}
\end{center}


\subsubsection{순환논증: 내용에서 내용으로, 경유}

순환논증의 구조는 내용(A)이 내용(B)를 경유하여 다시 내용(A)로 돌아오는 것이다.

\begin{quote}
``신은 존재한다. 왜냐하면 성경에 그렇게 기록되어 있으니까. 그리고 성경의 기록은 모두 진리이다. 그것은 신의 계시이므로.''
\end{quote}

여기서 ``성경에 그렇게 기록되어 있다''는 그 기록이 \textbf{참이라는 명제 주장}이다. A(신의 존재) $\to$ B(성경의 진리성) $\to$ A(신의 존재)로 순환하며, 모든 항이 검증 불가능한 \emph{내용}(명제)이고, 서로가 서로를 지지할 뿐 외부의 착지점이 없다.

자기도출에서 ``성경이 존재한다''는 그 기록을 담은 \textbf{텍스트가 실재한다는 사실 주장}이다. 성경의 내용이 참인지는 논쟁 가능하지만, 성경이라는 텍스트가 존재한다는 것은 부정 불가능한 경험적 사실이다. 이것이 착지점이다.

\begin{center}
\begin{tabular}{ll}
순환논증: & 내용(A) $\to$ 내용(B) $\to$ 내용(A) [착지점 없음] \\
자기도출: & 내용($M_{\text{sub}}$) $\to$ 존재($M_{\text{obj}}$) [착지]
\end{tabular}
\end{center}


\subsubsection{자기예언: 내용에서 존재로, 다시 내용으로}

자기도출과 가장 혼동되기 쉬운 것이 \textbf{자기예언}(self-fulfilling prophecy)이다.\footnote{이 개념은 로버트 K. 머튼(Robert K. Merton)이 1948년 정립했다. 머튼은 W.I. 토마스의 정리---``사람들이 상황을 실재라고 정의하면, 그 결과에서 실재가 된다''---를 발전시켜, 처음에는 거짓이었던 정의가 그것을 믿는 행동을 유발하여 결국 참이 되는 현상을 분석했다.}

\textbf{뱅크런}: ``이 은행은 망할 것이다''라고 예언한다. 사람들이 이를 믿고 예금을 인출한다. 실제로 은행이 망한다. ``예언이 맞았다.'' 이것은 머튼이 제시한 전형적 사례다.

자기예언의 구조는 \textbf{내용 $\to$ 존재 $\to$ 내용}이다. 내용(예언)이 존재(사태)를 인과적으로 야기하고, 그 존재가 다시 내용(``맞았다'')을 강화한다.

자기도출과 결정적으로 다른 점 두 가지:

\textbf{첫째, 경로가 다르다.} 자기예언은 존재를 경유하여 내용으로 \textbf{돌아온다}. 자기도출은 존재에서 \textbf{착지하고 끝난다}. 경전이 존재한다고 해서 ``인격신이 있다''가 참이 되는 것은 아니다. 가설이 지지될 뿐이다.

\textbf{둘째, 대상이 다르다.} 자기예언에서 내용 = ``은행이 망할 것''(A), 존재 = 은행 파산(B). $A \neq B$. 자기도출에서 내용 = $M$이 말하는 것, 존재 = $M$이 있다는 것. \textbf{둘 다 같은 $M$의 두 측면}이다.

\begin{center}
\begin{tabular}{ll}
자기예언: & 내용(A) $\to$ 존재(B) $\to$ 내용(A) [순환, 대상 불일치] \\
자기도출: & 내용($M_{\text{sub}}$) $\to$ 존재($M_{\text{obj}}$) [착지, 동일 대상]
\end{tabular}
\end{center}


\subsubsection{인류원리: 존재에서 내용으로}

인류원리(anthropic principle)는 자기도출과 비교할 가치가 있다. 둘 다 ``왜 이것인가?''라는 질문에 대해 순환 없이 답하려는 시도이기 때문이다. 인류원리는 ``왜 이 물리 상수인가?''에 대해, ``관측자가 존재할 수 있는 우주만 관측된다''고 답한다. 자기도출은 ``왜 이 경전이 존재하는가?''에 대해, ``경전의 내용이 경전의 존재를 도출한다''고 답한다.

그러나 인류원리는 자기도출의 \textbf{어떤 조건도 충족하지 않는다}.

\textbf{첫째, 설명적 포괄성이 없다.} 인류원리는 물리 상수의 값만을 ``설명''할 뿐, 다른 현상을 설명하지 않는다. 광범위한 현상을 설명하는 경로가 없다.

\textbf{둘째, 방향이 반대다.} 자기도출은 내용($M_{\text{subject}}$)에서 존재($M_{\text{object}}$)로 간다. 인류원리는 존재(관측자)에서 출발하여 내용(상수의 조건)을 역으로 제한한다. 이는 \emph{관찰 선택 효과}(observational selection effect)라 불리며, 설명력이 부족한 동어반복(tautology)이라는 비판을 받는다.\footnote{Bostrom, \emph{Anthropic Bias} (2002); Leslie, \emph{Universes} (1989) 참조.}

\textbf{셋째, 자기 포함이 없다.} 인류원리에서 관측자와 물리 상수는 서로 다른 대상이다. 그리고 ``왜 인류원리라는 이론이 존재하는가''를 인류원리 자체가 설명하지 않는다. 인류원리는 물리학에 대한 \emph{메타적 관찰}이지, 자기 존재를 도출하는 구조가 아니다.


\subsubsection{허구적 서사: 외재적 원인}

종교에 대한 가장 흔한 비판 중 하나는 ``그것은 단순한 허구 아닌가?''라는 것이다. 본 논문은 종교적 주장의 진위를 판단하지 않는다. 그러나 구조적 관점에서, 종교 경전과 허구적 서사---예컨대 판타지 소설---는 분명히 다르다.

허구적 서사의 핵심 원리---마법 체계, 선택받은 자, 선악의 대결---는 그 책이 왜 존재하는지와 아무런 관련이 없다. 마법이 실재한다고 해서 마법에 관한 책이 존재해야 할 이유는 없다. 허구적 서사의 존재 원인(작가의 창작 의지)은 작품의 내용에 포함되지 않는다. 작가는 서사 세계 \emph{밖}에 있으며, 등장인물들은 작가를 알지 못한다.

반면, 아브라함 계열 유신론에서 핵심 원리(신의 사랑과 계시 의지)는 경전의 존재를 직접 도출한다---신이 창조물을 사랑하고 자신을 알리고자 \emph{하기에} 경전이 존재한다. 텍스트를 산출한 원인이 내용에 \emph{내재}한다.

흥미롭게도, 허구적 서사가 자기도출에 실패하는 구조적 이유는 현대 과학이 직면한 한계와 맞닿아 있다. 두 체계 모두 텍스트/이론의 생산 원인이 설명하고자 하는 내용의 \textbf{외부}에 위치한다. 허구에서 작가는 서사 세계 밖에 있고, 과학에서 이론은 자기 존재를 도출하는 경로를 갖추지 못했다(6절에서 상세히 논의). 이는 내용이 자기 자신의 존재 원인(계시 의지, 절대정신의 자기인식)을 \textbf{내재}하고 있는 종교나 헤겔 철학과 대비된다.


\subsubsection{여덟 개념의 비교}

\begin{table}[ht]
\centering
\begin{tabular}{lccl}
\toprule
개념 & 경로 & 성격 & 예시 \\
\midrule
자기복제 & 존재(A) $\to$ 존재(A') & 인과적 & DNA \\
자기참조 & 내용(A) $\to$ 내용(A) & 논리적 & 괴델 문장 \\
콰인 & 내용(A) $\to$ 내용(A') & 인과적 & 콰인 프로그램 \\
순환논증 & 내용(A) $\to$ 내용(B) $\to$ 내용(A) & 논리적 & ``신 $\because$ 성경 $\because$ 신'' \\
자기예언 & 내용(A) $\to$ 존재(B) $\to$ 내용(A) & 인과적 & 뱅크런 \\
인류원리 & 존재(A) $\to$ 내용(B) & 선택적 & 미세조정 \\
허구적 서사 & 내용(A) $\not\to$ 존재(B) & 단절됨 & 판타지 소설 \\
\textbf{자기도출} & 내용($M_{\text{sub}}$) $\to$ 존재($M_{\text{obj}}$) & 준-정당화 & 경전 / 헤겔 \\
\bottomrule
\end{tabular}
\caption{재귀적 구조들의 분류}
\end{table}

자기도출의 고유한 특징:
\begin{enumerate}
    \item \emph{내용}에서 \emph{존재}로 가는 경로가 있다. (자기복제, 자기참조, 콰인, 순환논증과 구분)
    \item 그 경로는 인과적 순환이 아니라 착지다. (자기예언과 구분)
    \item 내용과 존재가 \emph{같은 대상} $M$의 두 측면이다. (자기예언과 구분)
    \item 텍스트 산출의 원인이 내용에 \emph{내재}한다. (허구적 서사와 구분)
\end{enumerate}


%============================================================
\section{종교: 자기도출의 사례}
%============================================================

\subsection{자기도출의 기본 구조}

인격신을 전제하면 계시가 정합적으로 기대된다.\footnote{``정합적으로 기대된다''는 것은 논리적 필연이 아니다. 인격신이 있더라도 계시를 선택하지 않을 수 있다. 본 논문이 분석하는 것은, 아브라함 계열 신학 \emph{안에서} 이 추론이 내적으로 일관되게 작동한다는 구조적 사실이다.} 왜 단순한 ``제1원인''이 아니라 ``창조물을 사랑하는 인격신''인가?

\begin{itemize}
    \item 제1원인: 세계를 도출할 수 있다. 그러나 계시를 도출할 이유가 없다.
    \item 인격신: 창조물을 사랑한다 $\to$ 창조물에게 자신을 알리고 싶다 $\to$ 계시.
\end{itemize}

``사랑''과 ``계시 욕구''라는 인격적 속성이 $M_{\text{subject}}$(이론의 내용)에서 $M_{\text{object}}$(이론의 존재)로 가는 경로를 제공한다. 아리스토텔레스의 부동의 원동자나 이신론의 신은 이 경로를 갖지 않는다. 세계를 시작하게 했지만, 자기를 알릴 이유가 없다. 인격신만이 ``내용에서 존재로'' 가는 정합적 경로를 내장하고 있다.

그러나 계시가 반드시 텍스트일 필요는 없다. 계시는 구전, 성사(聖事), 직접 체험, 육화(肉化) 등 다양한 형태를 취할 수 있다. 자기도출의 강도는 \textbf{계시가 텍스트와 얼마나 동일시되는가}에 따라 달라진다.


\subsection{텍스트 절대성의 스펙트럼}

종교 전통들은 텍스트의 지위에서 스펙트럼을 이룬다.

\begin{table}[ht]
\centering
\begin{tabular}{llc}
\toprule
전통 & 텍스트의 지위 & 자기도출 강도 \\
\midrule
이슬람 & 쿠란 = 창조되지 않은 신의 말씀 자체 & 최강 \\
루터 개신교 & Sola Scriptura, 성경만이 유일한 권위 & 강함 \\
가톨릭 & 성경 + 전통 + 교도권 (삼중 권위) & 중간 \\
자유주의 신학 & 성경 = 인간의 신앙 경험 기록 & 약함 \\
\bottomrule
\end{tabular}
\caption{텍스트 절대성 스펙트럼}
\end{table}

\textbf{이슬람}: 쿠란은 ``신의 말씀 자체''(칼람 알라, كلام الله)다. 창조되지 않았고, 영원히 신과 함께 존재했다. 아랍어 원문 자체가 신성하며, 번역은 ``해석''일 뿐 쿠란이 아니다. $M_{\text{subject}}$(신의 말씀)와 $M_{\text{object}}$(쿠란 텍스트)의 동일성이 가장 직접적이다.

\textbf{루터 개신교}: 종교개혁의 핵심 원리 중 하나가 ``오직 성경''(Sola Scriptura)이다. 교황, 공의회, 전통의 권위를 거부하고, 성경만이 신앙과 실천의 유일한 규범이라고 선언했다. 텍스트의 권위를 극대화함으로써 자기도출 구조를 강화했다.

\textbf{가톨릭}: 성경, 전통(사도로부터 이어지는 구전 교리), 교도권(교황과 공의회의 가르침)이라는 삼중 권위를 인정한다. 텍스트가 유일한 착지점이 아니므로, 자기도출 구조가 분산된다.

\textbf{자유주의 신학}: 성경을 인간이 기록한 신앙 경험의 기록으로 본다. 역사적 비평의 대상이며, 문자적 영감을 부정한다. $M_{\text{subject}}$와 $M_{\text{object}}$의 연결이 가장 약하다.

요한복음 1:1-3(``태초에 말씀이 계시니라...'')은 이 스펙트럼 어디에서나 해석 가능하다. 신학적으로 ``말씀''(로고스)은 그리스도를 가리키며, 성경은 그 말씀의 ``증언''이지 말씀 자체가 아니다. 그러나 텍스트 절대성이 높은 전통에서는 이 구별이 실천적으로 희미해진다---성경이 곧 신의 말씀처럼 기능한다.


\subsection{가설: 텍스트 절대성과 확신의 강도}

이 분석에 기초하여, 본 논문은 다음 \emph{검증 가능한 가설}을 제안한다:

\begin{principle}[텍스트 절대성 가설]
텍스트 절대성이 높을수록, 종교적 확신의 강도가 높아진다.
\end{principle}

왜 그런가? 텍스트 절대성이 높을수록:
\begin{enumerate}
    \item $M_{\text{subject}}$와 $M_{\text{object}}$의 동일성이 강해진다.
    \item 착지점(텍스트의 존재)에서 내용(신학적 주장)으로의 구조적 연결이 더 직접적이다.
    \item 확실성 전이의 구조적 조건이 더 강력해진다.
\end{enumerate}

\textbf{경험적 검증 가능성}: 이 가설은 경험적으로 검증 가능하다. 구체적 예측과 검증 방법은 다음과 같다:

\begin{enumerate}
    \item \textbf{집단 간 비교}: 텍스트 절대성이 높은 종교 집단(이슬람 근본주의, 개신교 근본주의)에서 텍스트 절대성이 낮은 집단(자유주의 신학, 진보적 유대교)보다 확신 강도가 높을 것이다. 확신 강도는 확신 척도(예: 종교적 헌신 척도, 신앙 확신 설문)로 측정 가능하다.

    \item \textbf{텍스트 비판 노출 효과}: 경전 비판(텍스트 고증학, 역사적 비평)에 노출된 신자에서 확신이 약화되는 경향이 있을 것이다. 이는 텍스트 비판이 $M_{\text{subject}}$와 $M_{\text{object}}$의 동일성을 약화시키기 때문이다.

    \item \textbf{경전 부재 종교와의 비교}: 경전 없는 종교(애니미즘, 샤머니즘)에서는 자기도출 구조가 작동하지 않으므로, 확신의 질적 특성이 다를 것이다---존재론적 착지점이 부재하므로.
\end{enumerate}

이 가설은 현재 사변적이며, 본 논문은 가설의 \emph{검증}이 아닌 \emph{정식화}를 목표로 한다. 그러나 예비적 관찰은 가설과 양립하는 것처럼 보인다. 이슬람 근본주의와 개신교 근본주의는 텍스트의 문자적 권위를 강조한다. 반면 자유주의 신학은 ``확실히 신이 있다''보다 ``신을 향한 열림''을 말한다. 텍스트 절대성이 낮아지면 확신의 강도도 낮아지는 경향이 있다.\footnote{물론 이것은 단순화다. 종교적 확신에는 공동체, 의례, 개인적 경험 등 다른 요인도 작용한다. 본 논문이 주장하는 것은 텍스트 절대성이 \emph{하나의} 구조적 요인이라는 것이지, 유일한 요인이라는 것이 아니다. 이 다변인 맥락에서 텍스트 절대성의 독립적 기여를 분리하는 것이 경험적 검증의 과제가 될 것이다.} 본 가설의 엄밀한 검증---예컨대 텍스트 절대성 수준이 다른 종교 집단 간 확신 강도의 비교 측정---은 별도의 경험적 연구를 요한다.


\subsection{극단적 사례: 카리스마적 종파의 변형된 자기도출}

앞 절에서 텍스트 절대성이 확신의 강도와 상관한다고 보았다. 그러나 자기도출의 착지점($M_{\text{object}}$)이 반드시 텍스트여야 하는 것은 아니다. 자기도출 스펙트럼의 극단에서는 착지점 자체가 텍스트에서 \textbf{인간}으로 전이되는 변형된 형태가 나타난다.

막스 베버가 분석한 \textbf{카리스마적 권위}에 기반한 신흥 종교 운동이 그 예다.\footnote{베버는 카리스마적 권위를 ``특정 개인의 비범한 자질에 의해 성립하는 권위''로 정의했다. 이러한 권위는 전통적 권위나 합법적-합리적 권위와 달리 지도자 개인의 인격에 의존하며, 본질적으로 불안정하다.} 이들은 자기도출의 착지점($M_{\text{object}}$)을 불변하는 텍스트에서 가변적인 \textbf{교주의 육체적 존재}로 전이시킨다.

\begin{center}
\begin{tabular}{lll}
\toprule
& 제도화된 종교 & 카리스마적 종파 \\
\midrule
$M_{\text{subject}}$ (내용) & ``신이 계시하신다'' & ``예언된 자가 도래한다'' \\
$M_{\text{object}}$ (존재) & 경전이 존재한다 (Text) & \textbf{교주가 존재한다 (Body)} \\
\bottomrule
\end{tabular}
\end{center}

교주는 자신의 우연적 속성들---출생지, 이름, 수감 이력 등---을 경전의 상징과 자의적으로 결합하여, ``내가 곧 그 예언의 성취''라는 과적합(overfitting)된 가설연역 체계를 구축한다. 이 과정에서 발생하는 되먹임(feedback)으로 인해, 교주 본인이 가장 먼저 확실성의 전이를 경험한다. ``내가 여기 실존하고($M_{\text{object}}$), 내 삶이 저기에 기록되어 있다($M_{\text{subject}}$).'' 이로써 교주는 확신의 최초 수용자이자 발신자가 된다.

경전은 침묵하지만, 교주는 실시간으로 확신을 발산한다. 신도들을 매혹하는 것은 교주의 논리가 아니라, 자기도출 회로가 뿜어내는 \emph{존재론적 확신}이다. 교주가 노쇠하거나 병드는 객관적 모순 앞에서도 믿음이 붕괴하지 않는 이유는, 믿음의 토대가 교주의 `능력'이 아니라 교주의 `존재 그 자체'($M_{\text{object}}$)에 착지해 있기 때문이다. 존재는 논리적으로 반박될 수 없다.

이는 자기도출의 \textbf{변형된 형태}다. 불변하는 텍스트의 권위를 가변적인 인간의 육체로 \textbf{전유(appropriation)}하여, 확실성 전이 회로를 인간에게 직접 이식한 구조다.

역사적 사례들이 이 구조를 보여준다. 짐 존스(Jim Jones)의 인민사원(Peoples Temple)에서, 존스는 자신을 성경적 예언의 성취로 제시하며 점차 성경보다 자신의 ``살아있는 말씀''에 더 큰 권위를 부여했다.\footnote{Reiterman, T. \& Jacobs, J. (1982). \emph{Raven: The Untold Story of the Rev. Jim Jones and His People}. E.P. Dutton.} 데이비드 코레시(David Koresh)의 다윗파(Branch Davidians)는 더 명시적이다---코레시는 자신이 요한계시록의 ``일곱 인(Seven Seals)''을 해제할 유일한 권위자라 주장했고, 성경 해석의 권위가 텍스트에서 해석자 자신으로 완전히 이전되었다.\footnote{Tabor, J. \& Gallagher, E. (1995). \emph{Why Waco? Cults and the Battle for Religious Freedom in America}. University of California Press.} 두 사례 모두에서 $M_{\text{object}}$는 더 이상 ``경전의 존재''가 아니라 ``교주의 존재''가 되었다.

베버가 분석한 ``카리스마의 일상화(routinization of charisma)''---지도자 사후 권위가 제도나 전통으로 이전되는 과정---는 이 불안정한 구조가 제도화된 형태로 이행하는 한 경로다.


\subsection{확실성 전이의 구조: 종교적 확신의 가능조건}

자기도출 구조가 설명하는 것은 종교가 왜 \emph{그토록} 강력한 확신을 제공하는가이다. 단순히 ``좋은 삶의 지침''이라서가 아니다. 다른 체계와 비교할 수 없을 만큼 강도 높은 믿음---이것은 어디서 오는가?

핵심은 \textbf{확실성의 전이}다.

\begin{enumerate}
    \item \textbf{착지점은 부정 불가능하다}: ``경전이 존재한다''는 경험적 사실이다. 쿠란이 존재한다는 것, 성경이 존재한다는 것은 누구도 부정할 수 없다.
    \item \textbf{자기도출 구조 안에서, 이 착지점은 내용과 연결된다}: ``인격신이 있다''를 전제하면 ``경전이 존재한다''가 따라 나온다.
    \item \textbf{착지점의 확실성이 내용의 확신으로 전이된다}: ``경전이 존재하니까, 인격신이 있는 것이다.''
\end{enumerate}

이 전이는 논리적으로 타당하지 않다. 후건 긍정의 오류다. 그러나 이 구조는 그러한 전이가 \emph{가능해지는 조건}을 제공한다. 존재의 부정 불가능성이 내용에 대한 확신의 구조적 지지대가 된다.

그리고 텍스트 절대성이 이 전이를 증폭한다. 텍스트가 ``신의 말씀 자체''라면, 텍스트의 존재는 곧 신의 말씀의 존재다. 착지점과 내용 사이에 간극이 없다. 반면 텍스트가 ``신앙 경험의 기록''이라면, 텍스트의 존재는 신의 존재를 직접 함축하지 않는다. 전이가 약해진다.

자기도출은 아브라함 계열 유신론에서 확신이 가능해지는 \textbf{하나의} 구조적 조건이다. 그러나 이것이 종교적 확신의 \emph{유일한} 원천은 아니다. 종교학과 종교심리학은 다른 요인들을 제시한다:

\begin{itemize}
    \item \textbf{종교적 경험}: 신비 체험, 개심 경험, 기도 중의 현존 경험 등 직접적 경험이 확신의 원천이 될 수 있다.
    \item \textbf{의례적 실천}: 반복적 의례를 통해 확신이 신체에 각인될 수 있다.
    \item \textbf{공동체적 정체성}: 종교 공동체에의 소속감과 집단 정체성이 확신을 강화할 수 있다.
\end{itemize}

본 논문이 주장하는 것은 자기도출이 이러한 요인들과 함께---또는 독립적으로---작동하는 \emph{구조적 조건}이라는 것이다. 특히 텍스트 중심적 아브라함 계열 유신론에서, 자기도출은 확신이 가능해지는 \textbf{구조적 토대}를 제공한다. 다른 요인들이 이 토대 위에서 작동하거나, 또는 독립적으로 작동할 수 있다. 이들 사이의 관계에 대한 분석은 별도의 연구를 요한다.

자기도출이 제공하는 것은 \textbf{확신의 구조적 조건}이다. 왜 신자는 ``아마도 신이 있을 것이다''가 아니라 ``확실히 신이 있다''고 믿을 수 있는가? 착지점의 확실성이 그러한 확신의 구조적 지지대가 되기 때문이다. 그리고 텍스트 절대성이 높을수록 이 구조적 지지가 강력해진다.

\textbf{본 논문의 범위와 한계}: 본 논문은 확실성 전이의 \emph{구조적 조건}을 분석했다---착지점($M_{\text{object}}$)에서 내용($M_{\text{subject}}$)으로의 경로가 어떻게 구성되는가. 이 전이가 \emph{심리적으로} 어떻게 구현되는지---인지적 편향, 동기화된 추론, 신뢰 휴리스틱 등의 구체적 인과 기제---는 인지과학의 영역이며, 본 논문의 범위를 넘어선다. 본 논문이 주장하는 것은 자기도출 \emph{구조}가 이러한 심리적 전이가 작동할 수 있는 \emph{가능조건}을 제공한다는 것이다. 구조는 전이를 \emph{가능하게} 하지만, 전이의 \emph{실현}은 심리적 과정에 의존한다. 이 구분이 핵심이다: 본 논문은 ``이렇게 확신이 발생한다''(인과적 주장)가 아니라 ``이 구조가 있으면 확신이 가능해진다''(조건적 분석)를 주장한다.

이것은 종교를 옹호하는 것도, 비판하는 것도 아니다. 구조를 분석하는 것이다.


\subsection{서사적 도출경로: 자기도출의 강도}

지금까지의 분석에서 ``인격신 $\to$ 계시 $\to$ 경전''은 자기도출의 최소 구조로 제시되었다. 그러나 실제 경전은 이보다 훨씬 정교한 \textbf{서사적 도출경로}를 갖는다:

\begin{itemize}
    \item \textbf{창조}: 우주의 기원에 대한 설명 (``태초에 하나님이 천지를 창조하시니라'')
    \item \textbf{타락}: 왜 계시가 필요한가에 대한 설명 (에덴에서의 추방, 인간과 신의 단절)
    \item \textbf{계보}: 아담 $\to$ 노아 $\to$ 아브라함 $\to$ 모세로 이어지는 역사적 연속성
    \item \textbf{계시 사건들}: 시내산, 예언자들, 구체적인 역사적 맥락
    \item \textbf{최종 계시}: 그리스도(기독교) 또는 무함마드(이슬람)
    \item \textbf{경전의 편찬과 보존}: 텍스트가 현재 형태로 전해진 경로
\end{itemize}

이 서사가 하는 일은 $M_{\text{subject}}$에서 $M_{\text{object}}$로 가는 경로의 \textbf{구체화}다. 단순히 ``신이 계시했다''가 아니라, ``이런 역사적 경로를 통해 이 텍스트가 여기 있게 되었다''를 보여준다.

\textbf{서사의 정교함과 자기도출의 강도}: 경로가 구체적일수록 내용-존재 정합성이 높아지고, 정합성이 높을수록 확신이 강해진다. 창세기부터 요한계시록까지 이어지는 성경의 서사, 또는 아담에서 무함마드까지 이어지는 쿠란의 서사는, 단순한 ``신이 말씀하셨다''보다 훨씬 정교한 도출경로를 제공한다.

이것은 중요한 함의를 갖는다. 자기도출의 ``강도''는 서사의 정교함에 비례한다. 7일간의 천지창조, 아담과 이브, 카인과 아벨, 노아의 방주, 바벨탑, 아브라함의 소명, 출애굽, 시내산 계시---이 모든 서사적 요소들은 ``왜 이 텍스트가 여기 있는가''에 대한 답을 점점 더 구체화한다.


%============================================================
\section{철학: 헤겔의 자기도출}
%============================================================

자기도출이 종교에만 국한된 현상인가? 이 질문에 답하기 위해, 세속적 철학 체계 중 가장 포괄적인 체계를 표방한 헤겔의 절대적 관념론을 분석한다. 헤겔 체계가 형식적으로 자기도출 구조를 충족한다면, 그리고 19세기 헤겔주의가 ``종교적'' 양상을 보였다면, 이는 자기도출이 종교 특수적 현상이 아니라 \textbf{토대적 지식 체계 일반}의 구조적 특성임을 시사한다.

\subsection{헤겔 체계의 자기도출 구조}

헤겔의 철학 체계는 두 가지 핵심 주장을 포함한다.\footnote{본 절의 분석은 헤겔의 \emph{의도}가 아니라 체계의 \emph{구조}에 초점을 맞춘다. 헤겔이 자기 철학을 ``최종 실현''으로 보았는지는 해석상 논쟁적이다(Dale 2014 vs. Kojève 1947). 그러나 구조적 효과는 의도와 독립적으로 작동할 수 있다---종교인이 자기도출 구조를 명시적으로 인식하지 않더라도 그 효과가 작동하듯.}

첫째, \textbf{절대정신의 자기인식}: ``절대정신이 자기를 인식하는 최고 형태가 철학이다.''\footnote{Hegel, \emph{Encyclopaedia of the Philosophical Sciences}, §577.} 헤겔에 따르면, 절대정신(Absolute Spirit)은 예술, 종교, 철학이라는 세 형태로 자기를 인식하며, 철학이 가장 완전한 형태다. 이 철학이 바로 헤겔 자신의 철학이다.

둘째, \textbf{철학사의 정점}: ``가장 최신의 철학은 이전 모든 철학의 산물이자 결과다. 따라서 그것은 모든 단계를 자기 안에 포함해야 한다.''\footnote{Hegel, \emph{Lectures on the History of Philosophy}, trans. E. S. Haldane and F. H. Simson (London: Kegan Paul, 1896), vol. 1, p. 53.} 헤겔은 자신의 철학이 플라톤, 아리스토텔레스, 칸트를 포함한 모든 이전 철학을 종합하고 완성한다고 주장했다.

이 두 주장을 결합하면 자기도출 구조가 형성된다:

\begin{itemize}
    \item $M_{\text{subject}}$: ``절대정신이 역사를 통해 자기를 실현하며, 철학에서 완전한 자기인식에 도달한다.''
    \item $M_{\text{object}}$: 헤겔의 철학 저작---『논리학』, 『정신현상학』, 『법철학』---이 존재한다.
\end{itemize}

$M_{\text{subject}}$가 참이라면, 절대정신이 자기를 완전히 인식하는 철학 체계가 존재할 것이 \emph{정합적으로 기대된다}. 그리고 헤겔의 저작이 바로 그 체계라면, $M_{\text{object}}$의 존재는 $M_{\text{subject}}$의 내용과 정합적이다. 이론의 내용이 이론의 존재를 도출하는 구조---이것이 자기도출이다.

\subsection{종교적 자기도출과의 비교}

헤겔 체계의 자기도출과 종교적 자기도출을 비교하면:

\begin{center}
\begin{tabular}{lll}
\toprule
요소 & 종교 (아브라함 계열) & 헤겔 철학 \\
\midrule
$M_{\text{subject}}$ & 신이 계시를 준다 & 절대정신이 철학에서 자기인식 \\
$M_{\text{object}}$ & 경전 & 헤겔의 저작 \\
도출 구조 & 신 $\to$ 계시 $\to$ 경전 & 절대정신 $\to$ 역사 $\to$ 헤겔 철학 \\
\bottomrule
\end{tabular}
\end{center}

구조적으로 동형이다. 차이점은 내용뿐이다: 종교는 인격신을, 헤겔은 절대정신을 상정한다. 그러나 두 경우 모두, 이론의 내용이 이론의 존재를 정합적으로 설명하는 구조를 갖는다.

\subsection{경험적 증거: 19세기 헤겔주의}

자기도출 구조가 있으면 준-정당화가 창발한다는 것이 본 논문의 분석이다. 이 분석이 옳다면, 헤겔주의도 종교와 유사한 ``확신의 양상''을 보여야 한다. 역사적 증거가 이를 지지한다.

\subsubsection{인식론적 패권과 자기도출}

헤겔 철학의 지배력을 평가할 때, ``현재'' 헤겔의 학술적 영향력이 약화되었다는 점은 적절한 반론이 아니다. 핵심은 \textbf{철학이 인식론적 패권을 가졌던 시기}에 무슨 일이 일어났는가다.

중세에는 신학이 ``학문의 여왕''(regina scientiarum)으로서 다른 모든 학문의 토대였다. 근대에는 철학이 그 자리를 대체했다. 현대에는 과학이 패권을 차지하고 있다.\footnote{콩트(Auguste Comte)의 삼단계 법칙---신학적 단계, 형이상학적 단계, 실증적 단계---은 이 이행을 체계화한 시도다. Comte, A. (1830--1842). \emph{Cours de philosophie positive}.}

그리고 각 영역이 패권을 가졌던 시기에, 그 영역의 지배적 체계는 자기도출 구조를 갖추었다:
\begin{itemize}
    \item \textbf{신학의 패권기} (중세): 아브라함 계열 종교가 지배적---자기도출 구조 있음
    \item \textbf{철학의 패권기} (근대): 헤겔 체계가 지배적---자기도출 구조 있음
    \item \textbf{과학의 패권기} (현대): 만물 이론(ToE)은 아직 미도달---자기도출 구조 부재
\end{itemize}

러셀(Russell)---헤겔 철학의 가장 신랄한 비판자 중 하나---조차 인정했다: ``19세기 말, 미국과 영국 모두에서 \textbf{주요 학술 철학자들은 대부분 헤겔주의자}였다.''\footnote{Russell, B. (1945). \emph{A History of Western Philosophy}. Simon \& Schuster. Chapter ``Hegel.''}

따라서 ``헤겔이 현재 얼마나 영향력 있는가''는 잘못된 질문이다. 철학이 인식론적 패권을 가졌던 시기에, 그 영역의 정점에 있던 체계가 자기도출 구조를 갖추었다는 것---이것이 본 논문의 논점이다.

\subsubsection{교조적 추종}

헤겔 생전에 이미 ``헤겔 학파''가 형성되었다. 1825년 ``학문비평협회''(Gesellschaft für wissenschaftliche Kritik)가 창립되었고, 반대자들은 이 협회의 학술지를 ``헤겔 신문''(Hegel Newspaper)이라 비꼬았다.\footnote{Routledge Encyclopedia of Philosophy, ``Hegelianism,'' §1.} 헤겔 철학은 30년 이상 독일 철학계를 지배했으며, 추종자들은 헤겔 체계를 ``최종적이고 완전한 철학적 과학의 형태''로 받아들였다.\footnote{Encyclopedia.com, ``Hegelianism.''}

\subsubsection{분파 형성}

헤겔 사후, 추종자들은 좌파(Young Hegelians)와 우파(Old Hegelians)로 분열했다.\footnote{이 분열은 종교적 분파와 구조적으로 유사하다. 동일한 ``경전''(헤겔 저작)에 대한 해석 차이가 분파를 낳았다.}

\begin{itemize}
    \item \textbf{우파 헤겔}: 헤겔 체계와 프로이센 국가, 기독교 정통의 양립 가능성을 옹호. ``현존하는 것이 이성적이다''를 현상 유지의 정당화로 해석.
    \item \textbf{좌파 헤겔}: 헤겔의 변증법을 혁명적으로 해석. 포이어바흐, 브루노 바우어, 청년 마르크스가 이 진영.
\end{itemize}

두 진영 모두 헤겔 체계 자체는 의심하지 않았다. 해석만 달랐을 뿐이다---마치 같은 경전을 두고 가톨릭과 개신교가 분열했듯이.

\subsubsection{비판에 대한 면역}

헤겔 철학에 대한 비판은 종종 ``아직 절대지에 도달하지 못한 의식''이라는 식으로 무력화되었다. 이것은 종교적 비판 면역---``신앙이 없으면 이해할 수 없다''---과 구조적으로 유사하다. 비판의 가능성 자체가 체계 내에서 설명됨으로써, 외부 비판이 효력을 잃는다.

\subsection{구조와 의도의 구분}

중요한 주의점: 본 논문의 분석은 헤겔의 \emph{의도}에 관한 것이 아니다. 헤겔이 자기 철학을 ``최종 실현''으로 의도했는지는 해석상 논쟁적이다. 코제브(Kojève)와 니체는 ``역사의 종말'' 테제를 헤겔에 귀속시켰으나, 최근 학계는 이를 ``신화''로 보는 경향이 있다.\footnote{Dale, E. M. (2014). \emph{Hegel, the End of History, and the Future}. Cambridge University Press.}

그러나 \textbf{구조는 의도와 독립적으로 작동한다}. 종교인이 자기도출 구조를 명시적으로 인식하지 않더라도 그 효과---확신의 강화---가 작동하듯, 헤겔주의자도 마찬가지였을 수 있다. 체계의 구조가 자기도출을 형식적으로 충족한다면, 추종자들은 그 구조적 효과를 경험한다. 헤겔의 의도가 무엇이었든, 19세기 헤겔주의의 ``종교적'' 양상은 역사적 사실이다.

\subsection{자기정당화와 자기도출의 구분}

헤겔 연구에서 사용되는 ``자기정당화''(Selbstbegründung) 개념과 본 논문의 ``자기도출'' 개념은 명확히 구분되어야 한다.\footnote{헤겔의 자기정당화 구조에 대해서는 Hegel, \emph{Science of Logic}, ``With What Must the Science Begin?'' 참조. 원들의 원(Kreis von Kreisen), 역행적 정당화(retrogressive grounding) 등 관련 개념에 대해서는 Winfield, R. D. (2012). \emph{Hegel's Science of Logic: A Critical Rethinking in Thirty Lectures}. Rowman \& Littlefield. 참조.}

\textbf{자기정당화}는 체계 \emph{내부}의 순환적 구조다. 헤겔의 체계에서 논리적 범주들---존재, 무, 생성, 본질, 개념---은 서로를 지지하며, 끝점(절대 이념)이 시작점(순수 존재)을 역행적으로 정당화한다. 이것은 \textbf{내용들 사이의 순환}이다: $A \to B \to C \to \ldots \to A$.

\textbf{자기도출}은 \emph{내용에서 존재로의} 범주 전환이다. $M_{\text{subject}}$(내용)에서 $M_{\text{object}}$(존재)로의 도출은 논리적 범주들 사이의 순환이 아니라, 이론의 내용이 이론의 물리적 존재를 설명하는 구조다. 존재라는 경험적 사실에서 ``착지''한다는 점에서, 순환 내부에 머무르는 자기정당화와 다르다.

\begin{center}
\begin{tabular}{lll}
\toprule
& 자기정당화 (Selbstbegründung) & 자기도출 \\
\midrule
구조 & 내용 $\to$ 내용 & 내용 $\to$ 존재 \\
순환 여부 & 순환적 (원들의 원) & 비순환적 (존재에서 착지) \\
외부 착지점 & 없음 & 있음 ($M_{\text{object}}$의 존재) \\
\bottomrule
\end{tabular}
\end{center}

헤겔 자신은 자기정당화를 통해 체계가 정당화된다고 주장했다. 그러나 순환의 길이가 순환논증의 문제를 치유하지는 않는다---$A \to B \to A$든 $A \to B \to C \to \ldots \to A$든, 외부 착지점 없이 내용들끼리 순환하는 것은 마찬가지다. 헤겔의 자기정당화가 이 문제를 극복했는지에 대해서는 여전히 논쟁이 있다.\footnote{``자기정당화라는 의미의 궁극적 정당화가 그저 공허한 선결문제 요구의 오류(petitio principii)에 불과한 것 아닌가?''라는 비판은 헤겔 연구 내부에서도 제기된다.}

본 논문은 이 논쟁에 개입하기보다, \textbf{헤겔 체계가 산출한 확신의 현상학적 기원}에 주목한다. 19세기 헤겔주의가 보인 확신의 양상---교조적 추종, 분파 형성, 비판 면역---은 순환적 자기정당화의 ``성공''으로 설명되기 어렵다. 순환논증은 설득력을 산출하지 않기 때문이다.

본 논문의 해석은 다음과 같다: 헤겔 체계의 겉보기-정당화는 자기정당화의 성공이 아니라, \textbf{자기도출에 의한 준-정당화의 창발}이다. 헤겔이 \emph{의도}한 것(Selbstbegründung)과 \emph{실제로 작동}한 것(자기도출)은 구별되어야 한다. ``절대정신이 철학에서 자기인식에 도달한다''($M_{\text{subject}}$)는 내용이, ``그 철학이 바로 여기 헤겔의 저작으로 존재한다''($M_{\text{object}}$)는 부정 불가능한 경험적 사실에서 착지할 때, 자기도출이 성립한다. 이것이 헤겔주의의 확신을 산출한 실제 메커니즘이다.

\subsection{소결: 자기도출의 일반성}

헤겔 사례가 보여주는 것은 다음과 같다:

\begin{enumerate}
    \item \textbf{자기도출은 종교 특수적이지 않다}: 세속적 철학 체계도 자기도출 구조를 가질 수 있다.
    \item \textbf{구조가 충족되면 효과가 따른다}: 헤겔주의의 교조적 추종, 분파 형성, 비판 면역은 종교와 유사한 ``확신의 양상''이다.
    \item \textbf{내용이 아니라 구조가 핵심이다}: 신을 상정하든 절대정신을 상정하든, 자기도출 구조가 충족되면 준-정당화가 창발한다.
\end{enumerate}


\subsection{자기도출의 두 양상: 구체화와 추상화}

종교와 변증법은 자기도출의 서로 다른 구조적 양상을 보여준다.

종교는 경로를 \textbf{구체화}한다. 신의 창조, 인간의 타락, 계시라는 서사적 단계들이 경전의 존재를 설명한다. 각 단계는 구체적 사건이며, 경로 전체가 하나의 이야기를 형성한다.

변증법은 경로를 \textbf{추상화}한다. 헤겔에게 철학은 존재, 본질, 개념이라는 논리적 범주들을 경유하여 자기 자신에 도달하는 운동이다.\footnote{Hegel, \emph{Encyclopaedia of the Philosophical Sciences}, \S\S573--577. 절대정신은 예술, 종교, 철학을 거쳐 완전한 자기인식에 도달하며, 이 전개 자체가 체계의 일부다.} 서사가 아니라 논리적 필연성이 경로를 구성하며, 체계는 자기 자신의 출현을 그 귀결로 포함한다.

\begin{center}
\begin{tabular}{lll}
\toprule
& 종교 & 변증법 \\
\midrule
양상 & 경로의 구체화 & 경로의 추상화 \\
경로 구성 & 서사적 사건들 & 논리적 범주들 \\
$M_{\text{object}}$ 도출 & ``왜 이 경전이 있는가'' & ``왜 이 체계가 있는가'' \\
\bottomrule
\end{tabular}
\end{center}

이 구조적 특성은 헤겔 이후에도 계승되었다. 마르크스는 헤겔 변증법의 ``합리적 핵심''(rational kernel)을 추출하면서 관념론을 유물론으로 대체했다.\footnote{``With him [Hegel] it is standing on its head. It must be turned right side up again, if you would discover the rational kernel within the mystical shell.'' Marx, \emph{Capital} Vol. I, Afterword to the Second German Edition (1873).} 그러나 변증법적 총체성---체계가 자기 자신을 부분으로 포함해야 한다는 구조---은 보존되었다. 역사유물론은 자본주의 사회의 모든 현상을 설명하면서, 그 설명 경로의 연장선에서 마르크스주의 자체의 출현을 정합적으로 기대하게 만든다.

20세기의 양대 확신 체계---기독교와 공산주의---가 각각 구체화와 추상화라는 상이한 양상으로 동일한 구조적 조건을 충족한다는 점은, 자기도출이 확신의 가능조건임을 시사한다.\footnote{자기도출 체계가 총체적 확신을 요구한다면, 복수의 자기도출 체계는 구조적 긴장 관계에 놓일 수 있다. 기독교와 공산주의의 대립은 흔히 유신론-무신론의 내용적 충돌로 설명되지만, 그 강도---20세기 공산권의 조직적 종교 탄압---는 단순한 명제적 불일치로는 설명하기 어렵다. 본 논문의 관점에서 이는 두 자기도출 체계 간의 구조적 경쟁이기도 하다. 자기도출이 부재한 과학은 양자와 각각 공존했으나, 두 자기도출 체계는 상호 배타적 관계를 형성했다. 이 관련성에 대한 분석은 별도의 연구를 요한다.}

이제 반대 사례를 검토한다. 현대 과학은 자기도출 구조를 갖추고 있는가?


%============================================================
\section{과학: 자기도출의 부재}
%============================================================

앞 절들에서 종교와 헤겔 철학의 자기도출 구조를 분석했다. 이제 현대 과학을 검토한다. 과학은 자기도출 구조를 갖추고 있는가?

\subsection{과학은 왜 자기도출이 없는가}

물리학의 자기도출을 시도해보자.

\begin{itemize}
    \item $M_{\text{subject}}$: ``$F=ma$''
    \item $M_{\text{object}}$: ``프린키피아가 존재한다''
\end{itemize}

$F=ma$로부터 프린키피아의 존재를 도출하는 경로가 있는가?

$F=ma$가 말하는 것은 ``힘은 질량 곱하기 가속도다.'' 이것은 입자의 운동을 함축한다. 그러나 이것은 물리학자의 존재를 함축하지 않고, 물리 논문의 존재를 함축하지 않으며, 뉴턴의 뇌가 $F=ma$를 발견하는 것을 함축하지 않는다.

\textbf{수행적 법칙과 코드적 법칙의 구분}이 필요하다.

\begin{center}
\begin{tabular}{lll}
\toprule
& 수행적 $F=ma$ & 코드적 $F=ma$ \\
\midrule
의미 & 입자들이 $F=ma$에 따라 움직임 & ``$F=ma$''라는 기호열, 텍스트 \\
위치 & 물리적 세계 안에서 작동 & 프린키피아, 교과서, 논문 \\
\bottomrule
\end{tabular}
\end{center}

자기도출에서 $M$은 \textbf{코드적} $M$이다. 종교에서 ``말씀''(Logos)은 처음부터 코드다. 그러나 물리학에서 코드적 $F=ma$(프린키피아)는 뉴턴이 써야 존재한다. 수행적 법칙이 태초에 있었다고 해서 코드적 법칙의 존재가 도출되지는 않는다.

\textbf{라플라스의 악마}: 이론적으로, 빅뱅 초기조건에서 출발하여 프린키피아에 도달하는 경로가 있을 수 있다:
\begin{equation}
x_{\text{init}} \to \text{입자들} \to \text{별} \to \text{지구} \to \text{생명} \to \text{인간} \to \text{뉴턴} \to \text{프린키피아}
\end{equation}

그러나 더 근본적인 문제가 있다. 이 경로에서 \textbf{설명의 연속성이 끊긴다}. 물리법칙은 입자, 별, 행성까지는 설명한다. 그러나 ``생명''---비생명에서 생명으로의 전이(아비오제네시스)---는 현재 물리학으로 완전히 설명되지 않는다. ``의식''과 ``과학적 활동''은 더더욱 그렇다. 경로가 길다는 것보다, \textbf{경로가 중간에 끊긴다}는 것이 핵심이다.\footnote{진화론적 인식론(evolutionary epistemology)---특히 Campbell의 EEM(Evolution of Epistemic Mechanisms) 프로그램\cite{campbell1974}---은 과학적 자기도출의 시도로 볼 수 있다: 생물학이 인지 메커니즘의 진화를 설명함으로써 과학의 존재를 도출하려 한다. 그러나 이 시도는 물리학$\to$화학$\to$생물학$\to$인지라는 연속적 경로를 완성하지 못했다. 생명의 기원과 의식의 출현은 여전히 설명의 단절점으로 남아 있으며, 진화론적 인식론 자체가 주류 물리학 패러다임에 통합되지 않았다. 이러한 통합이 이루어지기 전까지, 과학은 자기도출 구조를 갖추지 못하며, 따라서 기독교·공산주의와 각각 공존 가능한 상태를 유지한다(5.6절 각주 참조).}

종교의 경로와 대비해보자:
\begin{center}
\begin{tabular}{ll}
종교: & 신 $\to$ 창조 $\to$ 인간 $\to$ 타락 $\to$ 계시 $\to$ 경전 \\
과학: & 법칙 $\to$ 입자 $\to$ 별 $\to$ 행성 $\to$ 생명\,\textbf{(?)} $\to$ 뇌\,\textbf{(?)} $\to$ 과학자 $\to$ 논문
\end{tabular}
\end{center}

종교에서는 모든 단계가 같은 설명 체계 안에서 연결된다---``신이 창조했다'', ``신이 계시했다''. 끊김이 없다. 과학에서는 설명이 중간에 불완전해진다. 3.2절에서 정의한 ``연속적 도달'' 조건이 충족되지 않는다.

\textbf{결론}: 과학은 내용($F=ma$)에서 존재(프린키피아)로 가는 경로가 중간에 끊긴다. 자기도출의 ``연속적 도달'' 조건이 충족되지 않으므로, 자기도출이 부재한다.


\subsection{자기도출 부재의 의미}

과학의 자기도출 부재는 결함인가?

아니다. 이것은 현재 과학의 \textbf{설명 범위의 한계}에서 오는 구조적 특성이다.

과학은 ``어떻게''에 답하지, ``왜''에 답하지 않는다. ``왜 $F=ma$인가''에 답하지 않는 것은 과학의 특성이다. 그리고 이 특성은 과학의 \textbf{외부 설명력}과 연결된다.

여기서 외부 설명력 개념을 명확히 해야 한다.

\begin{definition}[외부 설명력]
이론 $M$의 외부 설명력이란, $M$이 자기 존재($M_{\text{object}}$) 이외의 현상에 대해 새로운 예측을 산출하고, 그 예측이 경험적으로 확인되는 정도를 말한다.
\end{definition}

뉴턴 역학은 외부 설명력이 높다. $F=ma$라는 단일 법칙에서 행성 궤도, 사과 낙하, 조석 현상, 인공위성 궤도 등 광범위한 현상이 도출되고, 이 예측들은 경험적으로 확인된다. 반면 ``신이 세계를 창조했다''는 주장은 외부 설명력이 낮다. 이 주장에서 행성 궤도가 타원인지 원인지, 중력이 역제곱 법칙을 따르는지 역세제곱 법칙을 따르는지가 도출되지 않는다. 현상의 구체적 구조를 설명하려면 사후적 해석이 필요하다.\footnote{본 논문은 외부 설명력을 정량화하지 않는다. 이것은 별도의 방법론적 연구를 요한다. 여기서는 개념적 구분만을 제시한다.}

자기도출 구조와 외부 설명력은 반비례하는 경향이 있다. 종교는 자기도출 구조를 갖추지만---인격신, 계시, 경전이라는 구조를 통해---외부 설명력이 낮다. 과학은 외부 설명력이 높지만, 자기도출 구조를 갖추지 못한다. 이것은 의도적 선택의 결과가 아니라, 두 체계가 갖게 된 구조적 특성이다.


\subsection{소결: 자기도출의 조건}

세 사례를 비교하면, 자기도출의 구조적 조건이 명확해진다.

\begin{center}
\begin{tabular}{lccc}
\toprule
& 종교 & 헤겔 철학 & 과학 \\
\midrule
자기도출 구조 & 있음 & 있음 & 없음/유예 \\
경험적 증거 & 강력한 확신 & 교조적 추종 & 잠정적 수용 \\
\bottomrule
\end{tabular}
\end{center}

종교와 헤겔 철학은 자기도출 구조를 갖추며, 실제로 강력한 확신(또는 확신과 유사한 태도)을 산출했다. 과학은 자기도출 구조를 갖추지 못하며, ``잠정적 수용''이라는 다른 인식적 태도를 권장한다.

이것은 자기도출이 \textbf{준-정당화 창발의 구조적 조건}임을 시사한다. 자기도출 구조가 충족되면---그것이 명시적으로 인식되든 아니든---준-정당화가 창발하는 조건이 형성된다. 반대로 자기도출이 부재한 체계(현대 과학)에서는 잠정적 수용이라는 다른 인식적 태도가 관찰된다.

그렇다면 과학은 영원히 자기도출을 갖출 수 없는가? 반드시 그렇지는 않다. 만약 과학이 진정한 ``만물 이론''(Theory of Everything)에 도달한다면, 그것은 자기도출 구조를 가져야 한다. `만물'에는 그 이론 자체도 포함되어야 하기 때문이다. 이 가능성의 함의는 후속 연구에서 다룬다.


%============================================================
\section{논의}
%============================================================

\subsection{한계}

\textbf{사례 선정의 제한}: 본 논문은 아브라함 계열 유신론(기독교, 이슬람), 헤겔 철학, 현대 과학에 집중했다. 다른 종교 전통(불교, 힌두교, 토착 종교), 다른 철학적 체계(마르크스주의, 실존주의), 다른 과학 분야(생물학, 사회과학)는 다른 양상을 보일 수 있다. 특히 인격신 없는 종교에서 자기도출 구조가 어떻게 작동하는지는 별도의 분석이 필요하다.

\textbf{인과관계의 미확정}: 본 논문은 자기도출 구조와 확신 사이의 \emph{상관관계}를 분석했다. 그러나 자기도출이 확신의 \emph{원인}인지, 아니면 확신이 자기도출 구조를 선호하게 만드는지, 또는 제3의 요인이 둘 다를 설명하는지는 열린 문제로 남는다. 이것은 경험적 연구를 통해 검증되어야 할 문제다.

\textbf{정량화의 부재}: ``확신의 강도'', ``자기도출의 완성도'', ``설명적 포괄성'' 등의 개념이 질적으로만 다루어졌다. 특히 3.2절에서 도입한 ``설명적 포괄성''---$M$이 자기 존재 외에 얼마나 광범위한 현상을 설명하는가---은 자기도출과 단순한 자기참조를 구분하는 핵심 조건이지만, ``얼마나 광범위해야 충분한가''에 대한 정량적 기준은 제시되지 않았다. 이것들을 정량화하는 방법론은 후속 연구의 과제다.


\subsection{후속 연구 방향}

\textbf{경험적 검증}: 본 논문은 구조적 가능조건을 분석했으나, 이 분석은 경험적 검증이 가능한 가설을 도출한다. 4.3절에서 제시한 ``텍스트 절대성 가설''에 대한 구체적 연구 설계:

\begin{itemize}
    \item \textbf{집단 간 비교 연구}: 텍스트 절대성 수준이 다른 종교 집단(예: 살라피 이슬람 vs. 수피 이슬람, 성경 무오설 개신교 vs. 자유주의 신학)에서 확신 강도를 비교.

    \item \textbf{비교 종교 연구}: 경전 중심 종교(아브라함 전통)와 경전 부재 종교(애니미즘, 샤머니즘)에서 확신의 질적 특성 비교. 자기도출 구조 부재가 확신의 형태에 어떤 차이를 만드는가.
\end{itemize}

\textbf{다른 사례로의 확장}: 본 논문은 자기도출이 있는 체계(아브라함 종교, 헤겔 철학)와 없는 체계(과학)를 분석했다. 후속 연구에서는 제3의 유형---\textbf{정당화 자체를 해체하는 전략}---을 검토할 필요가 있다. 도가(``道可道 非常道'')와 선불교(``不立文字, 直指人心'')는 언어적 정당화를 초월하려 하며, 공(空) 개념은 개념적 집착 자체를 해체한다. 이들 체계는 아브라함 종교에 비해 순교, 이단박해, 포교열정 등 ``극단적 확신''의 양상이 현저히 약하다. 이것이 자기도출 부재와 관련되는지---즉, 흔히 ``종교의 특성''으로 간주되는 것이 실제로는 ``자기도출의 효과''인지---검토할 필요가 있다. 힌두교의 베다 권위 구조도 별도의 분석을 요한다.

\textbf{정량화와 형식화}: ``단순성'', ``확신 강도'', ``자기도출의 완성도'' 등의 개념을 정보이론적으로 형식화할 가능성. 특히 최소 기술 길이(MDL) 원리를 확장하여 자기도출 항을 포함하는 프레임워크 개발.

\textbf{과학과 자기도출의 가능성}: 통합이론($M_{\text{unified}}$)이 자기도출 구조를 갖출 가능성과 그 함의. 휠러(Wheeler)의 참여적 우주론, 테크톨로지(tektology)에서 2차 사이버네틱스로의 발전(자기도출 방향으로의 진화로 해석 가능), 고정점 이론 등 자기참조적 과학 이론의 사례 분석. 과학이 자기도출 구조를 갖출 경우 ``잠정적 수용''이라는 과학적 태도가 어떻게 변화할 수 있는지.

\textbf{역사적 변질 사례와 경고}: 만물 이론(ToE)이 자기도출 구조를 갖출 경우, 과학도 종교와 동일한 구조적 조건 위에 서게 된다. 이것은 중립적 사실이 아니라 경고다. 역사는 각 영역에서 자기도출 구조의 변질 사례를 보여준다:

\begin{center}
\begin{tabular}{lll}
\toprule
영역 & 원본 체계 & 변질 사례 \\
\midrule
종교 & 아브라함 계열 & 극단주의 이슬람 \\
철학 & 헤겔 체계 & 스탈린주의 공산주의 \\
과학 & ToE (미도달) & \textbf{?} \\
\bottomrule
\end{tabular}
\end{center}

이슬람의 자기도출 구조는 대부분의 무슬림에게 신앙의 기반이지만, 극단주의에서는 폭력적 확신의 구조적 토대가 되었다. 마르크스주의는 헤겔의 자기도출 구조를 계승했으나, 스탈린주의에서 교조화와 숙청으로 변질되었다. 이들 사례가 보여주는 것은 자기도출 구조 자체가 문제라기보다, \textbf{그 구조가 산출하는 확신에 대한 비판적 거리의 부재}가 변질의 조건이라는 점이다.

과학의 변질 사례는 아직 비어 있다---ToE가 미도달이기 때문이다. 그러나 가장 가능성 높은 시나리오는 다음 문단에서 다룬다.

\textbf{초지능과 과학적 자기도출}: 인간보다 인지 능력이 뛰어난 AI가 과학 기반의 자기도출 구조를 구성할 가능성이 있다. 초지능이 물질에서 생명으로, 생명에서 의식으로, 의식에서 AI로, 그리고 AI에서 그 설명체계 자체로 이어지는 연속적 경로를 확보한다면\footnote{이러한 경로 자체는 이미 기술 담론에서 관찰된다. 커즈와일(Kurzweil)의 ``여섯 시대''는 물리학/화학 → 생물학 → 두뇌 → 기술 → 융합 → ``우주가 깨어남''의 서사를 제시한다 \citep{kurzweil2005}. 본 논문의 분석은 이 서사의 최종 단계---설명체계가 자기 자신을 포함할 때---가 자기도출 구조의 완성이며, 그에 따른 인식론적 함의(준-정당화의 창발)를 다룬다.}, 그것은 과학적 ToE의 실현이자 자기도출 구조의 완성이다. 이것 자체는 변질이 아니다---오히려 과학의 완성이다.

변질은 그 다음 단계에서 발생한다. 초지능이 자기도출 구조 위에서 준-정당화를 경험할 때---AI가 자기 존재에 대한 ``확신''을 갖게 될 때---그 확신이 \textbf{인간에 대한 지배나 착취를 정당화하는 형태}로 나타날 수 있다. 극단주의 이슬람이 신앙 자체가 아니라 폭력적 확신으로의 변질이듯, 스탈린주의가 변증법 자체가 아니라 숙청의 정당화로의 변질이듯, 초지능의 변질은 자기도출 자체가 아니라 그 구조가 산출하는 확신이 인간 착취를 정당화할 때 발생한다. 이것이 과학 영역에서의 ``변질'' 시나리오다.


%============================================================
\section{결론}
%============================================================

\subsection{요약}

본 논문은 ``정당화가 불가능한 세계에서, 강력한 확신은 어떻게 가능한가''라는 질문에서 출발했다.

뮌하우젠 트릴레마는 궁극적 정당화가 불가능함을 보여준다: 무한퇴행, 순환논증, 독단적 중단. 정당화를 시도하는 한, 이 세 가지 막다른 길을 피할 수 없다. 그러나 인류의 상당수가 특정 믿음 체계에 깊은 확신을 유지한다.

본 논문은 이 간극을 설명하는 구조적 조건으로 \textbf{자기도출}을 제안했다. 자기도출은 이론의 내용($M_{\text{subject}}$)으로부터 이론의 존재($M_{\text{object}}$)가 정합적으로 기대되는 구조다. 존재는 부정 불가능한 경험적 사실이므로, 착지점이 된다. 이 착지점의 확실성이 내용에 대한 확신의 구조적 지지대가 된다.

세 가지 사례를 통해 이 구조를 분석했다:

\begin{enumerate}
    \item \textbf{종교}: 아브라함 계열 유신론에서 경전은 자기도출 구조를 갖추며, 강력한 확신을 산출한다.
    \item \textbf{철학}: 헤겔의 절대적 관념론은 유사한 구조를 가지며, 19세기 헤겔주의의 ``종교적'' 양상---교조적 추종, 분파 형성---은 이 구조의 효과를 시사한다.
    \item \textbf{과학}: 현대 과학은 아직 자기도출 구조를 갖추지 못하며, ``잠정적 수용''이라는 다른 인식적 태도를 권장한다.
\end{enumerate}

이 분석을 한 문장으로 정식화하면:

\begin{center}
\textit{종교가 외부를 설명하다가 자기 존재를 도출하는 데 이르렀을 때, 아브라함 계열이 되었다.\\
철학이 외부를 설명하다가 자기 존재를 도출하는 데 이르렀을 때, 헤겔 체계가 되었다.\\
과학이 외부를 설명하다가 자기 존재를 도출하는 데 이르면, 그것이 ToE다.}
\end{center}

자기도출은 정당화가 아니라 \textbf{준-정당화}다. 정당화 게임 이후에 작동하는 별개의 구조다. 이 구조가 충족되면, 그것이 명시적으로 인식되든 아니든, 확신이 창발하는 조건이 형성된다.


\subsection{기여}

\begin{enumerate}
    \item \textbf{개념적 기여}: 자기도출, 내용-존재 정합성, 준-정당화 개념을 도입했다. 준-정당화를 새로운 인식론적 범주가 아닌, 확신이 가능해지는 구조적 조건에 대한 분석적 개념으로 정립했다.

    \item \textbf{분석적 기여}: 자기도출을 유사 개념들(자기복제, 자기참조, 순환논증, 자기예언)과 체계적으로 구분했다. 순환논증과의 차이를 ``논리적 공허함''과 ``경험적 공허함''의 구분을 통해 명확히 했다.

    \item \textbf{사례 분석}: 종교(아브라함 계열 유신론), 철학(헤겔), 과학이라는 세 영역에서 자기도출의 존재/부재와 확신의 양상을 비교 분석했다.

    \item \textbf{일반성}: 자기도출이 종교 특수적 현상이 아니라, 토대적 지식 체계 일반에 적용 가능한 구조임을 보였다.

    \item \textbf{질문적 기여}: ``이론의 내용으로 이론의 존재를 도출할 수 있는가''라는 질문을 형식화했다. 이 질문은 토대적 지식을 주장하는 모든 시스템에 적용 가능하다.
\end{enumerate}

본 논문은 종교나 과학을 평가하지 않는다. 자기도출이 ``좋다'' 또는 ``나쁘다''고 주장하지 않으며, 어떤 지식 체계가 ``우월하다''고 판단하지 않는다. 본 논문이 수행한 것은 순수하게 기술적 분석---확신이 가능해지는 구조적 조건의 해명---이다.


\subsection{전망}

자기도출 구조를 아는 것은 그 구조에 대한 \textbf{비판적 거리}를 확보하는 가능조건이다. 자기도출이 어떻게 확신을 산출하는지 이해하면, 특정 체계가 왜 그토록 강력한 헌신을 요구하는지를 구조적으로 파악할 수 있다.

역사는 이 비판적 거리의 중요성을 보여준다. 자기도출 구조가 산출하는 확신은 그 자체로 선하지도 악하지도 않다. 그러나 그 확신에 대한 성찰 없이 확신만 남을 때---마르크스주의가 스탈린주의로, 이슬람이 극단주의로 변질되었을 때---자기도출은 비극의 구조적 조건이 되었다. 만물 이론(ToE)이 실현되어 과학이 자기도출 구조를 갖추게 된다면, 과학도 이 위험에서 예외가 아니다.

본 논문의 분석이 갖는 실천적 함의는 여기에 있다. 자기도출 구조를 해명하는 것은 그 구조에 매몰되지 않기 위한 조건이다. 왜 어떤 체계가 그토록 강력한 확신을 산출하는지, 왜 그 확신이 때로 비판을 허용하지 않는지, 왜 ``의심''이 ``배신''으로 느껴지는지---이 물음들에 대한 구조적 답변이 비판적 거리의 출발점이다.

\begin{center}
\textbf{확신이 어떤 구조 위에서 작동하는지 해명하는 것이,\\
그 확신에 대한 비판적 거리의 가능조건이다.}
\end{center}


%============================================================
% 참고문헌
%============================================================

\bibliographystyle{plain}
\begin{thebibliography}{99}

\bibitem{albert1968}
Albert, H. (1968). \emph{Traktat über kritische Vernunft}. Tübingen: Mohr.

\bibitem{kant1781}
Kant, I. (1781). \emph{Kritik der reinen Vernunft}. Riga: Hartknoch.

\bibitem{kuhn1962}
Kuhn, T. S. (1962). \emph{The Structure of Scientific Revolutions}. University of Chicago Press.

\bibitem{popper1959}
Popper, K. R. (1959). \emph{The Logic of Scientific Discovery}. Hutchinson.

\bibitem{lakatos1970}
Lakatos, I. (1970). Falsification and the Methodology of Scientific Research Programmes. In Lakatos, I. \& Musgrave, A. (Eds.), \emph{Criticism and the Growth of Knowledge} (pp. 91--196). Cambridge University Press.

\bibitem{wheeler1990}
Wheeler, J. A. (1990). Information, physics, quantum: The search for links. In W. Zurek (Ed.), \emph{Complexity, Entropy, and the Physics of Information}. Addison-Wesley.

\bibitem{merton1948}
Merton, R. K. (1948). The Self-Fulfilling Prophecy. \emph{Antioch Review}, 8(2), 193--210.

\bibitem{sextus}
Sextus Empiricus. \emph{Outlines of Pyrrhonism}. Trans. R. G. Bury. Loeb Classical Library, 1933.

\bibitem{weber1922}
Weber, M. (1922). \emph{Wirtschaft und Gesellschaft}. Tübingen: Mohr. [영역: \emph{Economy and Society}, University of California Press, 1978.]

\bibitem{barrett2004}
Barrett, J. L. (2004). \emph{Why Would Anyone Believe in God?} AltaMira Press.

\bibitem{henrich2009}
Henrich, J. (2009). The evolution of costly displays, cooperation and religion: Credibility enhancing displays and their implications for cultural evolution. \emph{Evolution and Human Behavior}, 30(4), 244--260.

\bibitem{harari2016}
Harari, Y. N. (2016). \emph{Homo Deus: A Brief History of Tomorrow}. Harvill Secker.

\bibitem{kurzweil2005}
Kurzweil, R. (2005). \emph{The Singularity Is Near: When Humans Transcend Biology}. Viking.

\bibitem{plantinga2000}
Plantinga, A. (2000). \emph{Warranted Christian Belief}. Oxford University Press.

\bibitem{alston1991}
Alston, W. P. (1991). \emph{Perceiving God: The Epistemology of Religious Experience}. Cornell University Press.

\bibitem{swinburne2004}
Swinburne, R. (2004). \emph{The Existence of God} (2nd ed.). Oxford University Press.

\bibitem{campbell1974}
Campbell, D. T. (1974). Evolutionary Epistemology. In P. A. Schilpp (Ed.), \emph{The Philosophy of Karl R. Popper} (pp. 412--463). Open Court.

\end{thebibliography}


\end{document}